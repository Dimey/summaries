\documentclass[a4paper, 11pt, accentcolor = tud3b]{tudreport}

% Core packages.
\usepackage[T1]{fontenc}
\usepackage[utf8]{inputenc}
\usepackage[ngerman]{babel}
% Other packages.
\usepackage[german, ruled, vlined, linesnumbered]{algorithm2e}
\usepackage{amsthm}
\usepackage{caption}
\usepackage{csquotes}
\usepackage{enumitem}
\usepackage[mathcal]{euscript} % Get readable mathcal font.
\usepackage{float}
\usepackage{hyperref}
\usepackage{mathtools}
\usepackage{siunitx}
\usepackage{qtree}
\usepackage{stmaryrd}
\usepackage{tabto}
\usepackage{tikz}
\usepackage[disable]{todonotes}
\usetikzlibrary{arrows.meta, shapes, backgrounds, angles, calc, decorations.markings}

% Basic information.
\title{Visual Computing}
\subtitle{Zusammenfassung \\ Fabian Damken}
\author{Fabian Damken}
\date{\today}

% Description-list styling.
\SetLabelAlign{parright}{\parbox[t]{\labelwidth}{\raggedleft#1}}
\setlist[description]{style = multiline, leftmargin = 4cm, align = parright}

\tikzset{> = { Latex[length = 2.5mm] }}
\tikzstyle{every path} = [ very thick ]

\MakeOuterQuote{"}

% New commands.
\DeclareMathOperator{\total}{d}
\DeclareMathOperator{\Rang}{Rang}
\DeclareMathOperator{\const}{const}
\DeclareMathOperator{\sign}{sign}
\newcommand{\dif}[1]{\,\total#1}
\newcommand{\N}{\mathbb{N}}
\newcommand{\R}{\mathbb{R}}
\newcommand{\Z}{\mathbb{Z}}
\newcommand{\qef}{\hfill \( \square \)}
% Matrix/Vector notation.
\makeatletter
% TODO: This does not make all symbols bold (e.g. 'v').
\newcommand{\mat}[1]{\boldsymbol{#1}}
% TODO: This adds crazy symbols for symbols that cannot be drawn bold (e.g. \dot{v} becomes \underline{v}).
\renewcommand{\vec}[1]{\boldsymbol{\mathbf{#1}}}
\makeatother
% Abbreviations.

% https://tex.stackexchange.com/a/333383
\makeatletter
\renewcommand*\env@matrix[1][*\c@MaxMatrixCols c]{%
	\hskip -\arraycolsep
	\let\@ifnextchar\new@ifnextchar
	\array{#1}}
\makeatother

\begin{document}
	\maketitle
	\tableofcontents
	\listoftodos

	\chapter{Einführung} % 1.1, 1.10
		\todo{Content}

		\section{Visual Computing} % 1.1, 1.18, 1.30, 1.31, 1.32, 1.33, 1.34, 1.50
			\todo{Content}

			\subsection{3D-Internet} % 1.35, 1.36, 1.37
				\todo{Content}
			% end

			\subsection{Skalierbare Objektmodellierung/-erkennung} % 1.38, 1.39, 1.40, 1.41, 1.42, 1.43
				\todo{Content}
			% end

			\subsection{Big Data, Visual Analytics} % 1.44, 1.45, 1.46
				\todo{Content}
			% end

			\subsection{Scene Understanding} % 1.47, 1.48, 1.49
				\todo{Content}
			% end
		% end

		\section{Generalisierte Dokumente} % 1.51, 1.52, 1.53, 1.54, 1.55, 1.56, 1.68
			\todo{Content}

			\subsection{Retro-Digitalisierung, Digital Creation} % 1.57, 1.58, 1.59
				\todo{Content}
			% end

			\subsection{Generative Modelling Language} % 1.62, 1.63, 1.64, 1.65, 1.66, 1.67
				\todo{Content}
			% end
		% end
	% end

	\chapter{Wahrnehmung} % 2.1
		\todo{Content}

		\section{Human Computer Interaction} % 2.4, 2.5, 2.6, 2.7, 2.8, 2.9
			\todo{Content}
		% end

		\section{Überblick} % 2.10, 2.11, 2.12, 2.13
			\todo{Content}

			\subsection{Menschliche Informationsverarbeitung} % 2.14, 2.15, 2.16, 2.20
				\todo{Content}

				\subsubsection{Eingabe (Wahrnehmung)} % 2.15
					\todo{Content}

					\paragraph{Klangwahrnehmung} % 2.17
						\todo{Content}
					% end

					\paragraph{Brührungswahrnehmung} % 2.18
						\todo{Content}
					% end
				% end

				\subsubsection{Ausgabe} % 2.15, 2.19
					\todo{Content}
				% end
			% end
		% end

		\section{Wahrnehmung} % 2.25
			\todo{Content}

			\subsection{Das Auge} % 2.25
				\todo{Content}

				\subsubsection{Reiz} % 2.26, 2.27, 2.28
					\todo{Content}
				% end

				\subsubsection{Das visuelle System} % 2.29, 2.30, 2.31, 2.32, 2.33, 2.34
					\todo{Content}
				% end

				\subsubsection{Photorezeptoren} % 2.35, 2.36
					\todo{Content}
				% end

				\subsubsection{Skotopisches und Photopisches Sehen} % 2.37
					\todo{Content}
				% end

				\subsubsection{Zapfenverteilung} % 2.38, 2.39, 2.40, 2.41, 2.42, 2.43, 2.44
					\todo{Content}
				% end
			% end

			\subsection{Vorverarbeitung visueller Informationen} % 2.45, 2.46, 2.47
				\todo{Content}

				\subsubsection{Signalverarbeitung in der Retina} % 2.48
					\todo{Content}
				% end

				\subsubsection{Optische Täuschungen} % N/A
					\todo{Content}

					\paragraph{Peripheral Drift} % 2.53
						\todo{Content}
					% end

					\paragraph{Mach Bänder} % 2.55
						\todo{Content}
					% end

					\paragraph{Chevreul Illusion} % 2.57
						\todo{Content}
					% end
				% end

				\subsubsection{Helligkeit} % 2.59, 2.66
					\todo{Content}

					\paragraph{Simultankontrast} % 2.60, 2.61, 2.62, 2.63
						\todo{Content}
					% end

					\paragraph{Kontrast als Reizverhältnis} % 2.64, 2.65
						\todo{Content}
					% end
				% end

				\subsubsection{Erkennung von Details} % 2.67
					\todo{Content}

					\paragraph{Bildschärfenbestimmung} % 2.68
						\todo{Content}
					% end

					\paragraph{Kontrastempfindlichkeit} % 2.69, 2.70, 2.71
						\todo{Content}
					% end
				% end

				\subsubsection{Frühe Wahrnehmung} % 2.73, 2.74, 2.75, 2.76, 2.77, 2.78, 2.79
					\todo{Content}
				% end
			% end

			\subsection{Informationsextraktion} % 2.80, 2.81, 2.82
				\todo{Content}

				\subsubsection{Raumwahrnehmung} % 2.83, 2.84, 2.85, 2.103, 2.104
					\todo{Content}
				% end

				\subsubsection{Depth Cue Theorie} % 2.86
					\todo{Content}

					\paragraph{Stereoskopie} % 2.87, 2.88
						\todo{Content}
					% end

					\paragraph{Picorial Depth Cues} % 2.89
						\todo{Content}

						\subparagraph{Linearperspektive} % 2.90
							\todo{Content}
						% end

						\subparagraph{Texturgradient} % 2.91
							\todo{Content}
						% end

						\subparagraph{Fokus und Blur} % 2.92
							\todo{Content}
						% end

						\subparagraph{Atmosphärische Tiefe} % 2.93
							\todo{Content}
						% end

						\subparagraph{Schattenwurf} % 2.94, 2.95
							\todo{Content}
						% end

						\subparagraph{Vertraute Größen} % 2.96
							\todo{Content}
						% end
					% end

					\paragraph{Dynamische Depth Cues} % 2.97
						\todo{Content}

						\subparagraph{Motion Parallax} % 2.98
							\todo{Content}
						% end

						\subparagraph{Raumwahrnehmung durch Bewegung} % 2.99
							\todo{Content}
						% end

						\subparagraph{Kinetic Depth Effect, Structure from Motion} % 2.100
							\todo{Content}
						% end
					% end

					\paragraph{Auswertung von Depth Cues} % 2.101
						\todo{Content}
					% end
				% end

				\subsubsection{Mögliche Tasks} % 2.102
					\todo{Content}
				% end
			% end
		% end

		\section{Aufmerksamkeit} % 2.105
			\todo{Content}

			\subsection{Veränderungsblindheit} % 2.106
				\todo{Content}
			% end

			\subsection{Limitierung der Wahrnehmung} % 2.115
				\todo{Content}
			% end

			\subsection{Das Gedächtnis und "Gateway to Memory"} % 2.116, 2.117, 2.118
				\todo{Content}
			% end
		% end
	% end

	\chapter{Objekterkennung und Bayes} % 3.1, 3.3, 3.4, 3.6
		\todo{Content}

		\section{Computer Vision} % 3.8
			\todo{Content}

			\subsection{Lochkamera} % 3.9, 3.10, 3.11
				\todo{Content}
			% end

			\subsection{Digitale Bilder} % 3.12, 3.13
				\todo{Content}
			% end

			\subsection{Konfluenz} % 3.14
				\todo{Content}
			% end

			\subsection{Fallstudie} % 3.17 ff.
				\todo{Content}
			% end

			\subsection{Intuitionen} % 3.31
				\todo{Content}
			% end

			\subsection{Bildaufbau} % 3.33
				\todo{Content}
			% end
		% end

		\section{Bayesian Decision Theory} % 3.36, 3.37
			\todo{Content}

			\subsection{Konzepte und Bayes Theorem} % 3.38, 3.39, 3.40, 3.44
				\todo{Content}
			% end

			\subsection{Problemstellung} % 3.41, 3.42, 3.43, 3.45, 3.46
				\todo{Content}
			% end

			\subsection{Entscheidungsregel} % 3.47, 3.48, 3.49, 3.50
				\todo{Content}
			% end

			\subsection{Naive Bayes Classifier} % 3.51, 3.52, 3.53
				\todo{Content}
			% end
		% end

		\section{Probability Density Estimation} % 3.54, 3.55, 3.56, 3.57, 3.58, 3.59
			\todo{Content}
		% end

		\section{Gesichtsdetektion} % 3.60
			\todo{Content}

			\subsection{Appearance-Based Methods} % 3.61
				\todo{Content}
			% end

			\subsection{Sliding Window Approach} % 3.62, 3.63
				\todo{Content}
			% end

			\subsection{Beispiel: Gesichtsdetektion} % 3.64, 3.65, 3.66, 3.67
				\todo{Content}
			% end

			\subsection{Naive Bayes Classifier} % 3.68, 3.69
				\todo{Content}
			% end

			\subsection{Erkennungsarten} % 3.74, 3.75
				\todo{Content}
			% end
		% end

		\section{Diskussion und Ausblick} % 3.76, 3.77, 3.79
			\todo{Content}
		% end
	% end

	\chapter{Fouriertheorie} % 4.1, 4.4, 4.5, 4.6, 4.7, 4.8, 4.9, 4.10, 4.11
		\todo{Content}

		\section{Mathematische Grundlagen} % 4.12
			\todo{Content}

			\subsection{Vektorraum} % 4.13, 4.14
				\todo{Content}
			% end

			\subsection{Basis eines Vektorraums} % 4.15
				\todo{Content}
			% end

			\subsection{Krummlinige Koordinatensysteme} % 4.16, 4.17
				\todo{Content}
			% end

			\subsection{Andere Räume} % 4.18
				\todo{Content}
			% end

			\subsection{Komplexe Zahlen} % 4.20, 4.48
				\todo{Content}
			% end

			\subsection{Gerade/Ungerade Funktionen} % 4.41
				\todo{Content}
			% end
		% end

		\section{Fourier-Reihe} % 4.19, 4.21
			\todo{Content}

			\subsection{Dirichlet-Bedingungen} % 4.22
				\todo{Content}
			% end

			\subsection{\(2\pi\)-periodische Funktion} % 4.23
				\todo{Content}
			% end

			\subsection{Skalarprodukt, Orthogonale Basis} % 4.24, 4.25, 4.26, 4.27, 4.28
				\todo{Content}
			% end

			\subsection{Berechnung der Koeffizienten \(a_n\), \(b_n\)} % 4.29, 4.30, 4.31
				\todo{Content}
			% end

			\subsection{Beispiel: Rechteck-Schwingung} % 4.32, 4.33, 4.34, 4.35, 4.36, 4.37, 4.38, 4.39, 4.40
				\todo{Content}
			% end
		% end

		\section{Fourier-Transformation} % 4.46, 4.47
			\todo{Content}

			\subsection{Mathematische Herleitung} % 4.49, 4.50, 4.51, 4.52
				\todo{Content}
			% end

			\subsection{Zusammenfassung} % 4.53
				\todo{Content}
			% end

			\subsection{Beispiel: Rechteckimpuls} % 4.54, 4.55
				\todo{Content}
			% end

			\subsection{Fourier-Darstellung} % 4.56
				\todo{Content}
			% end

			\subsection{Transformationspaare} % 4.57, 4.58, 4.59
				\todo{Content}
			% end
		% end

		\section{2D-Fourier-Transformation} % 5.56, 5.57, 5.58, 5.59, 5.60, 5.61
			\todo{Content}
		% end

		\section{Faltung} % 4.60, 4.63, 4.64, 4.65
			\todo{Content}

			\subsection{Anwendung: Filter} % 4.68
				\todo{Content}
			% end
		% end

		\section{Abtastung} % 4.69, 4.70, 4.71
			\todo{Content}

			\subsection{Diskrete Abtastung} % 4.72, 4.73
				\todo{Content}
			% end

			\subsection{Abtasttheorie} % 4.74, 4.75
				\todo{Content}
			% end

			\subsection{Abtasttheorem von Whittaker-Shannon} % 4.78
				\todo{Content}
			% end
		% end
	% end

	\chapter{Bilder} % 5.1, 5.3, 5.4
		\todo{Content}

		\section{Bildverbesserung} % 5.8, 5.9, 5.10, 5.11
			\todo{Content}

			\subsection{Histogramm} % 5.14, 5.15, 5.16, 5.17, 5.18, 5.19
				\todo{Content}
			% end

			\subsection{Pixeloperationen} % 5.12, 5.13, 5.20
				\todo{Content}

				\subsubsection{Bildnegativ} % 5.21
					\todo{Content}
				% end

				\subsubsection{Binärisierung/Thresholding} % 5.22
					\todo{Content}
				% end

				\subsubsection{Graufensterung} % 5.23
					\todo{Content}
				% end
			% end

			\subsection{Kontrastspreizung} % 5.24
				\todo{Content}
			% end

			\subsection{Histogrammausgleich} % 5.25, 5.26, 5.27
				\todo{Content}
			% end

			\subsection{Mittelung} % 5.28
				\todo{Content}
			% end
		% end

		\section{Bildfilterung} % 5.29, 5.30
			\todo{Content}

			\subsection{Ortsraum} % 5.31
				\todo{Content}

				\subsubsection{Lineare Filterung (Faltung)} % 5.32
					\todo{Content}
				% end

				\subsubsection{Tiefpass-Filter} % 5.33
					\todo{Content}
				% end

				\subsubsection{Mittelwert-Filter} % 5.34, 5.35
					\todo{Content}
				% end

				\subsubsection{Gauss-Filter} % 5.36, 5.37, 5.38
					\todo{Content}
				% end

				\subsubsection{Median-Filter} % 5.39
					\todo{Content}
				% end

				\subsubsection{Hochpass-Filter} % 5.40
					\todo{Content}
				% end

				\subsubsection{Diskretisierte Ableitungen} % 5.42, 5.43
					\todo{Content}

					\paragraph{Laplacian-Filter} % 5.44, 5.47, 5.48
						\todo{Content}
					% end

					\paragraph{Laplacian of Gaussian Filter} % 5.45, 5.46
						\todo{Content}
					% end
				% end

				\subsubsection{Bilateral Filter} % 6.10
					\todo{Content}
				% end
			% end

			\subsection{Frequenzraum} % 5.51, 5.52, 5.62, 5.63, 5.64, 5.65
				\todo{Content}

				\subsubsection{Idealer Tiefpass-Filter} % 5.66
					\todo{Content}
				% end

				\subsubsection{Gaussscher Tiefpass-Filter} % 5.67, 5.68
					\todo{Content}
				% end

				\subsubsection{Idealer Hochpass-Filter} % 5.70
					\todo{Content}
				% end
			% end

			\subsection{Vergleich: Orts- und Frequenzraum-Filter} % 5.71
				\todo{Content}
			% end
		% end

		\section{Bildkompression} % 5.72, 5.73, 5.74, 5.75, 5.76, 5.77, 5.96
			\todo{Content}

			\subsection{Harmonische Transformation} % 5.78
				\todo{Content}

				\subsubsection{JPEG} % 5.79, 5.95
					\todo{Content}

					\paragraph{Schritt 1: Umwandlung in den YCC-Farbraum} % 5.80
						\todo{Content}
					% end

					\paragraph{Schritt 2: Farb-Subsampling} % 5.81
						\todo{Content}
					% end

					\paragraph{Schritt 3: Diskrete Kosinustransformation} % 5.82, 5.83, 5.84
						\todo{Content}
					% end

					\paragraph{Schritt 4: Quantisierung} % 5.85
						\todo{Content}
					% end

					\paragraph{Schritt 5: Kodierung der Koeffizienten} % 5.86, 5.87
						\todo{Content}
					% end
				% end

				\subsubsection{MH17} % 6.6, 6.8
					\todo{Content}
				% end
			% end
		% end
	% end

	\chapter{Bildverarbeitung} % 6.1, 6.90
		\todo{Content}

		\section{Deblurring} % 6.13, 6.15, 6.41
			\todo{Content}

			\subsection{Inverser Operator} % 6.19, 6.20
				\todo{Content}
			% end

			\subsection{Rekonstruktion} % 6.17, 6.21
				\todo{Content}
			% end

			\subsection{Rauschen} % 6.23, 6.26
				\todo{Content}
			% end

			\subsection{Korrekt gestellte Probleme} % 6.27, 6.28, 5.29
				\todo{Content}
			% end
		% end

		\section{Einschrittverfahren} % 6.30, 6.41
			\todo{Content}

			\subsection{Wiener Filter} % 6.31, 6.32, 6.33, 6.34, 6.35, 6.36
				\todo{Content}
			% end

			\subsection{Ansatz mit mehreren Komponenten} % 6.37
				\todo{Content}

				\subsubsection{Scale-Space-Ansatz} % 6.38, 6.39, 6.40, 6.43
					\todo{Content}
				% end

				\subsubsection{Gaussscher Scale-Space} % 6.44, 6.45
					\todo{Content}
				% end
			% end
		% end

		\section{Mehrschrittverfahren (Iterative Methoden)} % 6.47, 6.67
			\todo{Content}

			\subsection{Variationsableitung} % 6.48, 6.49, 6.50
				\todo{Content}
			% end

			\subsection{Alternativen} % 6.53
				\todo{Content}
			% end

			\subsection{Perona-Malik} % 6.54
				\todo{Content}

				\subsubsection{Nichtlineare Diffusionsgleichungen} % 6.55
					\todo{Content}
				% end

				\subsubsection{Die Perona-Malik-Gleichung} % 6.56, 6.57, 6.58
					\todo{Content}
				% end

				\subsubsection{Implementierung} % 6.59, 6.60
					\todo{Content}
				% end

				\subsubsection{Einfluss und Wahl von \(k\)} % 6.61
					\todo{Content}
				% end

				\subsubsection{Stoppzeit} % 6.62, 6.63, 6.64
					\todo{Content}
				% end
			% end

			\subsection{Eingeschränkte Evolution: Totale Variation} % 6.68, 6.69
				\todo{Content}

				\subsubsection{Distance Penalty} % 6.69, 6.70, 6.71
					\todo{Content}
				% end

				\subsubsection{Basisenergien} % 6.72
					\todo{Content}
				% end

				\subsubsection{Totale Variation} % 6.73, 6.74
					\todo{Content}
				% end

				\subsubsection{Erweiterungen} % 6.81
					\todo{Content}
				% end
			% end
		% end
	% end

	\chapter{Grafikpipeline} % 7.1
		\todo{Content}

		\section{Hardware} % 7.8, 7.9, 7.14
			\todo{Content}

			\subsection{Large-Scale-Computing} % 7.10
				\todo{Content}
			% end

			\subsection{Personal/Desktop Computing} % 7.11
				\todo{Content}
			% end

			\subsection{Networked Computing} % 7.12
				\todo{Content}
			% end

			\subsection{Mobile Computing} % 7.13
				\todo{Content}
			% end

			\subsection{Collaborative Computing} % 7.15
				\todo{Content}
			% end

			\subsection{Virtual Reality} % 7.16
				\todo{Content}
			% end

			\subsection{Augmented Reality} % 7.18, 7.19, 7.20
				\todo{Content}
			% end

			\subsection{Ambient/Invisible} % 7.21
				\todo{Content}
			% end

			\subsection{Wearable/Ubiquitous} % 7.22
				\todo{Content}
			% end
		% end

		\section{Computergrafik} % 7.23, 7.24, 7.25
			\todo{Content}

			\subsection{Geschichte} % N/A
				\todo{Content}

				\subsubsection{Die Anfänge} % 7.26, 7.27
					\todo{Content}
				% end

				\subsubsection{60er Jahre} % 7.28
					\todo{Content}
				% end

				\subsubsection{70er Jahre} % 7.29
					\todo{Content}
				% end

				\subsubsection{80er Jahre} % 7.31
					\todo{Content}
				% end

				\subsubsection{90er Jahre} % 7.32
					\todo{Content}
				% end

				\subsubsection{2000 bis 2005} % 7.33
					\todo{Content}
				% end

				\subsubsection{2006 bis 2020} % 7.34, 7.35
					\todo{Content}
				% end
			% end

			\subsection{Virtuelle Charaktere} % 7.36, 7.37, 7.38, 7.39
				\todo{Content}
			% end
		% end

		\section{Grafikpipeline} % 7.41, 7.42, 7.43, 7.61, 7.87, 7.98, 7.100
			\todo{Content}
		% end

		\section{Eingabe grafischer Daten} % 7.44, 7.45, 7.46
			\todo{Content}
		% end

		\section{Repräsentation von 3D-Daten} % N/A
			\todo{Content}

			\subsection{Grafische Primitive} % 7.47, 7.48, 7.49, 7.50
				\todo{Content}
			% end

			\subsection{Transformationen} % 7.51, 7.52
				\todo{Content}
			% end
		% end

		\section{Räumliche Datenstrukturen} % 7.53, 7.54, 7.55
			\todo{Content}

			\subsection{Hüllkörper (Bounding Volumes)} % 7.56
				\todo{Content}
			% end

			\subsection{Hüllkörperhierarchien} % 7.57
				\todo{Content}
			% end

			\subsection{Raumunterteilung} % N/A
				\todo{Content}

				\subsubsection{Achsenparallele Gitter (Grids)} % 7.58
					\todo{Content}
				% end

				\subsubsection{Quadtree/Octree} % 7.59
					\todo{Content}
				% end

				\subsubsection{Binary Space Partition} % 7.60
					\todo{Content}
				% end
			% end
		% end

		\section{Geometrieverarbeitung} % N/A
			\todo{Content}

			\subsection{Modelltransformation (Ortientierung, Position)} % 7.62, 7.63
				\todo{Content}
			% end

			\subsection{Simulation der Beleuchtung} % 7.64, 7.65
				\todo{Content}

				\subsubsection{Phong-Beleuchtungsmodell} % 7.66, 7.67
					\todo{Content}
				% end
			% end

			\subsection{Perspektivische Transformation und Clipping (Abschneiden)} % 7.78, 7.79
				\todo{Content}

				\subsubsection{Painters Algorithmus} % 7.80
					\todo{Content}
				% end
			% end

			\subsection{Culling (Verdeckungsrechnung im Objektraum)} % 7.81, 7.82, 8.83
				\todo{Content}
			% end

			\subsection{Projektion} % 7.85, 7.86
				\todo{Content}
			% end
		% end

		\section{Rasterisierung} % 7.88
			\todo{Content}

			\subsection{Scan-Konvertierung} % 7.89
				\todo{Content}

				\subsubsection{Rasterisierung von Linien (Bresenham-Algorithmus)} % 7.90, 7.91
					\todo{Content}
				% end

				\subsubsection{Rasterisierung von Polygonen (Scanline Algorithmus)} % 7.92
					\todo{Content}
				% end
			% end

			\subsection{Verdeckungsrechnung} % 7.93
				\todo{Content}

				\subsubsection{z-Buffer-Algorithmus} % 7.94, 7.95, 7.96, 7.97
					\todo{Content}
				% end
			% end
		% end

		\section{Ausgabe} % 7.99
			\todo{Content}
		% end

		\section{Beispiele} % 7.101, 7.102, 7.103
			\todo{Content}
		% end
	% end

	\chapter{Transformationen} % 8.1
		\todo{Content}

		\section{Affine Abbildungen} % 8.7, 8.8, 8.9, 8.12, 8.13
			\todo{Content}

			\subsection{Eigenschaften} % 8.14, 8.15, 8.16
				\todo{Content}
			% end

			\subsection{Translation} % 8.17, 8.18
				\todo{Content}
			% end

			\subsection{Homogene Koordinaten} % 8.19, 8.20
				\todo{Content}

				\subsubsection{Translation} % 8.21, 8.22, 8.23, 8.24, 8.25
					\todo{Content}
				% end
			% end

			\subsection{Matrizenschreibweise} % 8.26, 8.27
				\todo{Content}
			% end
		% end

		\section{Skalierung, Scherung, Rotation} % 8.28, 8.29, 8.30, 8.31
			\todo{Content}

			\subsection{Skalierung} % 8.34, 8.35, 8.36
				\todo{Content}
			% end

			\subsection{Scherung} % 8.37, 8.38
				\todo{Content}
			% end

			\subsection{Rotation} % 8.39, 8.40, 8.41, 8.42
				\todo{Content}

				\subsubsection{Rotation um beliebige Achse} % 8.43, 8.44, 8.45
					\todo{Content}
				% end

				\subsubsection{Rotation um beliebige Raumachse} % 8.46, 8.47
					\todo{Content}
				% end
			% end

			\subsection{Nicht-Kommutativität von Transformationen} % 8.48
				\todo{Content}
			% end

			\subsection{Rechenaufwand} % 8.49
				\todo{Content}
			% end
		% end

		\section{Projektion} % 8.50, 8.52, 8.53, 8.54
			\todo{Content}

			\subsection{Perspektive Projektion} % 8.55, 8.59, 8.60
				\todo{Content}

				\subsubsection{Kanonisches Sichtvolumen} % 8.61, 8.62
					\todo{Content}
				% end

				\subsubsection{Allgemeine perspektische Transformation} % 8.63
					\todo{Content}
				% end
			% end

			\subsection{Parallele Projektion} % 8.56
				\todo{Content}
			% end

			\subsection{Ein-, Zwei- und Dreipunktperspektive} % 8.65
				\todo{Content}
			% end
		% end

		\section{3D-Interaktion} % 8.68, 8.69, 8.70
			\todo{Content}

			\subsection{Manipulatoren} % 8.71, 8.72, 8.73, 8.74
				\todo{Content}
			% end
		% end
	% end

	\chapter{3D-Visualisierung} % 9.1
		\todo{Content}

		\section{3D-Daten} % 9.4
			\todo{Content}

			\paragraph{Terrain} % 9.5
				\todo{Content}
			% end

			\paragraph{Laser Scanning} % 9.6
				\todo{Content}
			% end

			\paragraph{Range Images} % 9.7
				\todo{Content}
			% end

			\paragraph{Medizinische Bilddaten} % 9.8, 9.9
				\todo{Content}
			% end

			\paragraph{Schneiden} % 9.10
				\todo{Content}
			% end

			\paragraph{Wetter} % 9.11
				\todo{Content}
			% end

			\paragraph{Klimaforschung} % 9.12
				\todo{Content}
			% end
		% end

		\section{Triangulation von Punktwolken} % 9.13, 9.14
			\todo{Content}

			\subsection{Ideal Triangulation} % 9.15
				\todo{Content}
			% end

			\subsection{Voronoi-Diagramm und Delaunay-Triangulation} % 9.16, 9.17, 9.18
				\todo{Content}
			% end
		% end

		\section{Indirekte Volumenvisualisierung} % 9.22, 9.25
			\todo{Content}

			\subsection{3D-Volumen und Nachbarschaft} % 9.23, 9.24
				\todo{Content}
			% end

			\subsection{2D: Konturlinien} % 9.26
				\todo{Content}
			% end

			\subsection{3D: Isoflächen} % 9.27, 9.28
				\todo{Content}
			% end

			\subsection{2D: Marching Squares} % 9.29, 9.30
				\todo{Content}
			% end

			\subsection{3D: Marching Cubes} % 9.31
				\todo{Content}
			% end

			\subsection{Große Polygonmodelle und Performanz} % 9.33, 9.34
				\todo{Content}

				\subsubsection{Culling von Geometrie} % 9.35
					\todo{Content}
				% end

				\subsubsection{Meshreduktion} % 9.36, 9.37
					\todo{Content}
				% end

				\subsubsection{Mesh-Glättung} % 9.38, 9.39
					\todo{Content}
				% end
			% end
		% end

		\section{Direkte Volumenvisualisierung} % 9.40, 9.41
			\todo{Content}

			\subsection{Density Emitter Model} % 9.42
				\todo{Content}
			% end

			\subsection{Volumen-Rendering-Gleichung} % 9.43, 9.44
				\todo{Content}
			% end
		% end

		\section{Volumen-Rendering-Pipeline} % 9.45
			\todo{Content}

			\subsection{Pipeline} % N/A
				\todo{Content}

				\paragraph{Abtastung} % 9.46, 9.47, 9.48
					\todo{Content}
				% end

				\paragraph{Klassifikation und Beleuchtung} % 9.50
					\todo{Content}
				% end

				\paragraph{Komposition} % 9.51
					\todo{Content}

					\subparagraph{Back-to-Front-Komposition} % 9.52
						\todo{Content}
					% end

					\subparagraph{Front-to-Back-Komposition} % 9.53, 9.54, 9.55
						\todo{Content}
					% end
				% end
			% end

			\subsection{Transferfunktion} % 9.56, 9.57, 9.58, 9.59, 9.60
				\todo{Content}
			% end
		% end
	% end

	\chapter{Szenengraphen am Beispiel X3DOM} % 10.1, 10.2, 10.3, 10.40
		\todo{Content}

		\section{Strukturierung von 3D-Szenendaten} % 10.4, 10.5, 10.6
			\todo{Content}
		% end

		\section{Szenengraph} % 10.10, 10.11, 10.12, 10.14
			\todo{Content}
		% end

		\section{X3DOM} % 10.16, 10.17, 10.19
			\todo{Content}
		% end
	% end

	\chapter{Informationsvisualisierung} % 11.1, 11.2, 11.15, 11.16, 11.17, 11.18, 11.19, 11.20
		\todo{Content}

		\section{Informationsdesign} % 11.23, 11.24, 11.25, 11.34, 11.56, 11.57
			\todo{Content}

			\subsection{Referenzmodell von Card} % 11.26, 11.27, 11.28, 11.29, 11.30, 11.31, 11.32, 11.33
				\todo{Content}
			% end
		% end

		\section{Datentypen} % N/A
			\todo{Content}

			\subsection{1D-Daten, Zeitreihen} % 11.62, 11.63
				\todo{Content}
			% end

			\subsection{2D-Daten} % 11.73
				\todo{Content}
			% end

			\subsection{mD-Daten (multidimensional)} % 11.77, 11.78
				\todo{Content}
			% end

			\subsection{Hierarchien} % 11.91
				\todo{Content}
			% end

			\subsection{Graphen/Netzwerke} % 11.155
				\todo{Content}
			% end
		% end

		\section{Kuchendiagramm (1D)} % 11.64, 11.65
			\todo{Content}
		% end

		\section{Balkendiagramm (1D)} % 11.68
			\todo{Content}
		% end

		\section{Liniendiagramm (Zeitreihe)} % 11.67, 11.68, 11.69
			\todo{Content}

			\paragraph{Problem: Viele Zeitreihen} % 11.70, 11.71
				\todo{Content}
			% end

			\paragraph{Problem: Länge} % 11.72
				\todo{Content}
			% end
		% end

		\section{Scatterplot (2D, 3D)} % 11.74, 11.75
			\todo{Content}

			\paragraph{Problem: Overplotting} % 11.76
				\todo{Content}
			% end
		% end

		\section{Scatterplotmatrix (nD)} % 11.79, 11.80, 11.81
			\todo{Content}
		% end

		\section{Parallele Koordinaten (3D, nD)} % 11.82, 11.83, 11.84, 11.85, 11.86, 11.87, 11.88
			\todo{Content}

			\paragraph{Problem: Overplotting} % 11.89
				\todo{Content}
			% end

			\paragraph{Problem: Viele Dimensionen} % 11.90
				\todo{Content}
			% end
		% end

		\section{Node-Link-Diagramm (Hierarchien, Graphen)} % 11.92, 11.156
			\todo{Content}

			\paragraph{Problem: Layout} % 11.93, 11.157, 11.158
				\todo{Content}
			% end

			\paragraph{Problem: Viele Knoten} % 11.94, 11.159
				\todo{Content}
			% end
		% end

		\section{Treemap (Hierarchien)} % 11.95, 11.96, 11.97, 11.98, 11.99, 11.100, 11.101, 11.102
			\todo{Content}

			\paragraph{Problem: Überlappung} % 11.105
				\todo{Content}
			% end

			\paragraph{Problem: Größendarstellung} % 11.106, 11.107
				\todo{Content}
			% end
		% end

		\section{Zusammenfassung} % 11.160
			\todo{Content}
		% end
	% end

	\chapter{Farbe} % 12.1, 12.2, 12.3, 12.6, 12.6
		\todo{Content}

		\section{Dimensionalität} % 12.8, 12.9
			\todo{Content}
		% end

		\section{Wahrnehmungskorrelate} % 12.10, 12.11, 12.12
			\todo{Content}
		% end

		\section{Reproduktion} % 12.13
			\todo{Content}
		% end

		\section{Berechnung von Farbattributen} % 12.14
			\todo{Content}

			\subsection{Das Auge} % 12.15, 12.16, 12.17
				\todo{Content}
			% end

			\subsection{Sepktrale Charaktertisierung des Auges} % 12.18, 12.19, 12.20, 12.21, 12.22
				\todo{Content}
			% end

			\subsection{Sepktralwertfunktion} % 12.23, 12.24
				\todo{Content}
			% end

			\subsection{Cone Fundamentals} % 12.25
				\todo{Content}
			% end
		% end

		\section{Objektfarben, Lichtmatrix under XIEXYZ-Farbraum} % 12.26, 12.27, 12.28, 12.29
			\todo{Content}
		% end

		\section{Metamerie} % 12.30, 12.31, 12.32, 12.33
			\todo{Content}
		% end

		\section{Gegenfarbtheorie} % 12.35
			\todo{Content}
		% end

		\section{Stevenssche Potenzfunktion} % 12.36
			\todo{Content}
		% end

		\section{CIELAB Farbraum} % 12.37, 12.38
			\todo{Content}
		% end

		\section{Technische Farbräume} % 12.39
			\todo{Content}

			\subsection{Geräte RGB} % 12.40
				\todo{Content}
			% end

			\subsection{Geräteunabhängige RGB} % 12.40
				\todo{Content}
			% end

			\subsection{YCbCr} % 12.41
				\todo{Content}
			% end

			\subsection{HSI/HSV/HSL} % 12.41
				\todo{Content}
			% end

			\subsection{CMY/CMYK} % 12.42
				\todo{Content}
			% end
		% end

		\section{Komplexität von Farbe} % 12.43
			\todo{Content}

			\subsection{Chromatische Adaptation} % 12.45, 12.46
				\todo{Content}

				\subsubsection{Modellbildung} % 12.47, 12.48
					\todo{Content}
				% end
			% end

			\subsection{Farbwahrnehmungsphänomene} % 12.49
				\todo{Content}

				\subsubsection{Simultankontrast} % 12.49
					\todo{Content}
				% end

				\subsubsection{Crispening Effekt} % 12.50
					\todo{Content}
				% end

				\subsubsection{Stevens Effekt} % 12.51
					\todo{Content}
				% end

				\subsubsection{Hunt Effekt} % 12.52
					\todo{Content}
				% end
			% end

			\subsection{Farbwahrnehmungsmodelle} % 12.54
				\todo{Content}

				\subsubsection{CIECAM02} % 12.55, 12.56, 12.57
					\todo{Content}
				% end
			% end

			\subsection{Kontrastsensitivität} % 12.58, 12.59, 12.60
				\todo{Content}

				\subsubsection{S-CIELAB (Spacial-CIELAB)} % 12.61, 12.62, 12.63
					\todo{Content}
				% end

				\subsubsection{iCAM (Image Color Appearance Model)} % 12.64, 12.65, 12.66
					\todo{Content}
				% end
			% end
		% end
	% end

	\chapter{User Interfaces} % 13a.1, 13a.64
		\todo{Content}

		\section{Interaktion} % 13a.7
			\todo{Content}

			\subsection{Möglichkeiten} % 13a.8
				\todo{Content}

				\subsubsection{Kommandozeile} % 13a.9
					\todo{Content}
				% end

				\subsubsection{Menüs} % 13a.10, 13a.11
					\todo{Content}
				% end

				\subsubsection{Formulare} % 13a.12
					\todo{Content}
				% end

				\subsubsection{Fragen und Antworten} % 13a.13
					\todo{Content}
				% end

				\subsubsection{Direkte Manipulation} % 13a.14, 13a.15
					\todo{Content}
				% end

				\subsubsection{3D-Umgebungen} % 13a.17
					\todo{Content}
				% end

				\subsubsection{Natürliche Sprache} % 13a.18
					\todo{Content}
				% end

				\subsubsection{Gesten} % 13a.19
					\todo{Content}
				% end
			% end

			\subsection{Designprozess} % 13a.20, 13a.21, 13a.22
				\todo{Content}

				\subsubsection{Wasserfallmodell} % 13a.24
					\todo{Content}
				% end

				\subsubsection{Spiralmodell} % 13a.25
					\todo{Content}
				% end

				\subsubsection{V-Modell} % 13a.26
					\todo{Content}
				% end

				\subsubsection{Dynamic Systems Development Method (DSDM)} % 13a.27
					\todo{Content}
				% end

				\subsubsection{Design Process Model} % 13a.28
					\todo{Content}
				% end
			% end
		% end

		\section{GUI: Benutzeroberflächen} % 13a.30
			\todo{Content}

			\subsection{Das WIMP-Interface} % 13a.31
				\todo{Content}

				\subsubsection{Fenster-Komponenten} % 13a.32
					\todo{Content}

					\paragraph{Multiple Document Interface (MDI)} % 13a.33
						\todo{Content}
					% end

					\paragraph{Single Document Interface (SDI)} % 13a.34
						\todo{Content}
					% end

					\paragraph{Tabbed Document Interface} % 13a.35
						\todo{Content}
					% end
				% end

				\subsubsection{Dialogboxen} % 13a.36
					\todo{Content}

					\paragraph{Checkboxen} % 13a.37
						\todo{Content}
					% end

					\paragraph{Radio Buttons} % 13a.38
						\todo{Content}
					% end

					\paragraph{Listboxen} % 13a.39
						\todo{Content}
					% end

					\paragraph{Comboboxen} % 13a.40
						\todo{Content}
					% end

					\paragraph{Spinner} % 13a.41
						\todo{Content}
					% end

					\paragraph{Slider} % 13a.42
						\todo{Content}
					% end

					\paragraph{Weiteres} % 13a.43
						\todo{Content}
					% end
				% end
			% end

			\subsection{Menübasierte Programme} % 13a.44, 13a.45, 13a.46
				\todo{Content}

				\subsubsection{Untermenüs} % 13a.47
					\todo{Content}
				% end

				\subsubsection{Auswahl (if-then-else-Struktur)} % 13a.48, 13a.49
					\todo{Content}

					\paragraph{Verschachtelte Entscheidungsstrukturen} % 13a.50
						\todo{Content}
					% end
				% end

				\subsubsection{Die case-Struktur} % 13a.51
					\todo{Content}
				% end

				\subsubsection{Modularisierung} % 13a.52
					\todo{Content}
				% end
			% end

			\subsection{GUI-Anwendungen und eventbasiertes Programmieren} % 13a.53, 13a.54
				\todo{Content}

				\subsubsection{Graphical User Interfaces (GUIs)} % 13a.55
					\todo{Content}
				% end

				\subsubsection{Event-Handler} % 13a.56, 13a.57
					\todo{Content}
				% end

				\subsubsection{Nutzerinteraktionen} % 13a.58
					\todo{Content}
				% end

				\subsubsection{Das Delegationsmodell} % 13a.59
					\todo{Content}
				% end
			% end
		% end

		\section{3D-Interaktion} % 13a.61, 13a.62
			\todo{Content}
		% end
	% end

	\chapter{Multimedia Information Retrieval} % 13b.1, 13b.7, 13b.8, 13b.64
		\todo{Content}

		\section{Inhaltsbasierte Suche} % 13b.10, 13b.11, 13b.14, 13b.17
			\todo{Content}

			\subsection{Mathematische Beschreibung} % 13b.15
				\todo{Content}
			% end

			\subsection{Retrieval Ergebnis} % 13b.16
				\todo{Content}
			% end
		% end

		\section{Distanzmaße} % 13b.19, 13b.20, 13b.21, 13b.22, 13b.23, 13b.24
			\todo{Content}
		% end

		\section{Query-Modalitäten} % 13b.25, 13b.30, 13b.31
			\todo{Content}

			\subsection{Text} % 13b.32, 13b.33, 13b.34, 13b.35, 13b.36, 13b.37
				\todo{Content}
			% end

			\subsection{Example} % 13b.38, 13b.39, 13b.40, 13b.41, 13b.42
				\todo{Content}
			% end

			\subsection{Example-Bilder} % 13b.43
				\todo{Content}
			% end

			\subsection{Sketch} % 13b.44, 13b.45, 13b.46, 13b.47, 13b.48, 13b.49, 13b.50, 13b.51, 13b.52, 13b.53
				\todo{Content}
			% end
		% end

		\section{Explorative Suche} % 13b.54, 13b.55
			\todo{Content}

			\subsection{Research Data} % 13b.56, 13b.57, 13b.58, 13b.59, 13b.60
				\todo{Content}
			% end
		% end
	% end
\end{document}
