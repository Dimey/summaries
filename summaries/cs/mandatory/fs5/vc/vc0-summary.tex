\documentclass[a4paper, 11pt, accentcolor = tud3b]{tudreport}

% Core packages.
\usepackage[T1]{fontenc}
\usepackage[utf8]{inputenc}
\usepackage[ngerman]{babel}
% Other packages.
\usepackage[german, ruled, vlined, linesnumbered]{algorithm2e}
\usepackage{amsthm}
\usepackage{caption}
\usepackage{csquotes}
\usepackage{enumitem}
\usepackage[mathcal]{euscript} % Get readable mathcal font.
\usepackage{float}
\usepackage{pgfplots}
\usepackage{hyperref}
\usepackage{mathtools}
\usepackage{siunitx}
\usepackage{qtree}
\usepackage{stmaryrd}
\usepackage{tabto}
\usepackage{tikz}
\usepackage[disable]{todonotes}
\usetikzlibrary{arrows.meta, shapes, backgrounds, angles, calc, decorations.markings, positioning}

% Basic information.
\title{Visual Computing}
\subtitle{Zusammenfassung \\ Fabian Damken}
\author{Fabian Damken}
\date{\today}

% Description-list styling.
\SetLabelAlign{parright}{\parbox[t]{\labelwidth}{\raggedleft#1}}
\setlist[description]{style = multiline, leftmargin = 4cm, align = parright}

\colorlet{colorDensity}{tud1b}

\tikzset{density/.style = { colorDensity, line width = 2pt }}
\tikzset{declare function = { gaussian(\x,\m,\S) = 1/sqrt(2*pi*\S) * e^(-1/(2*\S) * (\x-\m)^2); }}
\tikzset{> = { Latex[length = 2.5mm] }}
\tikzstyle{every path} = [ very thick ]

\MakeOuterQuote{"}

% New commands.
\DeclareMathOperator{\total}{d}
\DeclareMathOperator{\Rang}{Rang}
\DeclareMathOperator{\const}{const}
\DeclareMathOperator{\sign}{sign}
\DeclareMathOperator{\sinc}{sinc}
\newcommand{\dif}[1]{\,\total#1}
\newcommand{\N}{\mathbb{N}}
\newcommand{\R}{\mathbb{R}}
\newcommand{\Z}{\mathbb{Z}}
\newcommand{\qef}{\hfill \( \square \)}
\newcommand{\given}{\,\vert\,}
% Matrix/Vector notation.
\makeatletter
% TODO: This does not make all symbols bold (e.g. 'v').
\newcommand{\mat}[1]{\boldsymbol{#1}}
% TODO: This adds crazy symbols for symbols that cannot be drawn bold (e.g. \dot{v} becomes \underline{v}).
\renewcommand{\vec}[1]{\boldsymbol{\mathbf{#1}}}
\makeatother
% Abbreviations.
\renewcommand{\dh}{d.\,h.~}
\newcommand{\bzw}{bzw.~}
\newcommand{\ca}{ca.~}
\newcommand{\bspw}{bspw.~}
\newcommand{\bzgl}{bzgl.~}
\newcommand{\zB}{z.\,B.~}
\newcommand{\iA}{i.\,A.~}
\newcommand{\ggf}{ggf.~}
\newcommand{\mglw}{mglw.~}
\newcommand{\vs}{vs.~}
\newcommand{\DIRKIN}{DIR\,KIN~}
\newcommand{\INVKIN}{INV\,KIN~}
\newcommand{\DIRJAC}{DIR\,JAC~}
\newcommand{\INVJAC}{INV\,JAC~}
\newcommand{\DIRDYN}{DIR\,DYN~}
\newcommand{\INVDYN}{INV\,DYN~}

% https://tex.stackexchange.com/a/333383
\makeatletter
\renewcommand*\env@matrix[1][*\c@MaxMatrixCols c]{%
	\hskip -\arraycolsep
	\let\@ifnextchar\new@ifnextchar
	\array{#1}}
\makeatother

\begin{document}
	\maketitle
	\tableofcontents
	\listoftodos

	\chapter{Einführung} % 1.1, 1.10
		\todo{Content}

		\section{Visual Computing} % 1.1, 1.18, 1.30, 1.31, 1.32, 1.33, 1.34, 1.50
			\todo{Content}

			\subsection{3D-Internet} % 1.35, 1.36, 1.37
				\todo{Content}
			% end

			\subsection{Skalierbare Objektmodellierung/-erkennung} % 1.38, 1.39, 1.40, 1.41, 1.42, 1.43
				\todo{Content}
			% end

			\subsection{Big Data, Visual Analytics} % 1.44, 1.45, 1.46
				\todo{Content}
			% end

			\subsection{Scene Understanding} % 1.47, 1.48, 1.49
				\todo{Content}
			% end
		% end

		\section{Generalisierte Dokumente} % 1.51, 1.52, 1.53, 1.54, 1.55, 1.56, 1.68
			\todo{Content}

			\subsection{Retro-Digitalisierung, Digital Creation} % 1.57, 1.58, 1.59
				\todo{Content}
			% end

			\subsection{Generative Modeling Language} % 1.62, 1.63, 1.64, 1.65, 1.66, 1.67
				\todo{Content}
			% end
		% end
	% end

	\chapter{Wahrnehmung}
		\section{Human-Computer-Interaction}
			Abbildung~\ref{fig:hci} zeigt den klassischen Zyklus der \emph{Human-Computer-Interaction} (HCI), \dh der Interaktion zwischen Mensch und Maschine. Dabei dient insbesondere die visuelle Interaktion und Kommunikation über das Auge eine große Rolle.
		
			\begin{figure}
				\centering
				\begin{tikzpicture}[->, every node/.style = { align = center }]
					\node [draw, rectangle] (human) {Human};
					\node [draw, rectangle, right = 4 of human] (computer) {Computer};
					
					\draw (human) to[bend right] node[below]{Explizite Eingabe, \\ Implizite Interaktion} (computer);
					\draw (computer) to[bend right] node[above]{Visuelle Ausgabe, \\ Andere Ausgabemodalitäten} (human);
				\end{tikzpicture}
				\caption{Klassischer Zyklus der Human-Computer-Interaction (HCI).}
				\label{fig:hci}
			\end{figure}
		% end

		\section{Überblick}
			Der Mensch hat fünf grundlegende Sinne: Sehen, Hören, Fühlen, Schmecken und Riechen, wobei das Sehen, Hören und Fühlen derzeit dominant sind. Der heute sicherlich relevanteste Sinn ist dabei das Sehen und das menschliche Auge. Da die meisten erzeugten Bilder der Kommunikation von und zum Menschen dienen sollen, ist es gut, das menschliche visuelle System zu kennen, um den Informationstransfer optimal zu gestalten (der Monitorausgang ist nicht das Ende des Informationsflusses).
			
			Hören und Fühlen sind dabei relevant für die Informationsaufnahme und Interaktion mit der realen Welt (außerhalb der Mensch-Maschine-Interaktion).
			
			Bei der Gestaltung von Kommunikation gibt es zwei große Probleme:
			\begin{itemize}
				\item Die Wahrnehmung ist nicht objektiv.
				\item Das visuelle System ist stark nichtlinear (es ist keine einfache Interpolation oder Extrapolation von Versuchsergebnissen möglich).
			\end{itemize}

			\subsection{Menschliche Informationsverarbeitung}
				Abbildung~\ref{fig:human_info_processing} zeigt die drei Stufen der menschlichen Informationsverarbeitung:
				\begin{itemize}
					\item \emph{Wahrnehmung} von Eindrücken durch die Sinne,
					\item \emph{Entscheidung}sfindungs im Gehirn und
					\item \emph{Reaktion} durch den Körper.
				\end{itemize}
				Dabei verhält sich die Ausführungszeit additiv und die Funktionen werden durch neurologisch getrennte Gehirnteile ausgeführt, die "elektronisch" verbunden sind.
				
				\begin{figure}
					\centering
					\begin{tikzpicture}[->, every node/.style = { draw, rectangle, minimum height = 0.8cm, minimum width = 5cm }]
						\node (a) {Wahrnehmung (Sensorik)};
						\node [below = 0.5 of a] (b) {Entscheidung (Kognition)};
						\node [below = 0.5 of b] (c) {Reaktion (Motorik)};
						
						\draw (a) -- (b);
						\draw (b) -- (c);
					\end{tikzpicture}
					\caption{Modulares Drei-Stufenmodell der menschlichen Informationsverarbeitung.}
					\label{fig:human_info_processing}
				\end{figure}
				
				Dabei braucht jede Bearbeitung in den einzelnen Stufen unterschiedlich lange und die benötigten Zeiten können verwendet werden, um die Performanz abzuschätzen, \bzw vorherzusagen (\bspw für die Bildfrequenz von Filmen, die maximale Morserate, \dots). Typische Zeiten sind in Tabelle~\ref{fig:human_processing_times} abgebildet.
				
				\begin{table}
					\centering
					\begin{tabular}{l|l|l}
						\textbf{Untersystem}     & \textbf{Durchschnitt}   & \textbf{Bereich}                 \\ \hline
						Wahrnehmung (Perzeption) & \SI{100}{\milli\second} & \SIrange{50}{200}{\milli\second} \\
						Entscheidung (Kognition) & \SI{70}{\milli\second}  & \SIrange{25}{170}{\milli\second} \\
						Reaktion (Motorik)       & \SI{70}{\milli\second}  & \SIrange{30}{100}{\milli\second}
					\end{tabular}
					\caption{Typische Bearbeitungszeiten der Untersysteme der menschlichen Informationsverarbeitung.}
					\label{fig:human_processing_times}
				\end{table}

				\subsubsection{Eingabe (Wahrnehmung)}
					Die Untersysteme der Wahrnehmung,
					\begin{itemize}
						\item Visuell (Sehen)
						\item Akustisch (Hören)
						\item Haptisch (Fühlen)
					\end{itemize}
					können dabei (theoretisch) parallel arbeiten.
				
					\paragraph{Klangwahrnehmung}
						Die Hauptkomponenten von Klängen sind
						\begin{itemize}
							\item Klangfarbe,
							\item Tonlage und
							\item Lautstärke.
						\end{itemize}
						Diese werden durch verschiedene Mechanismen wahrgenommen und Informationen (\zB der Ursprung eines Geräuschs) extrahiert.
					% end

					\paragraph{Berührungswahrnehmung}
						Die Hauptkomponenten der Haptik sind
						\begin{itemize}
							\item Fühl- und Tastsinn (Temperatur, Schmerz, Druck, Oberflächen) und
							\item Propriozeption (Wahrnehmung der Bewegung und Lage der eigenen Körperglieder).
						\end{itemize}
						Dabei interagiert die Haptik stark mit Sehen und Hören, was bei sich widersprechenden Informationen Illusionen hervorrufen kann. Ein User-Interface-Designer nutzt Illusionen dabei geziehlt aus, um bestimmte Informationen zu vermitteln.
					% end
				% end

				\subsubsection{Ausgabe (Reaktion)}
					Die Untersysteme der Reaktion,
					\begin{itemize}
						\item Artikulation (Sprechen)
						\item Motorisch (Bewegen)
					\end{itemize}
					können dabei (theoretisch) parallel arbeiten.
					
					Die motorische Ausgabe kann dabei auf verschiedene Weisen angewandt werden:
					\begin{itemize}
						\item Diskret (Schalter) oder
						\item Kontinuierlich (Heben).
					\end{itemize}
					Sie ist dabei beschränkt durch Geschwindigkeit, Stärke, Koordinationsvermögen, Wendigkeit, \dots. Neurologisch ist die motorische Ausgabe dabei mit dem haptischen System verbunden (Reflexe).
					
					Das \emph{Muskelgedächtnis} hilft dabei, relevante Positionen im Raum (\zB die Gangschaltung im Auto) zu lernen.
				% end
			% end
		% end

		\section{Wahrnehmung}
			\subsection{Das Auge}
				\subsubsection{Reiz und Licht}
					Einer äußerer, visueller Reiz (Licht) erzeugt beim Menschen eine physikalische Rezeption des äußeren Reizes (Input). Dies geschieht durch einen Sensor (\bspw das Auge) und die Reizung produziert ein neuro-physiologisches Signal. Dieses wird anschließend verarbeitet und interpretiert.
					
					Physikalisch ist ein solcher Reiz elektromagnetische Strahlung. Dabei wird monochromatisches, \dh einfarbiges, Licht durch die Angabe der Frequenz \(v\), \bzw der Wellenlänge \(\lambda\), beschrieben. Diese beiden Größen sind durch die Beziehung
					\begin{equation*}
						v \lambda = c, \quad c \approx \SI{3e8}{\meter\per\second}
					\end{equation*}
					miteinander verknüpft, wobei \(c\) die Ausbreitungsgeschwindigkeit des Lichts ist.
					
					Das menschliche Auge kann dabei Frequenzen im Wellenlängenbereich \( \SIrange{380}{750}{\nano\meter} \) wahrnehmen. Kleinere Wellenlängen haben \zB Ultraviolett-Licht, Röntgen- und \(\gamma\)-Strahlung. Darüber liegende Wellenlängen haben \zB Infrarot-Licht und Rundfunk-Wellen.
				% end

				\subsubsection{Das visuelle System}
					Das menschliche Auge ist aufgebaut aus:
					\begin{itemize}
						\item Hornhaut (Kornea)
						\item Linse (zur Scharfstellung)
						\item Iris (Blendenmechanismus)
						\item Retina (Netzhaut)
							\begin{itemize}
								\item Blinder Fleck: Hier geht der Sehnerv ab.
								\item Fovea Centralis (Gelber Fleck): Bereich mit der höchsten Auflösung.
							\end{itemize}
					\end{itemize}
				% end

				\subsubsection{Photorezeptoren}
					Die Photorezeptoren (welche auf der Retina platziert sind), bestehen aus:
					\begin{itemize}
						\item Stäbchen
							\begin{itemize}
								\item Hauptsächlich außerhalb der Fovea.
								\item Das Empfindlichkeitsmaximum liegt bei \SI{498}{\nano\meter} ("grün").
							\end{itemize}
						\item Zapfen
							\begin{itemize}
								\item Vor allem in der Fovea platziert.
								\item Es gibt drei Zapfentypen für Farbsehen.
								\item Das Empfindlichkeitsmaximum dieser Zapfen liegt bei \SI{420}{\nano\meter} ("blau"), \SI{534}{\nano\meter} ("grün") und \SI{564}{\nano\meter} ("rot").
							\end{itemize}
					\end{itemize}
				% end

				\subsubsection{Skotopisches und Photopisches Sehen}
					\begin{itemize}
						\item Nachtsehen (skotopisch): Dominanz der Stäbchen.
						\item Tagsehen (photopisch): Dominanz der Zapfen.
					\end{itemize}
				% end

				\subsubsection{Zapfenverteilung} % 2.38, 2.39, 2.40, 2.41, 2.42, 2.43, 2.44
					\todo{Content}
				% end
			% end

			\subsection{Vorverarbeitung visueller Informationen}
				\subsubsection{Signalverarbeitung in der Retina}
					Neben den Photorezeptoren gibt es noch weitere Zellen zur Signalverarbeitung in der Retina:
					\begin{itemize}
						\item Horizontale Zellen \\ Kombination von mehreren Rezeptoren einer Region.
						\item Amakrin-Zellen \\ Zeitliche Verarbeitung.
						\item Bipolar-Zellen \\ Informationsfilter (Sammeln, Gewichten und Weiterleiten).
						\item Ganglien-Zellen \\ Integration Informationen (\zB Kontrastwahrnehmung).
					\end{itemize}
				% end

				\subsubsection{Helligkeit}
					\begin{itemize}
						\item \emph{Helligkeit} (\emph{brightness}) entspricht der wahrgenommenen Menge an Licht, das von einer selbstleuchtenden Lichtquelle ausgeht.
						\item \emph{Helligkeit} (\emph{lightness}) entspricht der wahrgenommenen Menge an Licht, das von einer reflektierenden Oberfläche ausgeht.
							\begin{itemize}
								\item Dies ist keine absolute Wahrnehmungsgröße und abhängig von
									\begin{itemize}
										\item Reizstärke (Leuchtdichte)
										\item Vorherige Leuchtdichte (Adaption)
										\item Umgebungsleuchtdichte
										\item Größe (Fläche) des Reizes
									\end{itemize}
								\item Somit subjektiv!
							\end{itemize}
						\item Dies wirft einige nicht so einfach zu beantwortende Fragen auf, z.\,B.: Was ist weiß? Was ist schwarz? Was ist mittelgrau?
						\item Der Hell-Dunkel-Kontrast ist dabei eine wichtige Empfindungsgröße zum Form- und Objektsehen. Daher muss der Unterschied groß genug sein (für kleine Details mindestens \(3:1\), besser \(10:1\)).
					\end{itemize}

					\paragraph{Kontrast als Reizverhältnis}
						Für den Kontrast gibt es verschiedene Definitionen, \zB (dabei ist \(L\) stets die Leuchtdichte):
						\begin{equation*}
							m = k = \frac{L_\text{max} - L_\text{min}}{L_\text{max} + L_\text{min}}
						\end{equation*}
						oder
						\begin{equation*}
							K = \frac{L_R - L_H}{L_H} = \frac{\delta L}{L_H}
						\end{equation*}
						wobei \( L_R \) die Leuchtdichte des Vordergrunds und \(L_H\) die Leuchtdichte des Hintergrunds darstellt.
						
						\todo{Weber-Fechnersches Gesetz, Stevensches Gesetz; 2.65}
					% end
				% end

				\subsubsection{Erkennung von Details}
					Die Erkennung kleiner Details ist begrenzt durch
					\begin{itemize}
						\item Optische Eigenschaften des Auges, \zB Beugungserscheinungen,
						\item Abtastung durch Rezeptoren und
						\item nervöse Verarbeitung.
					\end{itemize}
					Zwei mögliche Maße zur "Erkennbarkeit" sind:
					\begin{itemize}
						\item Kontrastempfindlichkeit
						\item Schwellenkontrast
					\end{itemize}

					\paragraph{Kontrastempfindlichkeit}
						Die Kontrastempfindlichkeit ist die Auflösung des menschlichen Auges im Frequenzraum. Veränderliche Intensität kann dabei mit Sinus-förmigen Mustern gemessen werden.
					% end
				% end

				\subsubsection{Frühe Wahrnehmung}
					Das Auge nimmt einige Veränderungen der Umgebung schneller wahr als andere. Um die Aufmerksamkeit auf etwas zu lenken, können beispielsweise
					\begin{itemize}
						\item Farbe,
						\item Richtung,
						\item Bewegung,
						\item Größe,
						\item Beleuchtung/Schattierung
					\end{itemize}
					variiert werden.
				% end
			% end

			\subsection{Informationsextraktion}
				Ein reiner Reiz ist noch keine \emph{Wahrnehmung}. Dazu kommen noch andere Faktoren wie Kontext, Erwartungen, Adaption. Das Messen der tatsächlichen Wahrnehmung ist leider sehr schwierig, weshalb häufig nur statistische Aussagen auf Basis von User-Tests getätigt werden können.
				
				Dabei wird erschwert, dass die Wahrnehmung nicht immer der Realität entspricht. Es wird hingegen das Bild durch einen Wahrnehmungsprozess im Gehirn produziert. Dabei wird die menschliche Wahrnehmung adaptiert, \bspw dreht sich das Bild bei einem Kopfstand.

				\subsubsection{Raumwahrnehmung}
					Die Wahrnehmung des Raums (Raumwahrnehmung) enthält unter anderem
					\begin{itemize}
						\item Tiefenwahrnehmung,
						\item Entfernungs- und Distanzwahrnehmung und
						\item Ausrichtung des Körpers im Raum.
					\end{itemize}
					Daran sind viele Wahrnehmungssysteme beteiligt:
					\begin{itemize}
						\item Vestibuläres System (im Innenohr)
						\item Haptisch-somatisches System (Tasten und Berühren)
						\item Auditives Sehen (Gehört)
						\item Propriozeptives System (Eigenwahrnehmung)
						\item Visuelles System
					\end{itemize}
				
					Dabei ist die Raumwahrnehmung auch mit einem Auge (Monokular) möglich (tatsächlich sind \SIrange{5}{10}{\percent} aller Menschen stereoblind und \SI{20}{\percent} haben eine Stereo-Schwäche).
					
					Tatsächlich ist die Raumwahrnehmung ein sehr komplexer Prozess, der auch heute nur zu Teilen verstanden wird. Dabei fließen noch viele weitere Phänomene ein, \zB Größenkonstanz, Annahme starrer Körper oder Vektion. Letzteres ist dabei die scheinbare Eigenbewegung bei einem statischen Vordergrund als Referenzrahmen und einem bewegtem Hintergrund.
				% end

				\subsubsection{Depth Cue Theorie}
					Die Annahme der \emph{Depth Cue Theorie} ist, dass die Raumwahrnehmung des visuellen Systems auf Hinweisreizen (sogenannten \emph{Depth Cues}) basiert. Diese werden in drei Kategorien eingeteilt:
					\begin{enumerate}
						\item Binokulare Depth Cues (mit zwei Augen)
							\begin{itemize}
								\item Disparität/Parallaxe
								\item Akkomodation (Krümmung der Augenlinsen)
								\item Konvergenz (die Augen nach innen drehen)
							\end{itemize}
						\item Pictoiral Depth Cues (mit einem Auge)
							\begin{itemize}
								\item Linearperspektive
								\item Verdeckung
								\item Texturgradient
								\item Fokus und Blur
								\item Atmosphärische Tiefe
								\item Vertraute Größe
								\item Höhe im Gesichtsfeld
								\item Beleuchtung
								\item Schattenwurf
								\item Luminanzänderung
								\item Transluzenz
								\item Schattierung
							\end{itemize}
						\item Dynamische Depth Cues (Animation)
							\begin{itemize}
								\item Bewegungsparallaxe
								\item Kinetischer Tiefeneffekt
								\item Interposition
								\item Bewegung von Highlights
							\end{itemize}
					\end{enumerate}

					\paragraph{Stereoskopie}
						Bei der Stereoskopie nehmen beide Augen ein leicht unterschiedliches Bild wahr, woraus die Entfernung zu einem Objekt berechnet werden kann.
					% end

					\paragraph{Pictorial Depth Cues}
						\subparagraph{Linearperspektive} % 2.90
							\todo{Content}
						% end

						\subparagraph{Texturgradient}
							Sind als parallel angenommene Linien nicht mehr parallel, so ergibt sich eine scheinbare Tiefe (als wenn kariertes Papier um einen Ball gerollt und von oben betrachtet wird).
						% end

						\subparagraph{Fokus und Blur}
							Das Auge fokussiert einen Punkt und produziert somit eine Tiefenschärfe. Daran kann erahnt werden, welche Objekte im Vorder- oder Hintergrund sind.
						% end

						\subparagraph{Atmosphärische Tiefe}
							Anhand der Atmosphäre (\zB durch Nebel ausgelöst) wird erkannt, was vermutlich im Hintergrund liegt. So kann zum Beispiel bei einem Foto von einem Berg geschätzt werden, dass der Boden niedriger ist, wenn Wolken über diesem hängen.
						% end

						\subparagraph{Schattenwurf}
							Annahme: Beleuchtung von oben und Vorhandensein einer Grundebene. Dann kann durch den Abstand von Schatten zum Objekt erahnt werden, wie weit dieses vom Boden entfernt ist.
						% end
					% end

					\paragraph{Dynamische Depth Cues}
						\subparagraph{Motion Parallax} % 2.98
							\todo{Content}
						% end

						\subparagraph{Raumwahrnehmung durch Bewegung}
							Wird \zB eine schaukelnde Vase von oben betrachtet, so bewegt sich die Öffnung charakteristisch, sodass eine Wahrnehmung der Tiefe entsteht.
						% end

						\subparagraph{Kinetic Depth Effect, Structure from Motion} % 2.100
							\todo{Content}
						% end
					% end

					\paragraph{Auswertung von Depth Cues}
						Unterschiedliche Depth Cues haben im Allgemeinen einen unterschiedliche Informationsgehalt. Dabei sind sie nicht redundant, sondern additiv. Durch ein kompliziertes Zusammenspiel (flexible Gewichtung, Dominanz eines Depth Cue) bildet sich das Gehirn ein Bild. Dabei bildet es sich allerdings kein tatsächliches 3D-Modell, sondern verwendet sie unterschiedlichen Cues für verschiedene Aufgaben. Diese können \zB sein:
						\begin{itemize}
							\item Einschätzen von Objektgrößen
							\item Einschätzen von Entfernungen
							\item Verfolgung von Pfaden
							\item Navigation
							\item Einschätzen der Eigenbewegung
							\item Abschätzung der Kollisionszeit
						\end{itemize}
					% end
				% end
			% end
		% end

		\section{Aufmerksamkeit}
			\subsection{Limitierung der Wahrnehmung}
				Die initiale Reizaufnahme hat viele Limitieren, sodass nur ein Bruchteil des äußeren Reizes zur kognitiven Verarbeitung zur Verfügung steht, Dabei sind Aufmerksamkeit und externe Faktoren wichtige Einflüsse auf die tatsächliche Wahrnehmung. Die Wahrnehmung ist dabei eher eine partielle Hypothese, die auf Basis unvollständiger Informationen generiert wurde. Es wird dabei periodisch aktualisiert aufgrund von Beobachtungen, \dh die Hypothese wird gegen sensorische Daten getestet. Durch eine dynamische Suche des visuellen Systems wird nach der besten Hypothese/Interpretation/Modell gesucht.
			% end

			\subsection{Das Gedächtnis und "Gateway to Memory"}
				Das Gehirn kann sich auf bestimmte Dinge fokussieren und den Rest ignorieren. Dabei gibt es drei verschiedene Arten der Aufmerksamkeit:
				\begin{itemize}
					\item \emph{Gewählte Aufmerksamkeit} (selective): Zwischen mehreren Möglichkeiten wird eine zu fokussierende Sache aktiv ausgewählt.
						\begin{itemize}
							\item Das Auge folgt den Objekten von Interesse.
							\item Der Kopf folgt den Klängen von Interesse.
							\item Es gibt nur einen einzigen "Ort der Aufmerksamkeit".
						\end{itemize}
					\item \emph{Geteilte Aufmerksamkeit} (divided): Ein Versuch durch "Multitasking" mehrere Dinge zu fokussieren.
						\begin{itemize}
							\item Entweder "gleichzeitig" der durch schnelles Umschalte (time multiplexing).
							\item Dies wirkt sich negativ auf die Verarbeitung aus, wenn die Aufgaben überfordernd sind.
							\item Die Aufgaben beeinträchtigen sich gegenseitig.
						\end{itemize}
					\item \emph{Erfasste Aufmerksamkeit} (captured): Ein äußerer Reiz zieht alle Aufmerksamkeit auf sich.
						\begin{itemize}
							\item Im Gegensatz zur gewählten Aufmerksamkeit wird der "Ort" nicht aktiv ausgewählt.
							\item Dies geschieht \zB wenn man von einem Tier angefallen wird.
						\end{itemize}
				\end{itemize}
			
				Das menschliche Gedächtnis ist in mehrere "Teilgedächtnisse" aufgeteilt. Voran steht das \emph{Arbeitsgedächtnis}, auf das ein schneller Zugriff (\ca \SI{70}{\milli\second}) möglich ist, welches aber einen schnellen Verfall hat (nach \ca \SI{200}{\milli\second}). Nach wenigen Sekunden wird der Inhalt jedoch an das Langzeitgedächtnis weitergegeben. Es stellt sozusagen das "Schmierblatt" des Gehirns da.
				
				Das Langzeitgedächtnis ist langsamer (\ca \SI{100}{\milli\second}), dafür aber auch sehr viel größer (die genaue Größe ist unbekannt). Das Langzeitgedächtnis hat dabei drei Hauptaufgaben:
				\begin{itemize}
					\item Informationen speichern und sich an diese erinnern,
					\item Informationen abrufen und
					\item Informationen vergessen.
				\end{itemize}
			% end
		% end
	% end

	\chapter{Computer Vision: Objekterkennung und Bayes}
		Die \emph{Computer Vision} beschäftigt sich mit dem maschinellen Sehen, \dh der Suche nach einem Modell des menschlichen Sehens. Anwendungsgebiete sind \bspw Autos, die Fußgänger erkennen, medizinische Bildverarbeitung, Überwachung, Unterhaltung, Computergraphik, \dots.

		\section{Computer Vision}
			Das einfachste Standardmodell einer Lochkamera ist ein Kasten mit einem kleinen Loch. Um ein digitales Bild eines solchen Kameramodells zu erhalten, wird das Bild rasterisiert. Demnach ist ein Graustufenbild eine Matrix an Pixeln mit jeweils einem Wert (die "Grauigkeit" des Pixels).
			
			Die Computer Vision beschäftigt sich nun damit, aus einem solchen Bild Informationen zu extrahieren. Bei der Objekterkennung ist es wichtig, eine gute lokale Beschreibung/Merkmale zu haben (\zB Augen, Mund, Nase) und eine globale Anordnung der lokalen Merkmale (\zB relative Positionen, relative Größen). Es ist aber auch eine schnelle Generierung guter Hypothesen, Segmentierung der Bildbereiche und kennen des Szenenkontextes wichtig.
			
			Nach Fischler und Elschlager hat das Modell eines Bildes zwei Komponenten: Teile (2D Bildfragmente) und den Aufbau (die Anordnung der Teile). Mit diesem abstrakten Modell lassen sich viele Dinge (\zB ein Gesicht) charakterisieren.
		% end

		\section{Bayesian Decision Theory}
			Beispiel: Buchstabenerkennung. Es soll ein neu aufgenommener Buchstabe so klassifiziert werden, dass die Wahrscheinlichkeit der Fehlklassifikation minimiert wird.

			\subsection{Konzepte und Bayes Theorem}
				\paragraph{Vorbemerkung: Wahrscheinlichkeitsdichte und Wahrscheinlichkeit}
					Ist \( p(x) \) eine Wahrscheinlichkeitsdichte, so ist die Wahrscheinlichkeit, dass \(x\) im Intervall \( (x_0, x_y) \) liegt, gegeben durch:
					\begin{equation*}
						P(x_0 < x < x_1) = \int_{x_0}^{x_1} \! p(\tau) \dif{\tau}
					\end{equation*}
					Da für die Wahrscheinlichkeit, dass \( x \) im Intervall \( (x, x + \Delta x) \) mit \( \Delta x \to 0 \) gilt:
					\begin{equation*}
						\lim\limits_{\Delta x \to 0} P(x) = \lim\limits_{\Delta x \to 0} P(x < t < x + \Delta x) = p(x) \cdot \Delta x
					\end{equation*}
					kann Wahrscheinlichkeitsdichte und Wahrscheinlichkeit in den meisten Fällen gegeneinander ausgetauscht werden.
				% end
			
				\paragraph{1. Konzept: A-Priori Wahrscheinlichkeit (Prior)}
					Die \emph{a-priori Wahrscheinlichkeit} (Prior) enthält die Information, wie wahrscheinlich eine beliebige Messung der Klasse zugehört (\dh die "Klassenhäufigkeit"). Ist \( C_k \) eine Klasse, so ist \( P(C_k) \) der Prior \bzgl der Klasse \( C_k \) (analog für \( p(C_k) \)).
				% end
				
				\paragraph{2. Konzept: Bedingte Wahrscheinlichkeit (Likelihood)}
					Ist \(\vec{x}\) der Merkmalsvektor (Feature), welcher Eigenschaften der Messung beschreibt (Anzahl schwarzer Pixel, Höhe/Breite, \dots) und \( C_k \) eine Klasse, so ist \( P(\vec{x} \given C_k) \) die \emph{Likelihood}, \dh die Wahrscheinlichkeit, dass \(\vec{x}\) für einen Buchstaben der Klasse \( C_k \) gemessen wird (analog für \( p(\vec{x} \given X_k) \)).
				% end
				
				\paragraph{3. Konzept: A-Posteriori Wahrscheinlichkeit (Posterior), Bayes Theorem}
					Die \emph{a-posteriori Wahrscheinlichkeit} (Posterior) ist die Wahrscheinlichkeit, dass ein Merkmalsvektor \(\vec{x}\) einer Klasse \( C_k \) angehört, \dh \( P(C_k \given \vec{x}) \). Dieser Posterior kann durch Bayes Theorem gefunden werden:
					\begin{equation*}
						P(C_k \given \vec{x}) = \frac{P(\vec{x} \given C_k) \cdot P(C_k)}{P(\vec{x})}
					\end{equation*}
					Oder Namentlich:
					\begin{equation*}
						\text{Posterior} = \frac{\text{Likelihood} \times \text{Prior}}{\text{Normalisierung}}
					\end{equation*}
				% end
			% end

			\subsection{Problemstellung}
				Abbildung~\ref{fig:likelihoodPriorPosterior} zeigt die Likelihood, Prior und den Posterior auf. Die Zielstellung eines Bayesian Classifier ist nun, die Wahrscheinlichkeit der Fehlklassifikation zu minimieren und somit eine Entscheidungsgrenze zu bestimmen. Die Wahrscheinlichkeit eines Fehlers ist gegeben durch:
				\begin{align*}
					P(\text{Fehler}) &= P(x \in R_2, C_1) + P(x \in R_1, C_2) \\
						&= P(x \in R_2 \given C_1) P(C_1) + P(x \in R_1 \given C_2) P(C_2) \\
						&= \int_{R_2} \! p(x \in R_2 \given C_1) P(C_1) \dif{x} + \int_{R_2} \! p(x \in R_2 \given C_2) P(X_2) \dif{x}
				\end{align*}
				Dabei ist \( P(x \in R_i, C_j) \) die Wahrscheinlichkeit, dass \(x\) zu Klasse \(R_i\) gehört, aber als Klasse \(C_j\) klassifiziert wurde (für \( i \neq j \) entspricht dies einer Fehlklassifikation).

				\begin{figure}
					\centering
					\begin{tikzpicture}
						\begin{axis}[
									name = likelihood,
									domain = 0:15,
									xmin = 0,
									xmax = 15,
									ymin = 0,
									ymax = 0.5,
									xlabel = \(x\),
									width = 12cm,
									height = 7cm,
									legend style = { xshift = -0.5cm, yshift = -0.5cm },
									xticklabels = {,,}
								]
							\addplot [density, tud9b, smooth] { gaussian(x, 5, 2) }; \addlegendentry{\( p(x \given a) \)};
							\addplot [density, tud1b, smooth] { gaussian(x, 11, 2) }; \addlegendentry{\( p(x \given b) \)};
						\end{axis}
						\begin{axis}[
									name = likelihoodPrior,
									domain = 0:15,
									xmin = 0,
									xmax = 15,
									ymin = 0,
									ymax = 0.5,
									xlabel = \(x\),
									width = 12cm,
									height = 7cm,
									at = (likelihood.right of south east),
									anchor = north,
									xshift = -5.23cm,
									yshift = -0.5cm,
									legend style = { xshift = -0.5cm, yshift = -0.5cm },
									xticklabels = {,,}
								]
							\addplot [density, tud9b, smooth] { gaussian(x, 5, 2) * 0.8 }; \addlegendentry{\( p(x \given a) p(a) \)};
							\addplot [density, tud1b, smooth] { gaussian(x, 11, 2) * 0.2 }; \addlegendentry{\( p(x \given b) p(b) \)};
						\end{axis}
						\begin{axis}[
									name = posterior,
									domain = 0:15,
									xmin = 0,
									xmax = 15,
									ymin = 0,
									ymax = 1.1,
									xlabel = \(x\),
									width = 12cm,
									height = 7cm,
									at = (likelihoodPrior.right of south east),
									anchor = north,
									xshift = -5.23cm,
									yshift = -0.5cm,
									legend style = { xshift = -0.5cm, yshift = -0.5cm }
								]
							\addplot [density, tud9b, smooth] { (gaussian(x, 5, 2) * 0.8) / (gaussian(x, 5, 2) * 0.8 + gaussian(x, 11, 2) * 0.2) }; \addlegendentry{\( p(a \given x) \)};
							\addplot [density, tud1b, smooth] { (gaussian(x, 11, 2) * 0.2) / (gaussian(x, 5, 2) * 0.8 + gaussian(x, 11, 2) * 0.2) }; \addlegendentry{\( p(b \given x) \)};
						\end{axis}
					\end{tikzpicture}
					\caption{Likelihood, \( \text{Likelihood} \times \text{Prior} \) und Posterior.}
					\label{fig:likelihoodPriorPosterior}
				\end{figure}
			% end

			\subsection{Entscheidungsregel}
				Durch die Minimierung des Erwartungswertes des Fehlers kann die Entscheidungsregel, wann \(x\) in eine Klasse einsortiert wird, hergeleitet werden. Dabei soll \(x\) genau dann in Klasse \(C_1\) sortiert werden, wenn
				\begin{equation*}
					P(C_1 \given x) > P(C_2 \given x)
				\end{equation*}
				Da die Posteriors im Allgemeinen nicht bekannt sind, werden die über Bayes Theorem berechnet:
				\begin{align*}
					               &  & P(C_1 \given x)                         & > P(C_2 \given x)                     &  \\
					\quad\iff\quad &  & \frac{P(x \given C_1) P(C_1)}{P(x)}     & > \frac{P(c \given C_2) P(C_2)}{P(x)} &  \\
					\quad\iff\quad &  & P(x \given C_1) P(C_1)                  & > P(c \given C_2) P(C_2)              &  \\
					\quad\iff\quad &  & \frac{P(x \given C_1)}{P(x \given C_2)} & > \frac{P(C_1)}{P(C_2)}               &
				\end{align*}
				Dies wird auch \emph{Likelihood Ratio Test} genannt.
				
				Dieser Test kann sich für mehr als zwei Klassen verallgemeinern lassen: Wähle Klasse \( k \) genau dann, wenn
				\begin{equation*}
					P(C_k \given x) > P(C_j \given x) \quad\forall j \neq k
				\end{equation*}
				gilt. Äquivalent zu dem zwei-Klassen-Fall kann dies in einen Likelihood Ratio Test umgeformt werden:
				\begin{equation*}
					\frac{P(x \given C_k)}{P(x \given C_j)} > \frac{P(C_j)}{P(C_k)} \quad\forall j \neq k
				\end{equation*}
			% end

			\subsection{Naive Bayes Classifier}
				Bei mehr als zwei Merkmalen (\zB Höhe und Breite) werden \( P(x_1, x_2 \given C_k) \) und \( P(x_1, x_2) \) mehrdimensional und eine Schätzung der Dichte ist nicht immer möglich. Daher nimmt ein \emph{Naive Bayes Classifier} an, dass die Merkmale statistisch unabhängig sind. Damit gilt:
				\begin{align*}
					P(x_1, x_2 \given C_k) & = P(x_1 \given C_k) P(x_2 \given C_k) \\
					P(x_1, x_2)            & = P(x_1) P(x_2)
				\end{align*}
				In der Realität ist diese Annahme oft nicht korrekt, liefert aber häufig gute Ergebnisse und ist somit eine gute Basis zum Vergleich.
			% end
		% end

		\section{Probability Density Estimation}
			Bisher wurden die Wahrscheinlichkeiten \( P(x \given C_k) \) und \( P(C_k) \) als bekannt vorausgesetzt. In der Realität ist dies oft nicht der Fall, weshalb die Wahrscheinlichkeitsdichte geschätzt werden muss. Siehe hierzu auch Vorlesung \href{https://projects.frisp.org/documents/26}{Statistical Machine Learning}.
		% end

		\section{Gesichtsdetektion}
			Bei \emph{Appearance-Bases Methods} wird ein Erscheinungsmodell aus (üblicherweise) großen Mengen von Bildern gelernt. Dabei wird am häufigsten der Sliding Window Ansatz genutzt (siehe~\ref{sec:sliding_window}). Dabei sind vor allem drei Aspekte relevant:
			\begin{enumerate}
				\item Repräsentation des Objektes (lokale Merkmale, globale Anordnung)
				\item Trainingsdaten (positive und negative Beispiele)
				\item Klassifikator und Lernmethode
			\end{enumerate}

			\subsection{Sliding Window Ansatz}
				\label{sec:sliding_window}
				
				Bei dem \emph{Sliding Window Ansatz} wird ein Bild in Ein-Pixel-Schritten horizontal und vertikal gescannt. Nach jedem Durchlauf wird das Bild immer wieder verkleinert, bis das Bild zu klein ist. So können auch mit einem Klassifikator, der nur Bilder einer Größe entgegen nehmen kann, große Bilder durchsucht werden.
			% end

			\subsection{Beispiel: Gesichtsdetektion}
				\begin{enumerate}
					\item Repräsentation des Objekts
						\begin{itemize}
							\item Die Bilder werden in Wavelets zerlegt, \dh die Gesichtsmerkmale werden mit Frequenzen und deren Ort und Orientierung dargestellt.
							\item Lokale Merkmale: Wavelet Koeffizienten (Frequenzen von \zB Auge und Mund).
							\item Globale Merkmale: Absolute Position der Frequenzen im Bild.
						\end{itemize}
					\item Trainingsdaten
						\begin{itemize}
							\item Positive Beispiele
								\begin{itemize}
									\item Möglichst vielfältig.
									\item Jedes Bild eines Gesichts wird manuell an den Rändern abgeschnitten und auf eine Größe normalisiert.
									\item Zusätzlich werden virtuelle Beispiele erstellt (\zB durch Spiegelung).
								\end{itemize}
							\item Negative Beispiele
								\begin{itemize}
									\item Beliebige Bilder, die keine Gesichter enthalten.
									\item Teilbilder von großen Bildern.
								\end{itemize}
						\end{itemize}
					\item Klassifikator und Lernmethode
						\begin{itemize}
							\item Naive Bayes Classifier
							\item Merkmale \(x_i\): Wavelet Koeffizienten an einer bestimmten Position.
							\item Zwei-Klasse-Problem:
								\begin{itemize}
									\item \(C_1\): Gesichter
									\item \(C_2\): Alles andere (keine Gesichter)
								\end{itemize}
							\item Das "Lernen" entspricht dem Schätzen der Wahrscheinlichkeiten der Wavelet-Koeffizienten.
							\item Durch Diskretisierung von Koeffizienten und Positionen gibt es eine diskrete und endliche Anzahl von \(x_i\).
							\item Schätzen: Zählen, wie häufig jedes \(x_i\) in Bilder mit und ohne Gesichtern vorkommt.
							\item Dann wird ein Likelihood Ration Test verwendet.
						\end{itemize}
				\end{enumerate}
			
				Um Bilder aus verschiedenen Perspektiven zu erkennen, wird für jede Ansicht ein eigener Detektor verwendet (jeder für eine Ansicht) und diese kombiniert.
			% end

			\subsection{Erkennungsarten}
				Eine Gesichtserkennung zählt zu den biometrischen Verfahren und werden \bspw in sicherheitstechnischen, kriminalistischen und forensischen Gebieten eingesetzt. Der Zweck ist dir Identifikation und Verifikation natürlicher Personen.
				\begin{itemize}
					\item Verifikation: Die Person muss dem System ihren Namen oder User-ID mitteilen und das System entscheidet, ob die Person dazu gehört.
					\item Identifikation: Die Person offenbart ausschließlich ihre biometrischen Merkmale und das System ermittelt daraus den Namen oder die User-ID.
				\end{itemize}
			% end
		% end
	% end

	\chapter{Fouriertheorie}
		Bei der Beugung an einem einfachen Spalt der breite \(a\) ergibt sich auf dem Schirm ein Beugungsmuster, welches im Zentrum ein Intensitätsmaximum und nach außen hin immer wieder Intensitätsminima und -maxima hat. Der Spalt kann durch eine Rechteckfunktion
		\begin{equation*}
			\text{Rect}(x) =
				\begin{cases}
					1 & -1 \leq x \leq 1 \\
					0 & \text{sonst}
				\end{cases}
		\end{equation*}
		beschrieben werden. Das sich ergebende Beugungsmuster, \bzw die zeitlich gemittelte Intensität \(I\), hat dann die Form
		\begin{equation*}
			I(\theta) = I_0 \, \Bigg( \frac{\sin(\theta)}{\theta} \Bigg)^2 = I_0 \cdot \sinc^2(\theta)
		\end{equation*}
		mit der \(\sinc\)-Funktion \( \sinc(\theta) = \sin(\theta) / \theta \). Dabei stellt \(\theta\) den Ausfallwinkel des Lichts aus dem Spalt hinaus dar.
		
		Dieser Zusammenhang zwischen der Gestalt des beugenden Objekts (hier der Spalt) und der Amplitudenfunktion \( I(\theta) \) ist durch eine \emph{Fourier-Transformation} gegeben.

		\section{Mathematische Grundlagen}
			\subsection{Vektorraum}
				Ein \emph{Vektorraum} ist eine algebraische Struktur über einen Zahlenbereich mit Operationen wie Addition und Multiplikationen mit einem Skalar. Alle Operationen müssen dabei Elemente des Vektorraums wieder auf selbigen abbilden. Die Elemente eines solchen Raums sind \emph{Vektoren}.
				
				\subparagraph{Beispiel}
					Ein Beispiel ist er euklidische Vektorraum über den reellen Zahlen. Dabei repräsentieren Vektoren Verschiebungen und es lassen sich Längen und Winkel messen (rechtwinkliges, kartesisches Koordinatensystem). Es ist außerdem ein Skalarprodukt definiert:
					\begin{gather*}
						\langle \vec{v}, \vec{w} \rangle = \sum_{i = 1}^{n} v_i w_i \in \R \\
						\langle \vec{v}, \vec{w} \rangle = v_1 w_2 + v_2 w_2 = \lVert \vec{v} \rVert \cdot \lVert \vec{w} \rVert \cos \big( \angle(\vec{v}, \vec{w}) \big)
					\end{gather*}
					Die letztere Eigenschaft gilt nur für \( n = 2 \) (\iA lassen sich solche Winkel aber auch mit beliebigem \(n\) definieren). In der euklidischen Ebene \( \R^2 \) lassen sich Vektoren durch Ortsvektoren (Pfeile) darstellen.
				% end
			% end

			\subsection{Basis eines Vektorraums}
				Jeder Satz (Menge) an linear unabhängigen Vektoren eines Vektorraums kann als Basis verwendet werden. Zwei Vektoren \(\vec{v}\), \(\vec{w}\) sind genau dann linear unabhängig, wenn \( \big\lvert \langle \vec{v}, \vec{w} \rangle \big\rvert < \lVert \vec{v} \rVert \cdot \lVert \vec{w} \rVert \) gilt.
			
				\subparagraph{Beispiel}
					In der euklidischen Ebene \( \R^2 \) ist eine Basis durch
					\begin{equation*}
						\vec{e}_1 = \begin{bmatrix} 1 \\ 0 \end{bmatrix} \quad\quad \vec{e}_2 = \begin{bmatrix} 0 & 1 \end{bmatrix}
					\end{equation*}
					gegeben, wobei \(\vec{e}_1\) und \(\vec{e}_2\) orthogonal aufeinander stehen (\( \langle \vec{v}, \vec{w} \rangle = 0 \)) und somit linear unabhängig sind. Alle \( \vec{v} \in \R^2 \) lassen sich dann als \emph{Linearkombination} der Basisvektoren darstellen (mit geeigneten \( a_1, a_2 \in \R \)):
					\begin{equation*}
						\vec{v} = a_1 \vec{e}_1 + a_2 \vec{e}_2
					\end{equation*}
				% end
			% end

			\subsection{Krummlinige Koordinatensysteme}
				Gerade in physikalischen Anwendungen kann es von Vorteil sein, keine kartesischen Koordinaten (mit \(x\)- und \(y\)-Wert) zu nutzen, sondern auf \emph{krummlinige Koordinaten} umzusteigen. Ein typisches krummliniges Koordinatensystem sind \zB Polarkoordinaten. Dabei wird ein Punkt in der Ebene durch den Abstand \(r\) vom Ursprung und durch den Winkel \(\varphi\) mit der \(x\)-Achse beschrieben. Die Koordinaten lassen sich durch
				\begin{align*}
					x(r, \varphi) &= r \cdot \cos(\varphi) \\
					y(r, \varphi) &= r \cdot \sin(\varphi)
				\end{align*}
				in kartesische Koordinaten umrechnen.
				
				Weitere krummlinige Koordinatensysteme sind \zB Kugel- oder Zylinderkoordinaten.
			% end

			\subsection{Andere Räume}
				Es ist auch möglich, dass die Elemente eines Vektorraums Funktionen sind (Funktionenräume). Auch kann ein Raum unendlich-dimensional sein.
				
				Die \emph{Fourier-Theorie} beschäftigt sich mit der Frage, ob es möglich ist, Basisfunktionen zu finden, mit denen sich beliebige Funktionen \bzgl dieser Basen darstellen lassen.
			% end

			\subsection{Komplexe Zahlen} % 4.20, 4.48
				Komplexe Zahlen haben zwei Komponenten: Einen Real- und einen Imaginärteil. Dabei können sie als kartesische Koordinaten in einer zwei-dimensionalen Ebene (der komplexen Ebene) aufgefasst werden und entsprechen dargestellt werden (mit der imaginären Zahl \( i \) mit der Eigenschaft \( i^2 = -1 \)):
				\begin{equation*}
					z = a + bi
				\end{equation*}
				Oder als Polarkoordinaten (in einer zwei-dimensionalen Ebene) mit der Darstellung
				\begin{equation*}
					z = r e^{\varphi i}
				\end{equation*}
				wobei sich kartesische und Polardarstellung wie bei Polarkoordinaten ineinander umrechnen lassen.
				
				Die Äquivalenz der beiden Darstellung geht auf die Euler-Identität
				\begin{equation*}
					e^{i \varphi} = \cos(\varphi) + i \sin(\varphi)
				\end{equation*}
				zurück, wobei hier \( r = 1 \) gilt. Aus aus dieser folgt (für \( \lvert z \rvert = 1 \)) ebenfalls:
				\begin{align}
					a = \cos(\varphi) &= \frac{1}{2} \big( e^{i \varphi} + e^{-i \varphi} \big) \\
					b = \sin(\varphi) &= \frac{1}{2i} \big( e^{i \varphi} - e^{-i \varphi} \big)
				\end{align}
			% end

			\subsection{Gerade/Ungerade Funktionen}
				Für eine gerade Funktion gilt
				\begin{equation*}
					f(x) = f(-x)
				\end{equation*}
				für eine ungerade Funktion gilt
				\begin{equation*}
					f(x) = -f(-x)
				\end{equation*}
				für jeweils alle \(x\).
			% end
		% end

		\section{Fourier-Reihe}
			\subsection{Dirichlet-Bedingungen}
				Jede Funktion, die die \emph{Dirichlet-Bedingungen} erfüllt:
				\begin{enumerate}
					\item Die Anzahl Unstetigkeiten innerhalb einer Periode ist endlich.
					\item Die Anzahl Maxima und Minima innerhalb einer Periode ist endlich.
					\item Die Funktion ist in jeder Periode integrierbar (\dh die Fläche unter dem Betrag der Funktion ist endlich).
				\end{enumerate}
				Kann durch eine Summe von Kosinus- und Sinusfunktionen dargestellt werden.
			% end

			\subsection{\(2\pi\)-periodische Funktion}
				Ist \( f(x) \) eine periodische Funktion mit der Periodenlänge \( 2\pi \) (\dh die wiederholt sich alle \(2\pi\)), die die Dirichlet-Bedingungen erfüllt, so gilt
				\begin{equation*}
					f(x) = \sum_{n = 0}^{\infty} \big( a_n \cos(nx) + b_n \sin(nx) \big)
				\end{equation*}
				mit geeigneten \emph{Fourier-Koeffizienten} \( a_n \) und \( b_n \).
			% end

			\subsection{Skalarprodukt, Orthogonale Basis}
				Sei \(H\) der Raum aller \(2\pi\)-periodischen reellen Funktionen, die die Dirichlet-Bedingungen erfüllen. Dann wird durch
				\begin{equation*}
					\langle f, g \rangle \coloneqq \int_{-\pi}^{\pi} \! f(\tau) g(\tau) \dif{\tau}
				\end{equation*}
				ein Skalarprodukt definiert.
				
				Die Funktionen
				\begin{align*}
					u_n(x) &= \cos(nx) \\
					v_n(x) &= \sin(nx)
				\end{align*}
				bilden dann eine orthogonale Funktionenfolge in \(H\):
				\begin{align*}
					\langle u_n, u_m \rangle &=
						\begin{cases}
							0    & m \neq n  \\
							2\pi & m = n = 0 \\
							\pi  & m = n > 0
						\end{cases} \\
					\langle v_n, v_m \rangle &=
						\begin{cases}
							0   & m \neq n  \\
							0   & m = n = 0 \\
							\pi & m = n > 0
						\end{cases} \\
					\langle u_n, v_m \rangle = \langle v_m, u_n \rangle &= 0
				\end{align*}
				
				Durch diese Darstellung kann die allgemeine Fourier-Reihe mit \( u_n = u_n(x) \) und \( v_n = v_n(x) \) auch geschrieben werden als:
				\begin{equation*}
					f(x) = \sum_{n = 0}^{\infty} \big( a_n u_n + b_n v_n \big)
				\end{equation*}
			% end

			\subsection{Berechnung der Koeffizienten \(a_m\), \(b_m\)}
				Um die Koeffizienten \( a_m \), \( m = 1, 2, \cdots \) zu bestimmen, wird das Skalarprodukt zwischen \(f\) und \( u_m \) gebildet:
				\begin{equation*}
					\langle f, u_m \rangle
						= \Bigg\langle \sum_{n = 0}^{\infty} \big( a_n u_n + b_n v_n \big),\, u_m \Bigg\rangle
						= \Big\langle \big( a_m u_m + b_m v_m \big),\, u_m \Big\rangle
						= \langle a_m u_m,\, u_m \rangle
						= a_m \langle u_m, u_m \rangle
						= a_m \pi
				\end{equation*}
				Umstellen nach \( a_m \) liefert die Werte der Fourier-Koeffizienten:
				\begin{equation*}
					a_m = \frac{1}{\pi} \langle f, u_m \rangle = \frac{1}{\pi} \int_{-\pi}^{\pi} \! f(x) \cos(mx) \dif{x}
				\end{equation*}
				
				Analog für \( a_0 \) mit \( u_0 \):
				\begin{equation*}
					\langle f, u_0 \rangle
						= \Bigg\langle \sum_{n = 0}^{\infty} \big( a_n u_n + b_n v_n \big),\, u_0 \Bigg\rangle
						= \Big\langle \big( a_0 u_0 + b_0 v_0 \big),\, u_0 \Big\rangle
						= \langle a_0 u_0,\, u_0 \rangle
						= a_0 \langle u_0, u_0 \rangle
						= a_0 2\pi
				\end{equation*}
				Umstellen nach \( a_0 \):
				\begin{equation*}
					a_0 = \frac{1}{2\pi} \langle f, u_0 \rangle = \frac{1}{2\pi} \int_{-\pi}^{\pi} \! f(x) \cos(0x) \dif{x} = \frac{1}{2\pi} \int_{-\pi}^{\pi} f(x) \dif{x}
				\end{equation*}
				
				Analog für \( b_m \), \( m = 1, 2, \cdots \):
				\begin{equation*}
					\langle f, v_m \rangle
						= \Bigg\langle \sum_{n = 0}^{\infty} \big( a_n u_n + b_n v_n \big),\, v_m \Bigg\rangle
						= \Big\langle \big( a_m u_m + b_m v_m \big),\, v_m \Big\rangle
						= \langle b_m v_m, v_m \rangle
						= b_m \langle v_m, v_m \rangle
						= b_m \pi
				\end{equation*}
				Umstellen nach \( b_m \):
				\begin{equation*}
					b_m = \frac{1}{\pi} \langle f, v_m \rangle = \frac{1}{\pi} \int_{-\pi}^{\pi} \! f(x) \sin(mx) \dif{x}
				\end{equation*}
				Da \( \sin(0x) = \sin(0) = 0 \) ist, muss \( b_0 \) nicht berechnet werden.
			% end

			\subsection{Beispiel: Rechteck-Schwingung}
				Sei eine Rechteck-Schwingung
				\begin{equation*}
					f(x) =
						\begin{cases}
							-k & -\pi < x < 0 \\
							k  & 0 < x < \pi
						\end{cases},\quad f(x) = f(x + 2\pi)
				\end{equation*}
				gegeben. Für diese lauten die Fourier-Koeffizienten:
				\begin{align*}
					a_0 &= 0 \\
					a_n &= 0 \\
					b_n &= \frac{4k}{n\pi} \text{ für ungerade } n
				\end{align*}
				Daraus ergibt sich die Fourier-Reihe:
				\begin{equation*}
					f(x) = \frac{4k}{\pi} \sum_{n = 0}^{\infty} \frac{1}{2k + 1} \sin\big((2k + 1) x\big)
				\end{equation*}
				
				Die Rechteck-Schwingung ist dabei eine ungerade Funktion. Allgemein gilt:
				\begin{itemize}
					\item Für gerade Funktionen sind alle \( b_n = 0 \).
					\item Für ungerade Funktionen sind all \( a_n = 0 \).
				\end{itemize}
			% end
		% end

		\section{Fourier-Transformation}
			Mit der Fourier-Transformation wird versucht, eine ähnliche Darstellung wie die Fourier-Reihe für Funktionen zu finden, die nicht \(2\pi\)-periodisch sind.
			
			Durch die Euler-Identität kann die allgemeine Fourier-Reihe umgeformt werden:
			\begin{align*}
				f(x)
					&= a_0 + \sum_{n = 1}^{\infty} \big( a_n \cos(nx) + b_n \sin(nx) \big) \\
					&= a_0 + \sum_{n = 1}^{\infty} \Bigg( a_n \frac{e^{inx} + e^{-inx}}{2} + b_n \frac{e^{inx} - e^{-inx}}{2i} \Bigg) \\
					&= a_0 + \sum_{n = 1}^{\infty} \Bigg( a_n \frac{e^{inx} + e^{-inx}}{2} - b_n i \frac{e^{inx} - e^{-inx}}{2} \Bigg) \\
					&= a_0 + \sum_{n = 1}^{\infty} \Bigg( \frac{a_n - ib_n}{2} e^{inx} + \frac{a_n + ib_n}{2} e^{-inx} \Bigg) \\
					&= a_0 + \sum_{n = 1}^{\infty} \frac{a_n - ib_n}{2} e^{inx} + \sum_{n = 1}^{\infty} \frac{a_n + ib_n}{2} e^{-inx} \\
					&= a_0 + \sum_{n = 1}^{\infty} \frac{a_n - ib_n}{2} e^{inx} + \sum_{n = -\infty}^{-1} \frac{a_{-n} + ib_{-n}}{2} e^{inx} \\
					&= c_0 + \sum_{n = 1}^{\infty} c_n e^{inx} + \sum_{n = -\infty}^{-1} c_n e^{inx} \\
					&= \sum_{n = -\infty}^{\infty} c_n e^{inx}
			\end{align*}
			woraus sich eine äquivalente Formulierung der Fourier-Reihe mit den komplexen Koeffizienten
			\begin{equation*}
				c_n = \frac{a_n - ib_n}{2} e^{inx}, \quad n = 1, 2, \cdots \quad\quad\quad\quad c_n = \frac{a_{-n} + ib_{-n}}{2}, \quad n = -1, -2, \cdots \quad\quad\quad\quad c_0 = a_0
			\end{equation*}
			ergibt. Nun werden zunächst Funktionen \( f_L(x) \) mit einer beliebigen Periode \( 2L \) betrachtet:
			\begin{align*}
				f_L(x) &= \sum_{n = -\infty}^{\infty} c_n e^{in \frac{2\pi}{2L} x} \\
					 &= \sum_{n = -\infty}^{\infty} c_n e^{in \frac{\pi}{L} x} \\
				\intertext{Einsetzen der Koeffizienten \(c_n\):}
					&= \sum_{n = -\infty}^{\infty} \Bigg( \frac{1}{2L} \int_{-L}^{L} \! f(\tau) e^{-in \frac{\pi}{L} \tau} \dif{\tau} \Bigg) e^{in \frac{\pi}{L} x}
			\end{align*}
			Nun wird der Übergang \( L \to \infty \), \dh zu nicht-periodischen Funktionen, betrachtet:
			\begin{align*}
				\lim\limits_{L \to \infty} f(x)
					&= \lim\limits_{L \to \infty} \sum_{n = -\infty}^{\infty} \Bigg( \frac{1}{2L} \int_{-L}^{L} \! f(\tau) e^{-in \frac{\pi}{L} \tau} \dif{\tau} \Bigg) e^{in \frac{\pi}{L} x} \\
					&= \lim\limits_{L \to \infty} \sum_{n = -\infty}^{\infty} \frac{1}{2L} \int_{-L}^{L} \! f(\tau) e^{-in \frac{\pi}{L} (\tau - x)} \dif{\tau} \\
					&= \lim\limits_{L \to \infty} \sum_{n = -\infty}^{\infty} \int_{-L}^{L} \! \frac{1}{2L} f(\tau) e^{-in \frac{\pi}{L} (\tau - x)} \dif{\tau} \\
					&= \lim\limits_{L \to \infty} \int_{-L}^{L} \sum_{n = -\infty}^{\infty} \frac{1}{2L} f(\tau) e^{-in \frac{2\pi}{2L} (\tau - x)} \dif{\tau} \\
					&= \lim\limits_{L \to \infty} \int_{-L}^{L} \! f(\tau) \int_{-\infty}^{\infty} \! e^{-2\pi i u (\tau - x)} \dif{u} \dif{\tau} \\
					&= \int_{-\infty}^{\infty} \! f(\tau) \int_{-\infty}^{\infty} \! e^{-2\pi i u (\tau - x)} \dif{u} \dif{\tau}
			\end{align*}
			Dieser Übergang lässt sich als "Superposition" auffassen mit:
			\begin{align*}
				f(x) &= \int_{-\infty}^{\infty} \! F(u) e^{2\pi i u x} \dif{u} \\
				F(x) &= \int_{-\infty}^{\infty} \! f(\tau) e^{-2\pi i u \tau} \dif{\tau}
			\end{align*}
			
			Dabei heißt der Übergang \( f(x) \to F(u) \) \emph{Fourier-Transformation} und der Übergang \( F(u) \to f(x) \) \emph{Inverse Fourier-Transformation}. Dabei ist \(F(u)\) oft komplex und \(f(x)\) ist reell.

			\subsection{Beispiel: Rechteckimpuls}
				Für einen Rechteckimpuls
				\begin{equation*}
					f(x) =
						\begin{cases}
							1 & -1 < x < 1 \\
							0 & \text{sonst}
						\end{cases}
				\end{equation*}
				ergibt sich die Fourier-Transformation
				\begin{equation*}
					F(u) = \int_{-1}^{1} \! e^{-2\pi i u \tau} \dif{\tau} = \frac{1}{2\pi i u} \big[ e^{-2\pi i u \tau} \big]_{-1}^{1} = \frac{1}{\pi u} \cdot \frac{e^{2\pi i u} - e^{-2\pi i u}}{2i} = 2 \frac{\sin(2\pi u)}{2\pi u} = 2 \sinc(2\pi u)
				\end{equation*}
				wie erwartet ein Vielfaches der \( \sinc \)-Funktion.
			% end

			\subsection{Transformationspaare}
				Die Fourier-Transformation zerlegt eine Funktion in ihre Frequenzbestandteile! Beispielhafte Fourier-Transformationspaare sind:
				\begin{itemize}
					\item \( \cos(0) = 1 \): \tabto{2.5cm} Delta-Funktion bei \( u = 0 \)
					\item \( \cos(kx) \):    \tabto{2.5cm} Delta-Funktion bei \( u = \pm k \)
					\item \( \sin(kx) \):    \tabto{2.5cm} Delta-Funktion bei \( u = \pm ik \)
				\end{itemize}
			% end
		% end

		\section{Faltung}
			Werden zwei Funktionen \( F(u) \), \( G(u) \) im Frequenzraum multipliziert:
			\begin{align*}
				F(u) \cdot G(u)
					&= \int_{-\infty}^{\infty} \! f(\tau) e^{-2\pi i u \tau} \dif{\tau} \cdot \int_{-\infty}^{\infty} \! g(t) e^{-2\pi i u t} \dif{t} \\
					&= \int_{-\infty}^{\infty} \! f(\tau) e^{-2\pi i u \tau} \int_{-\infty}^{\infty} \! g(t - \tau) e^{-2\pi i u (t - \tau)} \dif{\tau} \dif{t} \\
					&= \int_{-\infty}^{\infty} \! e^{-2\pi i u t} \underbrace{\int_{-\infty}^{\infty} \! f(\tau) g(t - \tau) \dif{\tau}}_{h(t) \coloneqq} \dif{t} \\
					&= \int_{-\infty}^{\infty} \! h(t) e^{-2\pi i u t} \dif{t} \\
					&= H(t)
			\end{align*}
			Das Integral \( h(t) = \int_{-\infty}^{\infty} \! h(t) e^{-2\pi i u t} \dif{t} \eqqcolon f(t) \ast g(t) \) ist das sogenannte \emph{Faltungsintegral} der Funktionen \(f\) und \(g\). Eine Faltung im Ortsraum entspricht somit einer Multiplikation im Frequenzraum!
			
			Eine Faltung \( f(t) \ast g(t) \) kann als Mittelwertbildung der Werte von \(f\) mit Gewichten \(g\) verstanden werden. So kann \bspw analytisch ein gleitender Durchschnitt (mit einer Kastenfunktion \(g\)) erstellt werden.
			
			\subsection{Anwendung: Filter} % 4.68
				\todo{Content}
			% end
		% end

		\section{Abtastung}
			Ist eine kontinuierliche Funktion, \bzw ein analoges Signal, gegeben, so muss dieses für eine diskrete Repräsentation \emph{abgetastet} werden, \dh es müssen Messungen an einzelnen Stellen durchgeführt werden. Eine solche diskrete Abtastung kann durch die Funktion
			\begin{equation*}
				\hat{f}(x) = f(x) \cdot \sum_{n = -\infty}^{\infty} \delta(x - n \cdot \Delta x)
			\end{equation*}
			\dh als Produkt einer Funktion \( f(x) \) und einer Kamm-Funktion beschrieben werden. Die Fourier-Transformierte \( \hat{F}(u) \) der abgetasteten Funktion entspricht dann der Fourier-Transformierten \( F(u) \) der nicht abgetasteten Funktion, wird aber periodisch mit der Periode \( 1/\Delta x \) wiederholt und mit \( 1/\Delta x \) skaliert.

			\subsection{Abtasttheorie}
				Sei die Funktion \( f(x) \) bandbegrenzt durch eine Maximalfrequenz \( u_G \), \dh \( F(u) = 0 \) für \( \lvert u \rvert > u_G \).
				
				Gilt nun \( 2u_G < 1 / \Delta x \), so überlappen sich die Fouriertransformierten nicht, \dh die Spektren von \( F(u) \) und \( \hat{F}(u) \) stimmen auf dem Intervall \( [-u_G, u_G] \) (bis auf die Skalierung \( 1 / \Delta x \)) überein. Das Frequenzspektrum von \( F(u) \) kann somit vollständig aus dem Abtastsignal und den Abtastwerten berechnet werden.
				
				Gilt nun \( 2u_G > 1 / \Delta x \), so überlappen sich die Fouriertransformierten und in den Überschneidungsbereichen bilden sich Summen. Damit ist es unmöglich, das originale Frequenzspektrum von \( F(u) \) zu bestimmen (\emph{Aliasing}).
			% end

			\subsection{Abtasttheorem von Whittaker-Shannon}
				Aus den vorherigen Überlegungen ergibt sich das \emph{Abtasttheorem von Whittaker-Shannon}: Existiert für eine Funktion \( f(x) \) eine Grenzfrequenz \( u_G < \infty \), sodass \( F(u) = 0 \) für \( \lvert u \rvert > u_G \) gilt, dann ist \( f(x) \) fehlerfrei rekonstruierbar, sofern die Abtastfrequenz \( 1 / \Delta x \) mindestens doppelt so hoch wie \( u_G \) ist:
				\begin{equation*}
					\frac{1}{\Delta x} > 2 u_G
				\end{equation*}
			% end
		% end
	% end

	\chapter{Bilder} % 5.1, 5.3, 5.4
		\todo{Content}

		\section{Bildverbesserung} % 5.8, 5.9, 5.10, 5.11
			\todo{Content}

			\subsection{Histogramm} % 5.14, 5.15, 5.16, 5.17, 5.18, 5.19
				\todo{Content}
			% end

			\subsection{Pixeloperationen} % 5.12, 5.13, 5.20
				\todo{Content}

				\subsubsection{Bildnegativ} % 5.21
					\todo{Content}
				% end

				\subsubsection{Binärisierung/Thresholding} % 5.22
					\todo{Content}
				% end

				\subsubsection{Graufenfensterung} % 5.23
					\todo{Content}
				% end
			% end

			\subsection{Kontrastspreizung} % 5.24
				\todo{Content}
			% end

			\subsection{Histogrammausgleich} % 5.25, 5.26, 5.27
				\todo{Content}
			% end

			\subsection{Mittelung} % 5.28
				\todo{Content}
			% end
		% end

		\section{Bildfilterung} % 5.29, 5.30
			\todo{Content}

			\subsection{Ortsraum} % 5.31
				\todo{Content}

				\subsubsection{Lineare Filterung (Faltung)} % 5.32
					\todo{Content}
				% end

				\subsubsection{Tiefpass-Filter} % 5.33
					\todo{Content}
				% end

				\subsubsection{Mittelwert-Filter} % 5.34, 5.35
					\todo{Content}
				% end

				\subsubsection{Gauss-Filter} % 5.36, 5.37, 5.38
					\todo{Content}
				% end

				\subsubsection{Median-Filter} % 5.39
					\todo{Content}
				% end

				\subsubsection{Hochpass-Filter} % 5.40
					\todo{Content}
				% end

				\subsubsection{Diskretisierte Ableitungen} % 5.42, 5.43
					\todo{Content}

					\paragraph{Laplacian-Filter} % 5.44, 5.47, 5.48
						\todo{Content}
					% end

					\paragraph{Laplacian of Gaussian Filter} % 5.45, 5.46
						\todo{Content}
					% end
				% end

				\subsubsection{Bilateral Filter} % 6.10
					\todo{Content}
				% end
			% end

			\subsection{Frequenzraum} % 5.51, 5.52, 5.62, 5.63, 5.64, 5.65
				\todo{Content}

				\subsubsection{Idealer Tiefpass-Filter} % 5.66
					\todo{Content}
				% end

				\subsubsection{Gaußscher Tiefpass-Filter} % 5.67, 5.68
					\todo{Content}
				% end

				\subsubsection{Idealer Hochpass-Filter} % 5.70
					\todo{Content}
				% end
			% end

			\subsection{Vergleich: Orts- und Frequenzraum-Filter} % 5.71
				\todo{Content}
			% end
		% end

		\section{Bildkompression} % 5.72, 5.73, 5.74, 5.75, 5.76, 5.77, 5.96
			\todo{Content}

			\subsection{Harmonische Transformation} % 5.78
				\todo{Content}

				\subsubsection{JPEG} % 5.79, 5.95
					\todo{Content}

					\paragraph{Schritt 1: Umwandlung in den YCC-Farbraum} % 5.80
						\todo{Content}
					% end

					\paragraph{Schritt 2: Farb-Subsampling} % 5.81
						\todo{Content}
					% end

					\paragraph{Schritt 3: Diskrete Kosinustransformation} % 5.82, 5.83, 5.84
						\todo{Content}
					% end

					\paragraph{Schritt 4: Quantisierung} % 5.85
						\todo{Content}
					% end

					\paragraph{Schritt 5: Kodierung der Koeffizienten} % 5.86, 5.87
						\todo{Content}
					% end
				% end

				\subsubsection{MH17} % 6.6, 6.8
					\todo{Content}
				% end
			% end
		% end
	% end

	\chapter{Bildverarbeitung} % 6.1, 6.90
		\todo{Content}

		\section{Deblurring} % 6.13, 6.15, 6.41
			\todo{Content}

			\subsection{Inverser Operator} % 6.19, 6.20
				\todo{Content}
			% end

			\subsection{Rekonstruktion} % 6.17, 6.21
				\todo{Content}
			% end

			\subsection{Rauschen} % 6.23, 6.26
				\todo{Content}
			% end

			\subsection{Korrekt gestellte Probleme} % 6.27, 6.28, 5.29
				\todo{Content}
			% end
		% end

		\section{Einschrittverfahren} % 6.30, 6.41
			\todo{Content}

			\subsection{Wiener Filter} % 6.31, 6.32, 6.33, 6.34, 6.35, 6.36
				\todo{Content}
			% end

			\subsection{Ansatz mit mehreren Komponenten} % 6.37
				\todo{Content}

				\subsubsection{Scale-Space-Ansatz} % 6.38, 6.39, 6.40, 6.43
					\todo{Content}
				% end

				\subsubsection{Gaußscher Scale-Space} % 6.44, 6.45
					\todo{Content}
				% end
			% end
		% end

		\section{Mehrschrittverfahren (Iterative Methoden)} % 6.47, 6.67
			\todo{Content}

			\subsection{Variationsableitung} % 6.48, 6.49, 6.50
				\todo{Content}
			% end

			\subsection{Alternativen} % 6.53
				\todo{Content}
			% end

			\subsection{Perona-Malik} % 6.54
				\todo{Content}

				\subsubsection{Nichtlineare Diffusionsgleichungen} % 6.55
					\todo{Content}
				% end

				\subsubsection{Die Perona-Malik-Gleichung} % 6.56, 6.57, 6.58
					\todo{Content}
				% end

				\subsubsection{Implementierung} % 6.59, 6.60
					\todo{Content}
				% end

				\subsubsection{Einfluss und Wahl von \(k\)} % 6.61
					\todo{Content}
				% end

				\subsubsection{Stoppzeit} % 6.62, 6.63, 6.64
					\todo{Content}
				% end
			% end

			\subsection{Eingeschränkte Evolution: Totale Variation} % 6.68, 6.69
				\todo{Content}

				\subsubsection{Distance Penalty} % 6.69, 6.70, 6.71
					\todo{Content}
				% end

				\subsubsection{Basisenergien} % 6.72
					\todo{Content}
				% end

				\subsubsection{Totale Variation} % 6.73, 6.74
					\todo{Content}
				% end

				\subsubsection{Erweiterungen} % 6.81
					\todo{Content}
				% end
			% end
		% end
	% end

	\chapter{Grafikpipeline} % 7.1
		\todo{Content}

		\section{Hardware} % 7.8, 7.9, 7.14
			\todo{Content}

			\subsection{Large-Scale-Computing} % 7.10
				\todo{Content}
			% end

			\subsection{Personal/Desktop Computing} % 7.11
				\todo{Content}
			% end

			\subsection{Networked Computing} % 7.12
				\todo{Content}
			% end

			\subsection{Mobile Computing} % 7.13
				\todo{Content}
			% end

			\subsection{Collaborative Computing} % 7.15
				\todo{Content}
			% end

			\subsection{Virtual Reality} % 7.16
				\todo{Content}
			% end

			\subsection{Augmented Reality} % 7.18, 7.19, 7.20
				\todo{Content}
			% end

			\subsection{Ambient/Invisible} % 7.21
				\todo{Content}
			% end

			\subsection{Wearable/Ubiquitous} % 7.22
				\todo{Content}
			% end
		% end

		\section{Computergrafik} % 7.23, 7.24, 7.25
			\todo{Content}

			\subsection{Geschichte} % N/A
				\todo{Content}

				\subsubsection{Die Anfänge} % 7.26, 7.27
					\todo{Content}
				% end

				\subsubsection{60er Jahre} % 7.28
					\todo{Content}
				% end

				\subsubsection{70er Jahre} % 7.29
					\todo{Content}
				% end

				\subsubsection{80er Jahre} % 7.31
					\todo{Content}
				% end

				\subsubsection{90er Jahre} % 7.32
					\todo{Content}
				% end

				\subsubsection{2000 bis 2005} % 7.33
					\todo{Content}
				% end

				\subsubsection{2006 bis 2020} % 7.34, 7.35
					\todo{Content}
				% end
			% end

			\subsection{Virtuelle Charaktere} % 7.36, 7.37, 7.38, 7.39
				\todo{Content}
			% end
		% end

		\section{Grafikpipeline} % 7.41, 7.42, 7.43, 7.61, 7.87, 7.98, 7.100
			\todo{Content}
		% end

		\section{Eingabe grafischer Daten} % 7.44, 7.45, 7.46
			\todo{Content}
		% end

		\section{Repräsentation von 3D-Daten} % N/A
			\todo{Content}

			\subsection{Grafische Primitive} % 7.47, 7.48, 7.49, 7.50
				\todo{Content}
			% end

			\subsection{Transformationen} % 7.51, 7.52
				\todo{Content}
			% end
		% end

		\section{Räumliche Datenstrukturen} % 7.53, 7.54, 7.55
			\todo{Content}

			\subsection{Hüllkörper (Bounding Volumes)} % 7.56
				\todo{Content}
			% end

			\subsection{Hüllkörperhierarchien} % 7.57
				\todo{Content}
			% end

			\subsection{Raumunterteilung} % N/A
				\todo{Content}

				\subsubsection{Achsenparallele Gitter (Grids)} % 7.58
					\todo{Content}
				% end

				\subsubsection{Quadtree/Octree} % 7.59
					\todo{Content}
				% end

				\subsubsection{Binary Space Partition} % 7.60
					\todo{Content}
				% end
			% end
		% end

		\section{Geometrieverarbeitung} % N/A
			\todo{Content}

			\subsection{Modelltransformation (Orientierung, Position)} % 7.62, 7.63
				\todo{Content}
			% end

			\subsection{Simulation der Beleuchtung} % 7.64, 7.65
				\todo{Content}

				\subsubsection{Phong-Beleuchtungsmodell} % 7.66, 7.67
					\todo{Content}
				% end
			% end

			\subsection{Perspektivische Transformation und Clipping (Abschneiden)} % 7.78, 7.79
				\todo{Content}

				\subsubsection{Painters Algorithmus} % 7.80
					\todo{Content}
				% end
			% end

			\subsection{Culling (Verdeckungsrechnung im Objektraum)} % 7.81, 7.82, 8.83
				\todo{Content}
			% end

			\subsection{Projektion} % 7.85, 7.86
				\todo{Content}
			% end
		% end

		\section{Rasterisierung} % 7.88
			\todo{Content}

			\subsection{Scan-Konvertierung} % 7.89
				\todo{Content}

				\subsubsection{Rasterisierung von Linien (Bresenham-Algorithmus)} % 7.90, 7.91
					\todo{Content}
				% end

				\subsubsection{Rasterisierung von Polygonen (Scanline Algorithmus)} % 7.92
					\todo{Content}
				% end
			% end

			\subsection{Verdeckungsrechnung} % 7.93
				\todo{Content}

				\subsubsection{z-Buffer-Algorithmus} % 7.94, 7.95, 7.96, 7.97
					\todo{Content}
				% end
			% end
		% end

		\section{Ausgabe} % 7.99
			\todo{Content}
		% end

		\section{Beispiele} % 7.101, 7.102, 7.103
			\todo{Content}
		% end
	% end

	\chapter{Transformationen} % 8.1
		\todo{Content}

		\section{Affine Abbildungen} % 8.7, 8.8, 8.9, 8.12, 8.13
			\todo{Content}

			\subsection{Eigenschaften} % 8.14, 8.15, 8.16
				\todo{Content}
			% end

			\subsection{Translation} % 8.17, 8.18
				\todo{Content}
			% end

			\subsection{Homogene Koordinaten} % 8.19, 8.20
				\todo{Content}

				\subsubsection{Translation} % 8.21, 8.22, 8.23, 8.24, 8.25
					\todo{Content}
				% end
			% end

			\subsection{Matrizenschreibweise} % 8.26, 8.27
				\todo{Content}
			% end
		% end

		\section{Skalierung, Scherung, Rotation} % 8.28, 8.29, 8.30, 8.31
			\todo{Content}

			\subsection{Skalierung} % 8.34, 8.35, 8.36
				\todo{Content}
			% end

			\subsection{Scherung} % 8.37, 8.38
				\todo{Content}
			% end

			\subsection{Rotation} % 8.39, 8.40, 8.41, 8.42
				\todo{Content}

				\subsubsection{Rotation um beliebige Achse} % 8.43, 8.44, 8.45
					\todo{Content}
				% end

				\subsubsection{Rotation um beliebige Raumachse} % 8.46, 8.47
					\todo{Content}
				% end
			% end

			\subsection{Nicht-Kommutativität von Transformationen} % 8.48
				\todo{Content}
			% end

			\subsection{Rechenaufwand} % 8.49
				\todo{Content}
			% end
		% end

		\section{Projektion} % 8.50, 8.52, 8.53, 8.54
			\todo{Content}

			\subsection{Perspektive Projektion} % 8.55, 8.59, 8.60
				\todo{Content}

				\subsubsection{Kanonisches Sichtvolumen} % 8.61, 8.62
					\todo{Content}
				% end

				\subsubsection{Allgemeine perspektivische Transformation} % 8.63
					\todo{Content}
				% end
			% end

			\subsection{Parallele Projektion} % 8.56
				\todo{Content}
			% end

			\subsection{Ein-, Zwei- und Dreipunktperspektive} % 8.65
				\todo{Content}
			% end
		% end

		\section{3D-Interaktion} % 8.68, 8.69, 8.70
			\todo{Content}

			\subsection{Manipulatoren} % 8.71, 8.72, 8.73, 8.74
				\todo{Content}
			% end
		% end
	% end

	\chapter{3D-Visualisierung} % 9.1
		\todo{Content}

		\section{3D-Daten} % 9.4
			\todo{Content}

			\paragraph{Terrain} % 9.5
				\todo{Content}
			% end

			\paragraph{Laser Scanning} % 9.6
				\todo{Content}
			% end

			\paragraph{Range Images} % 9.7
				\todo{Content}
			% end

			\paragraph{Medizinische Bilddaten} % 9.8, 9.9
				\todo{Content}
			% end

			\paragraph{Schneiden} % 9.10
				\todo{Content}
			% end

			\paragraph{Wetter} % 9.11
				\todo{Content}
			% end

			\paragraph{Klimaforschung} % 9.12
				\todo{Content}
			% end
		% end

		\section{Triangulation von Punktwolken} % 9.13, 9.14
			\todo{Content}

			\subsection{Ideal Triangulation} % 9.15
				\todo{Content}
			% end

			\subsection{Voronoi-Diagramm und Delaunay-Triangulation} % 9.16, 9.17, 9.18
				\todo{Content}
			% end
		% end

		\section{Indirekte Volumenvisualisierung} % 9.22, 9.25
			\todo{Content}

			\subsection{3D-Volumen und Nachbarschaft} % 9.23, 9.24
				\todo{Content}
			% end

			\subsection{2D: Konturlinien} % 9.26
				\todo{Content}
			% end

			\subsection{3D: Isoflächen} % 9.27, 9.28
				\todo{Content}
			% end

			\subsection{2D: Marching Squares} % 9.29, 9.30
				\todo{Content}
			% end

			\subsection{3D: Marching Cubes} % 9.31
				\todo{Content}
			% end

			\subsection{Große Polygonmodelle und Performanz} % 9.33, 9.34
				\todo{Content}

				\subsubsection{Culling von Geometrie} % 9.35
					\todo{Content}
				% end

				\subsubsection{Meshreduktion} % 9.36, 9.37
					\todo{Content}
				% end

				\subsubsection{Mesh-Glättung} % 9.38, 9.39
					\todo{Content}
				% end
			% end
		% end

		\section{Direkte Volumenvisualisierung} % 9.40, 9.41
			\todo{Content}

			\subsection{Density Emitter Model} % 9.42
				\todo{Content}
			% end

			\subsection{Volumen-Rendering-Gleichung} % 9.43, 9.44
				\todo{Content}
			% end
		% end

		\section{Volumen-Rendering-Pipeline} % 9.45
			\todo{Content}

			\subsection{Pipeline} % N/A
				\todo{Content}

				\paragraph{Abtastung} % 9.46, 9.47, 9.48
					\todo{Content}
				% end

				\paragraph{Klassifikation und Beleuchtung} % 9.50
					\todo{Content}
				% end

				\paragraph{Komposition} % 9.51
					\todo{Content}

					\subparagraph{Back-to-Front-Komposition} % 9.52
						\todo{Content}
					% end

					\subparagraph{Front-to-Back-Komposition} % 9.53, 9.54, 9.55
						\todo{Content}
					% end
				% end
			% end

			\subsection{Transferfunktion} % 9.56, 9.57, 9.58, 9.59, 9.60
				\todo{Content}
			% end
		% end
	% end

	\chapter{Szenengraphen am Beispiel X3DOM} % 10.1, 10.2, 10.3, 10.40
		\todo{Content}

		\section{Strukturierung von 3D-Szenendaten} % 10.4, 10.5, 10.6
			\todo{Content}
		% end

		\section{Szenengraph} % 10.10, 10.11, 10.12, 10.14
			\todo{Content}
		% end

		\section{X3DOM} % 10.16, 10.17, 10.19
			\todo{Content}
		% end
	% end

	\chapter{Informationsvisualisierung} % 11.1, 11.2, 11.15, 11.16, 11.17, 11.18, 11.19, 11.20
		\todo{Content}

		\section{Informationsdesign} % 11.23, 11.24, 11.25, 11.34, 11.56, 11.57
			\todo{Content}

			\subsection{Referenzmodell von Card} % 11.26, 11.27, 11.28, 11.29, 11.30, 11.31, 11.32, 11.33
				\todo{Content}
			% end
		% end

		\section{Datentypen} % N/A
			\todo{Content}

			\subsection{1D-Daten, Zeitreihen} % 11.62, 11.63
				\todo{Content}
			% end

			\subsection{2D-Daten} % 11.73
				\todo{Content}
			% end

			\subsection{mD-Daten (multidimensional)} % 11.77, 11.78
				\todo{Content}
			% end

			\subsection{Hierarchien} % 11.91
				\todo{Content}
			% end

			\subsection{Graphen/Netzwerke} % 11.155
				\todo{Content}
			% end
		% end

		\section{Kuchendiagramm (1D)} % 11.64, 11.65
			\todo{Content}
		% end

		\section{Balkendiagramm (1D)} % 11.68
			\todo{Content}
		% end

		\section{Liniendiagramm (Zeitreihe)} % 11.67, 11.68, 11.69
			\todo{Content}

			\paragraph{Problem: Viele Zeitreihen} % 11.70, 11.71
				\todo{Content}
			% end

			\paragraph{Problem: Länge} % 11.72
				\todo{Content}
			% end
		% end

		\section{Scatterplot (2D, 3D)} % 11.74, 11.75
			\todo{Content}

			\paragraph{Problem: Overplotting} % 11.76
				\todo{Content}
			% end
		% end

		\section{Scatterplotmatrix (nD)} % 11.79, 11.80, 11.81
			\todo{Content}
		% end

		\section{Parallele Koordinaten (3D, nD)} % 11.82, 11.83, 11.84, 11.85, 11.86, 11.87, 11.88
			\todo{Content}

			\paragraph{Problem: Overplotting} % 11.89
				\todo{Content}
			% end

			\paragraph{Problem: Viele Dimensionen} % 11.90
				\todo{Content}
			% end
		% end

		\section{Node-Link-Diagramm (Hierarchien, Graphen)} % 11.92, 11.156
			\todo{Content}

			\paragraph{Problem: Layout} % 11.93, 11.157, 11.158
				\todo{Content}
			% end

			\paragraph{Problem: Viele Knoten} % 11.94, 11.159
				\todo{Content}
			% end
		% end

		\section{Treemap (Hierarchien)} % 11.95, 11.96, 11.97, 11.98, 11.99, 11.100, 11.101, 11.102
			\todo{Content}

			\paragraph{Problem: Überlappung} % 11.105
				\todo{Content}
			% end

			\paragraph{Problem: Größendarstellung} % 11.106, 11.107
				\todo{Content}
			% end
		% end

		\section{Zusammenfassung} % 11.160
			\todo{Content}
		% end
	% end

	\chapter{Farbe} % 12.1, 12.2, 12.3, 12.6, 12.6
		\todo{Content}

		\section{Dimensionalität} % 12.8, 12.9
			\todo{Content}
		% end

		\section{Wahrnehmungskorrelate} % 12.10, 12.11, 12.12
			\todo{Content}
		% end

		\section{Reproduktion} % 12.13
			\todo{Content}
		% end

		\section{Berechnung von Farbattributen} % 12.14
			\todo{Content}

			\subsection{Das Auge} % 12.15, 12.16, 12.17
				\todo{Content}
			% end

			\subsection{Spektrale Charakterisierung des Auges} % 12.18, 12.19, 12.20, 12.21, 12.22
				\todo{Content}
			% end

			\subsection{Spektralwertfunktion} % 12.23, 12.24
				\todo{Content}
			% end

			\subsection{Cone Fundamentals} % 12.25
				\todo{Content}
			% end
		% end

		\section{Objektfarben, Lichtmatrix und XIEXYZ-Farbraum} % 12.26, 12.27, 12.28, 12.29
			\todo{Content}
		% end

		\section{Metamerie} % 12.30, 12.31, 12.32, 12.33
			\todo{Content}
		% end

		\section{Gegenfarbtheorie} % 12.35
			\todo{Content}
		% end

		\section{Stevenssche Potenzfunktion} % 12.36
			\todo{Content}
		% end

		\section{CIELAB Farbraum} % 12.37, 12.38
			\todo{Content}
		% end

		\section{Technische Farbräume} % 12.39
			\todo{Content}

			\subsection{Geräte RGB} % 12.40
				\todo{Content}
			% end

			\subsection{Geräteunabhängige RGB} % 12.40
				\todo{Content}
			% end

			\subsection{YCbCr} % 12.41
				\todo{Content}
			% end

			\subsection{HSI/HSV/HSL} % 12.41
				\todo{Content}
			% end

			\subsection{CMY/CMYK} % 12.42
				\todo{Content}
			% end
		% end

		\section{Komplexität von Farbe} % 12.43
			\todo{Content}

			\subsection{Chromatische Adaptation} % 12.45, 12.46
				\todo{Content}

				\subsubsection{Modellbildung} % 12.47, 12.48
					\todo{Content}
				% end
			% end

			\subsection{Farbwahrnehmungsphänomene} % 12.49
				\todo{Content}

				\subsubsection{Simultankontrast} % 12.49
					\todo{Content}
				% end

				\subsubsection{Crispening Effekt} % 12.50
					\todo{Content}
				% end

				\subsubsection{Stevens Effekt} % 12.51
					\todo{Content}
				% end

				\subsubsection{Hunt Effekt} % 12.52
					\todo{Content}
				% end
			% end

			\subsection{Farbwahrnehmungsmodelle} % 12.54
				\todo{Content}

				\subsubsection{CIECAM02} % 12.55, 12.56, 12.57
					\todo{Content}
				% end
			% end

			\subsection{Kontrastsensitivität} % 12.58, 12.59, 12.60
				\todo{Content}

				\subsubsection{S-CIELAB (Spacial-CIELAB)} % 12.61, 12.62, 12.63
					\todo{Content}
				% end

				\subsubsection{iCAM (Image Color Appearance Model)} % 12.64, 12.65, 12.66
					\todo{Content}
				% end
			% end
		% end
	% end

	\chapter{User Interfaces} % 13a.1, 13a.64
		\todo{Content}

		\section{Interaktion} % 13a.7
			\todo{Content}

			\subsection{Möglichkeiten} % 13a.8
				\todo{Content}

				\subsubsection{Kommandozeile} % 13a.9
					\todo{Content}
				% end

				\subsubsection{Menüs} % 13a.10, 13a.11
					\todo{Content}
				% end

				\subsubsection{Formulare} % 13a.12
					\todo{Content}
				% end

				\subsubsection{Fragen und Antworten} % 13a.13
					\todo{Content}
				% end

				\subsubsection{Direkte Manipulation} % 13a.14, 13a.15
					\todo{Content}
				% end

				\subsubsection{3D-Umgebungen} % 13a.17
					\todo{Content}
				% end

				\subsubsection{Natürliche Sprache} % 13a.18
					\todo{Content}
				% end

				\subsubsection{Gesten} % 13a.19
					\todo{Content}
				% end
			% end

			\subsection{Designprozess} % 13a.20, 13a.21, 13a.22
				\todo{Content}

				\subsubsection{Wasserfallmodell} % 13a.24
					\todo{Content}
				% end

				\subsubsection{Spiralmodell} % 13a.25
					\todo{Content}
				% end

				\subsubsection{V-Modell} % 13a.26
					\todo{Content}
				% end

				\subsubsection{Dynamic Systems Development Method (DSDM)} % 13a.27
					\todo{Content}
				% end

				\subsubsection{Design Process Model} % 13a.28
					\todo{Content}
				% end
			% end
		% end

		\section{GUI: Benutzeroberflächen} % 13a.30
			\todo{Content}

			\subsection{Das WIMP-Interface} % 13a.31
				\todo{Content}

				\subsubsection{Fenster-Komponenten} % 13a.32
					\todo{Content}

					\paragraph{Multiple Document Interface (MDI)} % 13a.33
						\todo{Content}
					% end

					\paragraph{Single Document Interface (SDI)} % 13a.34
						\todo{Content}
					% end

					\paragraph{Tabbed Document Interface} % 13a.35
						\todo{Content}
					% end
				% end

				\subsubsection{Dialogboxen} % 13a.36
					\todo{Content}

					\paragraph{Checkboxen} % 13a.37
						\todo{Content}
					% end

					\paragraph{Radio Buttons} % 13a.38
						\todo{Content}
					% end

					\paragraph{Listboxen} % 13a.39
						\todo{Content}
					% end

					\paragraph{Comboboxen} % 13a.40
						\todo{Content}
					% end

					\paragraph{Spinner} % 13a.41
						\todo{Content}
					% end

					\paragraph{Slider} % 13a.42
						\todo{Content}
					% end

					\paragraph{Weiteres} % 13a.43
						\todo{Content}
					% end
				% end
			% end

			\subsection{Menübasierte Programme} % 13a.44, 13a.45, 13a.46
				\todo{Content}

				\subsubsection{Untermenüs} % 13a.47
					\todo{Content}
				% end

				\subsubsection{Auswahl (if-then-else-Struktur)} % 13a.48, 13a.49
					\todo{Content}

					\paragraph{Verschachtelte Entscheidungsstrukturen} % 13a.50
						\todo{Content}
					% end
				% end

				\subsubsection{Die case-Struktur} % 13a.51
					\todo{Content}
				% end

				\subsubsection{Modularisierung} % 13a.52
					\todo{Content}
				% end
			% end

			\subsection{GUI-Anwendungen und Event-basiertes Programmieren} % 13a.53, 13a.54
				\todo{Content}

				\subsubsection{Graphical User Interfaces (GUIs)} % 13a.55
					\todo{Content}
				% end

				\subsubsection{Event-Handler} % 13a.56, 13a.57
					\todo{Content}
				% end

				\subsubsection{Nutzerinteraktionen} % 13a.58
					\todo{Content}
				% end

				\subsubsection{Das Delegationsmodell} % 13a.59
					\todo{Content}
				% end
			% end
		% end

		\section{3D-Interaktion} % 13a.61, 13a.62
			\todo{Content}
		% end
	% end

	\chapter{Multimedia Information Retrieval} % 13b.1, 13b.7, 13b.8, 13b.64
		\todo{Content}

		\section{Inhaltsbasierte Suche} % 13b.10, 13b.11, 13b.14, 13b.17
			\todo{Content}

			\subsection{Mathematische Beschreibung} % 13b.15
				\todo{Content}
			% end

			\subsection{Retrieval Ergebnis} % 13b.16
				\todo{Content}
			% end
		% end

		\section{Distanzmaße} % 13b.19, 13b.20, 13b.21, 13b.22, 13b.23, 13b.24
			\todo{Content}
		% end

		\section{Query-Modalitäten} % 13b.25, 13b.30, 13b.31
			\todo{Content}

			\subsection{Text} % 13b.32, 13b.33, 13b.34, 13b.35, 13b.36, 13b.37
				\todo{Content}
			% end

			\subsection{Example} % 13b.38, 13b.39, 13b.40, 13b.41, 13b.42
				\todo{Content}
			% end

			\subsection{Example-Bilder} % 13b.43
				\todo{Content}
			% end

			\subsection{Sketch} % 13b.44, 13b.45, 13b.46, 13b.47, 13b.48, 13b.49, 13b.50, 13b.51, 13b.52, 13b.53
				\todo{Content}
			% end
		% end

		\section{Explorative Suche} % 13b.54, 13b.55
			\todo{Content}

			\subsection{Research Data} % 13b.56, 13b.57, 13b.58, 13b.59, 13b.60
				\todo{Content}
			% end
		% end
	% end
\end{document}
