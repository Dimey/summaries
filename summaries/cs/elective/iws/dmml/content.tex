\chapter{Einleitung} % 1.1, 1.2, 1.3, 1.4
	\todo{Content}

	\section{Was ist Künstliche Intelligenz?} % 1.5, 1.6, 1.7, 1.8, 1.9, 1.10, 1.11, 1.12, 1.13, 1.14, 1.15, 1.16
		\todo{Content}
	% end

	\section{Was ist ein Algorithmus} % 1.17, 1.18, 1.19
		\todo{Content}
	% end

	\section{Maschinelles Lernen} % 1.20, 1.21, 1.22
		\todo{Content}
	% end

	\section{Geschichte} % 1.23, 1.24, 1.25, 1.26, 1.46, 1.47, 1.48, 1.49, 1.50, 1.51, 1.52, 1.53
		\todo{Content}

		\subsection{Das Perzeptron} % 1.27, 1.44, 1.45
			\todo{Content}

			\subsubsection{Aufbau} % 1.28, 1.29, 1.30, 1.31, 1.32, 1.33, 1.34, 1.35, 1.36
				\todo{Content}
			% end

			\subsubsection{Lernalgorithmus} % 1.37, 1.38, 1.39, 1.40, 1.41, 1.42
				\todo{Content}
			% end

			\subsubsection{Mehrschichtige Netzwerke} % 1.43
				\todo{Content}
			% end
		% end
	% end

	\section{KI Heute} % 1.54, 1.55, 1.56, 1.57, 1.58, 1.59, 1.60, 1.61, 1.62, 1.63, 1.64, 1.65, 1.79
		\todo{Content}
	% end

	\section{Was ist Maschinelles Lernen?} % 1.80, 1.81, 1.82, 1.83, 1.84, 1.85, 1.92
		\todo{Content}

		\subsection{Kurzgefasst} % 1.86, 1.87, 1.88, 1.89
			\todo{Content}
		% end

		\subsection{Arten des Maschinellen Lernens} % 1.90, 1.91
			\todo{Content}
		% end
	% end

	\section{Ausblick} % 1.93, 1.94, 1.95
		\todo{Content}
	% end
% end

\chapter{Grundlagen} % N/A
	\todo{Content}

	\section{CRISP: Verlaufsmodell der Wissensentdeckung} % 2.25, 2.26
		\todo{Content}
	% end

	\section{Klassifikation und Regression} % 2.31, 2.32, 2.33, 2.34
		\todo{Content}
	% end

	\section{Statistik} % N/A
		\todo{Content}

		\subsection{Erwartungswert, 2.24, 3.12, 3.13} % 2.49, 2.50, 2.51, 2.52
			\todo{Content}
		% end

		\subsection{Bias} % 2.53, 2.54
			\todo{Content}
		% end

		\subsection{Normalverteilung} % 3.6
			\todo{Content}
		% end

		\subsection{Bayes-Statistik} % 3.14
			\todo{Content}
		% end

		\subsection{Bedingte Wahrscheinlichkeiten} % 4.11
			\todo{Content}
		% end

		\subsection{Konfidenzintervalle} % 4.26
			\todo{Content}
		% end
	% end
% end

\chapter{k-Nächste Nachbarn (kNN)} % 2.2, 2.3, 2.7, 2.8, 2.9
	\todo{Content}

	\section{Globale und Lokale Modelle} % 2.2
		\todo{Content}
	% end

	\section{Beispiel} % 2.4, 2.5
		\todo{Content}
	% end

	\section{Ähnlichkeitsmaße} % 2.6, 2.10, 2.11
		\todo{Content}
	% end

	\section{Auswahlfunktion} % 2.12
		\todo{Content}
	% end

	\section{Überanpassung} % 2.13, 2.14
		\todo{Content}
	% end

	\section{Asymptotische Ergebnisse und Fluch der hohen Dimension} % 2.15, 2.16, 2.17
		\todo{Content}
	% end
% end

\chapter{Lineare Modelle und Funktionsapproximation} % 2.63, 3.2, 3.3, 3.4, 3.24
	\todo{Content}

	\section{Lineare Modelle} % 2.28, 2.30, 2.37, 2.38, 2.39, 2.47
		\todo{Content}

		\subsection{Modell-Anpassung} % 2.42, 2.43, 2.44, 2.45, 2.46
			\todo{Content}
		% end
	% end

	\section{Fehler} % 2.48, 2.55, 2.56
		\todo{Content}

		\subsection{Bias und Varianz} % 2.57, 2.58, 2.59, 2.60
			\todo{Content}
		% end
	% end

	\section{Gütekriterien} % N/A
		\todo{Content}

		\subsection{Maximum Likelihood} % 3.5
			\todo{Content}

			\subsubsection{Maximum Likelihood für Lineare Modelle} % 3.7, 3.8
				\todo{Content}
			% end
		% end

		\subsection{Kreuzentropie} % 3.9
			\todo{Content}
		% end
	% end

	\section{Fluch der hohen Dimension} % 2.61
		\todo{Content}
	% end

	\section{Logistische Regression} % 3.36, 3.37
		\todo{Content}
	% end
% end

\chapter{Modellselektion und Evaluierung} % 2.18, 2.19, 2.20, 3.1, 3.2, 3.10, 3.11
	\todo{Content}

	\section{Aufteilung in Test- und Trainingsmenge} % 2.21, 2.22, 2.23
		\todo{Content}

		\subsection{Kreuzvalidierung} % 2.24, 3.12, 3.13
			\todo{Content}
		% end
	% end

	\section{Bayes'sche Modellselektion} % 3.15
		\todo{Content}

		\subsection{Approximation der A-Posteriori Wahrscheinlichkeit} % 3.16
			\todo{Content}
		% end

		\subsection{Bayes Information Criterion (BIC)} % 3.17, 3.18
			\todo{Content}
		% end

		\subsection{Minimum Description Length (MDL)} % 3.19, 3.20, 3.21, 3.22
			\todo{Content}
		% end

		\subsection{Zusammenhang zwischen BIC und MDL} % 3.23
			\todo{Content}
		% end
	% end

	\section{Evaluierungsmaße} % 3.25, 3.26, 3.27, 3.28
		\todo{Content}

		\subsection{ROC-Analyse und -Kurve} % 3.38, 3.39, 3.40, 3.41, 3.42
			\todo{Content}
		% end

		\subsection{Präzision und Recall} % 3.43, 3.44
			\todo{Content}
		% end

		\subsection{F-Measure, Breakeven Point} % 3.45
			\todo{Content}
		% end
	% end

	\section{Zusammenfassung von Experimentellen Ergebnissen} % 3.29, 3.30
		\todo{Content}

		\subsection{Vorzeichen-Test} % 3.31, 3.32, 3.33, 3.34
			\todo{Content}
		% end
	% end
% end

\chapter{Baumbasierte Verfahren} % 4.1, 4.2, 4.3, 4.4, 4.5, 4.6, 4.7, 4.45
	\todo{Content}

	\section{Information und Informationsgewinn} % 4.12
		\todo{Content}

		\subsection{Numerische Werte} % 4.15
			\todo{Content}
		% end
	% end

	\section{Top-Down Induction of Decision Trees (TDIDT): ID3} % 4.21, 4.22
		\todo{Content}
	% end

	\section{Stutzen (Pruning) des Baumes} % 4.23, 4.24, 4.25
		\todo{Content}

		\subsection{Fehlerschätzung} % 4.27, 4.28, 4.29, 4.30, 4.31
			\todo{Content}
		% end

		\subsection{Anwendung zum Stutzen} % 4.32
			\todo{Content}
		% end
	% end

	\section{Andere Gütemaße} % N/A
		\todo{Content}

		\subsection{Gini-Index} % 4.33, 4.34
			\todo{Content}
		% end

		\subsection{Regression} % 4.35
			\todo{Content}
		% end
	% end

	\section{Evaluierung} % 4.36, 4.37, 4.38
		\todo{Content}

		\subsection{Fehlergewichtung} % 4.39
			\todo{Content}
		% end
	% end
% end

\chapter{Ensemble-Methoden} % 5.1, 5.2, 5.3, 5.4, 5.52
	\todo{Content}

	\section{Zufallswälder (Random Forests)} % 5.5, 5.6, 5.7, 5.8, 5.9, 5.10, 5.11, 5.12, 5.13, 5.14
		\todo{Content}
	% end

	\section{Bagging} % 5.22, 5.23
		\todo{Content}
	% end

	\section{Boosting} % 5.25, 5.26, 5.27, 5.28, 5.29, 5.35, 5.38, 5.39
		\todo{Content}

		\subsection{AdaBoost} % 5.30, 5.31, 5.32, 5.33, 5.34
			\todo{Content}
		% end
	% end

	\section{Vergleich} % 5.40
		\todo{Content}
	% end

	\section{Gradient Boosting} % 5.41, 5.42, 5.43, 5.44, 5.45, 5.46
		\todo{Content}
	% end
% end

\chapter{Probabilistische Graphische Modelle und Stützvektormethode} % 6.1, 6.23, 6.34
	\todo{Content}

	\section{Naive Bayes Klassifikator} % 6.2, 6.13, 6.15, 6.16, 6.17, 6.18, 6.19, 6.20, 6.21, 6.22
		\todo{Content}

		\subsection{Beispiel} % 6.3, 6.4, 6.5, 6.6, 6.7, 6.8, 6.9, 6.10, 6.11, 6.14
			\todo{Content}
		% end
	% end

	\section{Bayes'sche Netzwerke} % 6.24, 6.25, 6.26, 6.27, 6.28
		\todo{Content}
	% end

	\section{Parameterschätzung bei Vollständigen Daten: Maximum Likelihood} % 6.29, 6.30
		\todo{Content}
	% end

	\section{Parameterschätzung bei Unvollständigen Daten: Expectation Maximization (EM)} % 6.31, 6.32, 6.33
		\todo{Content}
	% end

	\section{Diskriminative Ansätze} % 6.35, 6.36, 6.37, 6.40
		\todo{Content}

		\subsection{Stützvektormethode} % 6.42, 6.43, 6.44, 6.45, 6.51, 6.52, 6.62
			\todo{Content}

			\subsubsection{Optimalität der Hyperebene} % 6.53, 6.54, 6.55, 6.56
				\todo{Content}
			% end

			\subsubsection{Optimierungsproblem} % 6.57, 6.58, 6.59, 6.60, 6.61, 6.63
				\todo{Content}
			% end
		% end

		\subsection{Nicht linear trennbare Daten} % 6.64, 6.65, 6.69
			\todo{Content}

			\subsubsection{Transformation in ein lineares Problem} % 6.66, 6.67
				\todo{Content}
			% end

			\subsubsection{Kernel-Trick} % 6.68
				\todo{Content}
			% end
		% end
	% end
% end

\chapter{Clustering} % 6.70, 6.71, 6.72, 6.81, 6.94
	\todo{Content}

	\section{Dendrogramme und Hierarchisches Clustering (Agglomerativ/Aufteilend)} % 6.73, 6.74, 6.75, 6.76, 6.77
		\todo{Content}
	% end

	\section{(Vorgetäuschte) Strukturen, Anzahl Cluster und Ausreißer} % 6.78, 6.79, 6.80
		\todo{Content}
	% end

	\section{Partitionierung und K-Means} % 6.83, 6.84, 6.89, 6.90, 6.91, 6.92, 6.93
		\todo{Content}
	% end
% end

\chapter{Deep Learning und Faltende Neuronale Netzwerke} % 7.1, 7.5, 7.6, 7.7, 7.67, 7.68, 7.69, 8.41
	\todo{Content}

	\section{Modellierung eines Neurons} % 7.8, 7.12, 7.53
		\todo{Content}
	% end

	\section{Aufeinanderschichten von Einheiten} % 7.14
		\todo{Content}
	% end

	\section{Faltungsschichten} % 7.19, 7.20, 7.21, 7.22, 7.23, 7.24, 7.25, 7.26, 7.27, 7.49, 7.50
		\todo{Content}

		\subsection{Räumliche Auflösung, Stride und Zero-Padding} % 7.29, 7.30, 7.31, 7.32, 7.33, 7.34, 7.35, 7.36, 7.37, 7.38, 7.39, 7.40, 7.41, 7.42, 7.43, 7.44, 7.45, 7.46, 7.47, 7.48
			\todo{Content}
		% end
	% end

	\section{Räumliche Zusammenfassung} % 7.55, 7.56, 7.57
		\todo{Content}
	% end

	\section{Vollständig verbundene Schichten} % 7.59, 7.60, 7.61
		\todo{Content}
	% end

	\section{Training} % 8.7
		\todo{Content}

		\subsection{Stochastic Gradient Descent} % 7.12, 7.13, 7.15, 7.16, 7.17, 7.63, 8.9, 8.10, 8.21, 8.22
			\todo{Content}
		% end

		\subsection{Backpropagation} % 7.18, 8.11
			\todo{Content}

			\subsubsection{Sequential Brick} % 8.12
				\todo{Content}
			% end

			\subsubsection{Loss Bricks} % 8.13, 8.14
				\todo{Content}
			% end

			\subsubsection{Linear Brick} % 8.15
				\todo{Content}
			% end

			\subsubsection{Activation Function Bricks} % 8.16, 8.17
				\todo{Content}

				\paragraph{Subgradienten} % 8.18
					\todo{Content}
				% end

				\paragraph{Leaky ReLU} % 8.19
					\todo{Content}
				% end
			% end
		% end
	% end

	\section{Vermeidung von Überanpassung} % 8.23
		\todo{Content}

		\subsection{Dropout} % 8.24, 8.25, 8.26
			\todo{Content}
		% end

		\subsection{Datenaugmentierung} % 8.27
			\todo{Content}
		% end
	% end

	\section{Fine-Tuning und Transfer Learning} % 8.28, 8.29, 8.30
		\todo{Content}
	% end

	\section{Visualisierung} % 8.31, 8.32, 8.33, 8.34, 8.35, 8.36, 8.37, 8.38
		\todo{Content}

		\subsection{Täuschen von CNNs} % 8.39, 8.40
			\todo{Content}
		% end
	% end

	\section{Beispiele: LeNet-5, AlexNet, GoogLeNet} % 7.64, 7.65, 7.66, 8.2, 8.3, 8.4, 8.5, 8.6
		\todo{Content}
	% end
% end

\chapter{Data Mining: Apriori und PageRank} % 9.1, 9.2, 9.3, 9.18
	\todo{Content}

	\section{Assoziationsregeln} % 9.4, 9.5, 9.8, 9.9, 9.10
		\todo{Content}

		\subsection{Binäre Datenbanken} % 9.6, 9.7
			\todo{Content}
		% end
	% end

	\section{Apriori Algorithmus} % 9.11, 9.14, 9.15, 9.16, 9.17, 9.19
		\todo{Content}
	% end

	\section{Regelbewertung} % 9.20, 9.21, 9.22, 9.23
		\todo{Content}
	% end

	\section{Web Mining} % 9.24, 9.25
		\todo{Content}

		\subsection{Ranking von Webseiten} % 9.26
			\todo{Content}
		% end

		\subsection{PageRank} % 9.27, 9.28, 9.29, 9.30
			\todo{Content}
		% end
	% end
% end
