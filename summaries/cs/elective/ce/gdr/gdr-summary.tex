\documentclass[a4paper, 11pt, accentcolor = tud3b]{tudreport}

% Core packages.
\usepackage[T1]{fontenc}
\usepackage[utf8]{inputenc}
\usepackage[ngerman]{babel}
% Other packages.
\usepackage{csquotes}
\usepackage{enumitem}
\usepackage[mathcal]{euscript} % Get readable mathcal font.
\usepackage{float}
\usepackage{hyperref}
\usepackage{mathtools}
\usepackage{stmaryrd}
\usepackage{tabto}
\usepackage{tikz}
\usepackage[disable]{todonotes}
\usetikzlibrary{arrows.meta, shapes, backgrounds, angles, calc, decorations.markings}

% Basic information.
\title{Grundlagen der Robotik}
\subtitle{Zusammenfassung \\ Fabian Damken}
\author{Fabian Damken}
\date{\today}

% Description-list styling.
\SetLabelAlign{parright}{\parbox[t]{\labelwidth}{\raggedleft#1}}
\setlist[description]{style = multiline, leftmargin = 4cm, align = parright}

\tikzset{> = { Latex[length = 2.5mm] }}
\tikzstyle{every path} = [ very thick ]

% New commands.

% https://tex.stackexchange.com/a/333383
\makeatletter
\renewcommand*\env@matrix[1][*\c@MaxMatrixCols c]{%
	\hskip -\arraycolsep
	\let\@ifnextchar\new@ifnextchar
	\array{#1}}
\makeatother

\begin{document}
	\maketitle
	\tableofcontents
	\listoftodos

	\chapter{Einleitung} % 1.8
		\todo{Content}

		\section{Was ist ein Roboter?} % 1.9, 1.10, 1.11, 1.12, 1.13, 1.14
			\todo{Content}
		% end

		\section{Was ist KI?} % 1.15, 1.16, 1.17, 1.18, 1.19
			\todo{Content}
		% end

		\section{Was ist Robotik?} % 1.20, 1.21, 1.22
			\todo{Content}
		% end

		\section{Sense -- Plan -- Act} % 1.27
			\todo{Content}

			\paragraph{Act} % N/A
				\todo{Content}

				\subparagraph{Kinematik} % 1.28, 1.29, 1.30, 1.31, 2.32, 1.33, 1.34, 1.35
					\todo{Content}
				% end

				\subparagraph{Dynamik} % 1.36, 1.37
					\todo{Content}
				% end

				\subparagraph{Steuerung} % 1.38, 1.39, 1.40, 1.41
					\todo{Content}
				% end
			% end

			\paragraph{Sense} % N/A
				\todo{Content}

				\subparagraph{Sensoren} % 1.42, 1.43, 1.44, 1.45
					\todo{Content}
				% end
			% end

			\paragraph{Plan} % N/A
				\todo{Content}

				\subparagraph{Lokalisierung, Kartographie, Navigation, Bahnplanung} % 1.46, 1.47, 1.48, 1.49
					\todo{Content}
				% end
			% end
		% end

		\section{Geschichte der Robotik} % 1.50
			\todo{Content}

			\subsection{Historische Entwicklung} % 1.51, 1.54, 1.55
				\todo{Content}
			% end

			\subsection{Die drei Gebote der Robotik} % 1.52, 1.53
				\todo{Content}
			% end

			\subsection{Autonome Fahrzeuge} % 1.59
				\todo{Content}
			% end

			\subsection{Wntwicklungstrend} % 1.60
				\todo{Content}
			% end
		% end

		\section{Herausforderungen} % 1.61
			\todo{Content}

			\subsection{Humanoide Bewegung} % 1.61, 1.61, 1.63, 1.64, 1.65, 1.66, 1.67, 1.68, 1.69
				\todo{Content}
			% end

			\subsection{Roboter für menschliche Mobilität} % 1.112, 1.113, 1.114, 1.115, 1.116, 1.117, 1.118, 1.119, 1.120, 1.121, 1.122, 1.123
				\todo{Content}
			% end

			\subsection{Roboter-Avatare} % 1.70
				\todo{Content}

				\subsubsection{Beine} % 1.74, 1.75, 1.76, 1.77, 1.78, 1.79, 1.80, 1.81, 1.82, 1.83, 1.84
					\todo{Content}
				% end

				\subsubsection{Katastrophenbewältigung und -hilfe} % 1.71, 1.72, 1.73, 1.87, 1.88, 1.89, 1.90, 1.91, 1.92, 1.93, 1.94
					\todo{Content}
				% end

				\subsubsection{Objekt-Vorlagen} % 1.95, 1.96, 1.97, 1.98
					\todo{Content}
				% end

				\subsubsection{Geifen und Manipulation} % 1.99, 1.100, 1.101, 1.102, 1.103, 1.104, 1.105, 1.106, 1.107, 1.108, 1.109
					\todo{Content}
				% end
			% end

			\subsection{Die Robotik an sich} % 1.124, 1.125, 1.126, 1.127, 1.128
				\todo{Content}
			% end
		% end
	% end

	\chapter{Räumliche Darstellungen und Transformationen} % S.11
		\todo{Content}

		\section{Mathematische Grundlagen und Notation} % S.11, 2.1
			\todo{Content}

			\subsection{Vektoren} % S.11, 2.2, 2.3
				\todo{Content}
			% end

			\subsection{Matrizen} % S.12, S.13, 2.4, 2.5, 2.6, 2.7, 2.8, 2.9, 2.10, 2.11
				\todo{Content}
			% end
		% end

		\section{Koordinatensysteme} % S.14, 2.12, 2.13
			\todo{Content}

			\subsection{Position} % S.15, 2.14
				\todo{Content}
			% end

			\subsection{Orientierung} % S.16, 2.15, 2.16
				\todo{Content}
			% end
		% end

		\section{Klassische Transformationsbeziehungen} % S.17, 2.17, 2.18
			\todo{Content}
		% end

		\section{Rotation eines Koordinatensystems} % S.18, 2.20
			\todo{Content}

			\subsection{Rotationsmatrizen} % S.18, S.19, 2.21, 2.22, 2.23
				\todo{Content}
			% end

			\subsection{Verkettete Rotationen} % S.19, S.20, S.21, 2.24, 2.25, 2.26, 2.27, 2.28
				\todo{Content}
			% end

			\subsection{Winkelparameter} % S.22, 2.29, 2.35, 2.36
				\todo{Content}

				\subsubsection{RPY-Winkel} % S.22, S.22, 2.30, 2.31, 2.32, 2.33
					\todo{Content}
				% end

				\subsubsection{Euler-Winkel} % S.23, S.24, 2.34
					\todo{Content}
				% end

				\subsubsection{Kardan-Winkel} % S.23, S.24, 2.35
					\todo{Content}
				% end
			% end
		% end

		\section{Homogene Transformationen} % S.24, S.25, S.26, 2.19, 2.37, 2.38, 2.39, 2.40, 2.41, 2.42, 2.43
			\todo{Content}
		% end

		\section{Berechnungseffizienz} % S.26, S.27
			\todo{Content}
		% end
	% end

	\chapter{Roboterkinematik} % S.28
		\todo{Content}

		\section{Vorwärtskinematik} % S.28
			\todo{Content}

			\subsection{Kinematische Ketten} % S.28, S.29
				\todo{Content}
			% end

			\subsection{Kinematische Modellbildung} % S.30, S.31
				\todo{Content}
			% end

			\subsection{Denavit-Hartenberg (DH) Konventionen} % S.31, S.32, S.33, S.34, S.35
				\todo{Content}

				\subsubsection{Beispiel: SCARA-Manipulator} % S.36, S.37, S.38
					\todo{Content}
				% end

				\subsubsection{Beispiel: 2-DOF-Schub-Drehgelenk-Arm} % S.38, S.39
					\todo{Content}
				% end
			% end
		% end

		\section{Rückwärtskinematik (Inverse Kinematik)} % S.40, S.41, S.42
			\todo{Content}

			\subsection{Numerische Berechnung} % S.43, S.44
				\todo{Content}
			% end

			\subsection{Analytische Lösung} % S.45, S.46
				\todo{Content}

				\subsubsection{Beispiel: Ebener SCARA-Manipulator} % S.46, S.47
					\todo{Content}
				% end
			% end

			\subsection{Geometrische Ermittlung der analytischen Lösung} % S.48
				\todo{Content}
			% end

			\subsection{Algorithmische Ermittlung der analytischen Lösung} % S.49
				\todo{Content}
			% end
		% end

		\section{Genauigkeit des kinematischen Modells} % S.50
			\todo{Content}
		% end

		\section{Modellierung von Roboterbeinen und -armen} % S.51
			\todo{Content}

			\subsection{Dreigelenkiges Bein eines vierbeinigen Roboters} % S.51
				\todo{Content}
			% end

			\subsection{Sechgelenkiges Bein eines humanoiden Roboters} % S.52, S.53
				\todo{Content}
			% end

			\subsection{Arm eines humanoiden Roboters} % S.54
				\todo{Content}
			% end
		% end
	% end

	\chapter{Geschwindigkeit, Jacobi-Matrix und statische Kräfte} % S.55, S.56, S.57, S.58, S.59
		\todo{Content}

		\section{Schiefsymmetrische Matrizen, Vektoren der Winkelgeschwindigkeiten und -beschleunigungen} % S.60, S.61, S.62
			\todo{Content}

			\subsection{Relative Beschleunigung zwischen \(S_n\) und \(S_0\)} % S.63
				\todo{Content}
			% end
		% end

		\section{Jacobi-Matrix eines Manipulators} % S.64
			\todo{Content}

			\subsection{Addition von Winkelgeschwindigkeiten} % S.64, S.65
				\todo{Content}
			% end

			\subsection{Herleitung} % S.66
				\todo{Content}

				\subsubsection{Drehwinkelgeschwindigkeit} % S.67, S.68
					\todo{Content}
				% end

				\subsubsection{Lineare Geschwindigkeit} % S.69, S.70, S.71
					\todo{Content}
				% end

				\subsubsection{Zusammenfassung} % S.72
					\todo{Content}
				% end
			% end

			\subsection{Beispiel: Ebener SCARA-Manipulator} % S.73, S.74
				\todo{Content}
			% end
		% end

		\section{Inverses Jacobi-Modell} % S.75
			\todo{Content}

			\subsection{Geschwindigkeitssteuerung} % S.75, S.76, S.77
				\todo{Content}

				\subsubsection{Beispiel: SCARA-Manipulator} % S.77, S.78, S.79
					\todo{Content}
				% end
			% end
		% end

		\section{Kinematische Singularitäten} % S.79, S.80, S.83
			\todo{Content}

			\subsection{Beispiel: SCARA-Manipulator} % S.80, S.81
				\todo{Content}
			% end

			\subsection{Beispiel: Typischer Industrieroboter mit 6 Drehgelenken} % S.81, S.82
				\todo{Content}
			% end

			\subsection{Weitere Beispiele} % S.82
				\todo{Content}
			% end

			\subsection{Vermeidung} % S.82, S.83
				\todo{Content}
			% end

			\subsection{Umgang mit unvermeidbaren Singularitäten} % S.83
				\todo{Content}
			% end
		% end

		\section{Nicht-holonome Kinematik mehrrädriger Fahrzeuge} % S.84, S.85
			\todo{Content}

			\subsection{Differentialantrieb} % S.86, S.87, 4.2
				\todo{Content}
			% end

			\subsection{Allgemeines Vorwärtskinematikproblem für Fahrzeuge} % S.87, S.88, S.89, S.90
				\todo{Content}
			% end

			\subsection{Inverses Kinematikproblem} % S.90, S.91
				\todo{Content}
			% end

			\subsection{Omnidirektionale Dreirad-Kinematik} % S.91, S.92
				\todo{Content}
			% end

			\subsection{Weitere Antriebsarten von Fahrzeugen} % S.92, 4.5
				\todo{Content}
			% end
		% end

		\section{Statische Kräfte bei Manipulatoren} % S.92
			\todo{Content}
		% end
	% end

	\chapter{Roboterdynamik} % S.93, S.94, S.95
		\todo{Content}

		\section{Massenverteilung eines Starrkörpers} % S.96, S.97, S.98
			\todo{Content}
		% end

		\section{Newton-Euler Formulierung der Roboterdynamik} % S.99, S.100, S.110, S.118, S.119
			\todo{Content}

			\subsection{Iterative Berechnung von INV DYN} % S.101, S.102, S.103, S.104, S.105, S.106, S.107, S.108, S.109
				\todo{Content}
			% end

			\subsection{Zusammenfassung} % S.109, S.110
				\todo{Content}
			% end

			\subsection{Beispiel: SCARA-Manipulator} % S.111, S.112, S.113, S.114, S.115, S.116, S.117
				\todo{Content}
			% end
		% end

		\section{Lagrangesche Formulierung der Roboterdynamik} % S.120
			\todo{Content}

			\subsection{Kinetische und potentielle Energie, Lagrangefunktion} % S.120, S.122
				\todo{Content}

				\subsubsection{Kinetische Energie} % S.120
					\todo{Content}
				% end

				\subsubsection{Potentielle Energie} % S.121
					\todo{Content}
				% end

				\subsubsection{Lagrangefunktion} % S.121, S.122
					\todo{Content}
				% end
			% end

			\subsection{Beispiel: SCARA-Manipulator} % S.123, S.124, S.125, S.126, S.127, S.128, S.129
				\todo{Content}
			% end
		% end

		\section{Numerische Aspekte} % S.130
			\todo{Content}

			\subsection{Modularität} % S.130
				\todo{Content}
			% end

			\subsection{Simulation} % S.131
				\todo{Content}
			% end
		% end

		\section{Rekursive Verfahren zur Berechnung der Vorwärtsdynamik} % S.131
			\todo{Content}

			\subsection{Verfahren mit expliziter Berechnung der Massenmatrix} % S.132
				\todo{Content}

				\subsubsection{Verfahren 1: Berechnung von \(M\) durch wiederholte Auswertung des Newton-Euler-Verfahrens} % S.132, 5.20
					\todo{Content}
				% end

				\subsubsection{Verfahren 2: Ausnutzen der Symmetrie von \(M\)} % S.132, S.133, 5.20
					\todo{Content}
				% end

				\subsubsection{Verfahren 3: Aggregation von Teilmanipulatoren (CRBA)} % S.133, S.134, 5.20
					\todo{Content}
				% end

				\subsubsection{Vergleich der Verfahren} % S.135
					\todo{Content}
				% end
			% end

			\subsection{Verfahren ohne explizite Berechnung der Massenmatrix} % S.135, 5.22, 5.23, 5.24
				\todo{Content}
			% end

			\subsection{Multibody Systems Library MBSlib} % 5.25, 5.26, 5.27, 5.28, 5.29, 5.31, 5.32
				\todo{Content}

				\subsubsection{Beispiele} % 5.29, 5.30, 5.33, 5.34, 5.35, 5.36, 5.37
					\todo{Content}
				% end
			% end

		\section{Geschlossene kinematische Ketten} % S.136
			\todo{Content}
		% end

		\section{Berücksichtigung von Nichtstarrkörpereffekten} % S.137, 5.39
			\todo{Content}

			\subsection{Reibung} % S.137, S.138, 5.40, 5.41
				\todo{Content}

				\subsubsection{Reibungsmodell} % 5.42
					\todo{Content}

					\paragraph{Haftung} % 5.43
						\todo{Content}
					% end

					\paragraph{Stribeck-Reibung} % 5.44
						\todo{Content}
					% end

					\paragraph{Coloumbsche Reibung} % S.138
						\todo{Content}
					% end

					\paragraph{Viskose Gleitreibung} % S.138, S.139
						\todo{Content}
					% end
				% end
			% end

			\subsection{Elastizität} % S.139, 5.45, 5.46
				\todo{Content}

				\subsubsection{Grundlagen} % S.139, S.140, 5.47, 5.48, 5.49
					\todo{Content}
				% end

				\subsubsection{Elastizitäten in der Robotik} % S.141, S.142, S.143, S.144, 5.50, 5.51
					\todo{Content}

					\paragraph{Ersatzmodell} % 5.52
						\todo{Content}
					% end

					\paragraph{Dynamikgleichungen} % 5.53, 5.54, 5.55, 5.56
						\todo{Content}
					% end
				% end

				\subsubsection{Berechnung von INV DYN bei Drehgelenkelastizitäten} % S.145, 5.57, 5.58
					\todo{Content}
				% end

				\subsubsection{Elastizitäten in der Biologie} % S.146, 5.59, 5.60
					\todo{Content}

					\paragraph{Menschlicher Bewegungsapparat} % 5.61, 5.62, 5.63
						\todo{Content}

						\subparagraph{Gelenkmodelle} % 5.64, 5.65
							\todo{Content}
						% end

						\subparagraph{Skelettmuskulatur} % 5.66
							\todo{Content}
						% end

						\subparagraph{Muskel-Sehnen-Komplex} % 5.67, 5.68, 5.69
							\todo{Content}
						% end
					% end

					\paragraph{Muskelaktivierungsdynamik} % 5.70
						\todo{Content}

						\subparagraph{Muskelmodell} % 5.71, 5.72, 5.73
							\todo{Content}
						% end

						\subparagraph{Hebelarme} % 5.74
							\todo{Content}
						% end
					% end

					\paragraph{Weichteilmodelle} % 5.75, 5.76
						\todo{Content}
					% end

					\paragraph{Dynamikmodell} % 5.77
						\todo{Content}

						\subparagraph{Dynamiksimulation} % 5.78, 5.79, 5.80, 5.81, 5.82, 5.83
							\todo{Content}
						% end
					% end

					\paragraph{Software und Daten} % 5.84
						\todo{Content}
					% end

					\paragraph{Einschränkungen} % 5.85
						\todo{Content}
					% end
				% end

				\subsubsection{Steuerung und Regelung bei Mensch und Tier} % 5.87, 5.88
					\todo{Content}

					\paragraph{Reafferenzprinzip} % 5.89, 5.90, 5.91
						\todo{Content}
					% end
				% end
			% end
		% end

		\section{Spezielle Dynamikmodelle für zweibeinige, humanoide Roboter und deren Stabilitätsregelung} % S.147, 5.97, 5.98, 5.99
			\todo{Content}

			\subsection{Stabilität zweibeiniges Laufen} % 5.100, 5.101, 5.102
				\todo{Content}
			% end

			\subsection{Dynamik im Roboterstandfuß} % 5.104, 5.105, 5.106, 5.107, 5.108
				\todo{Content}
			% end

			\subsection{Zero-Moment-Point} % S.148, S.149, S.149, S.150, 5.109
				\todo{Content}
			% end

			\subsection{Center of Pressure (CoP)} % 5.110
				\todo{Content}
			% end

			\subsection{ZMP Preview Walking} % 5.111, 5.112, 5.113, 5.114, 5.115, 5.116
				\todo{Content}
			% end

			\subsection{Globale Stabilitätsbegriffe} % S.156, S.157
				\todo{Content}
			% end

			\subsection{Ausblicke} % N/A
				\todo{Content}

				\subsubsection{Capture Steps} % 5.117, 5.118
					\todo{Content}
				% end

				\subsubsection{Whole Body Control} % 5.119, 5.120, 5.121, 5.122
					\todo{Content}
				% end

				\subsubsection{Unebenes Terrain} % 5.123
					\todo{Content}
				% end
			% end
		% end

		\section{Spezielle Dynamikmodelle für zweibeinige, nichthumanoide Roboter und deren Stabilitätsregelung} % 5.125, 5.126
			\todo{Content}

			\subsection{Inverses Pendel} % S.150, S.151, S.152, S.153, 5.127
				\todo{Content}
			% end

			\subsection{Erweitertes Modell des menschlichen Gehens und Rennens} % S.153, 5.128, 5.129
				\todo{Content}

				\subsubsection{Feder-Masse-Modell (Rennen)} % S.153, S.154, S.155
					\todo{Content}
				% end

				\subsubsection{Feder-Masse-Modell (Gehen)} % S.153, S.155
					\todo{Content}
				% end
			% end

			\subsection{Hüpfende Roboter mit Teleskop-Beinen} % 5.130, 5.131, 5.132, 5.138
				\todo{Content}

				\subsubsection{Ein Modell dynamischer Stabilität} % 5.134, 5.135, 5.136, 5.137
					\todo{Content}
				% end
			% end

			\subsection{Passive Dynamic Walkers} % 5.142, 5.143, 5.144, 5.145, 5.147, 5.148, 5.149
				\todo{Content}
			% end

			\subsection{Elastische Roboter} % 5.150, 5.151, 5.152, 5.153, 5.154, 5.160, 5.164
				\todo{Content}
			% end
		% end
	% end

	\chapter{Antriebssysteme} % 6.6
		\todo{Content}

		\section{Gebräuchliche Antriebssysteme} % 6.6
			\todo{Content}

			\subsection{Hydraulische Antriebe} % 6.7
				\todo{Content}
			% end

			\subsection{Pneumatische Antriebe} % 6.8
				\todo{Content}
			% end

			\subsection{Elektrische Antriebe} % 6.9
				\todo{Content}
			% end
		% end

		\section{DC-Motoren} % 6.10, 6.11, 6.12, 6.13
			\todo{Content}
		% end

		\section{Getriebe} % 6.14, 6.15, 6.16
			\todo{Content}

			\subsection{Gewinde} % 6.17
				\todo{Content}
			% end

			\subsection{Riemen-/Seilzug-Getriebe} % 6.18
				\todo{Content}
			% end

			\subsection{Rädergetriebe} % 6.19
				\todo{Content}

				\subsubsection{Stirnradgetriebe} % 6.20
					\todo{Content}
				% end

				\subsubsection{Planetengetriebe} % 6.21, 6.22, 6.23, 6.24
					\todo{Content}
				% end

				\subsubsection{Harmonic Drive Getriebe} % 6.25, 6.26, 6.27, 6.28
					\todo{Content}
				% end

				\subsubsection{Wittenstein Galaxie Getriebe} % 6.29
					\todo{Content}
				% end
			% end
		% end

		\section{Alternative und elastische Antriebskonzepte} % 6.30, 6.31, 6.32
			\todo{Content}

			\subsection{Beine} % 6.33, 6.34
				\todo{Content}
			% end

			\subsection{Neue Materialien} % 6.35, 6.36, 6.36, 6.37, 6.42
				\todo{Content}
			% end

			\subsection{Compliant Robot Actuation} % 6.38
				\todo{Content}
			% end

			\subsection{Elastische Antriebskonzepte} % 6.43, 6.44, 6.45
				\todo{Content}

				\subsubsection{Variable Stiffness Actuator} % 6.46
					\todo{Content}
				% end

				\subsubsection{Variable Impedance Actuators} % 6.47, 6.48, 6.49, 6.50, 6.51
					\todo{Content}
				% end
			% end

			\subsection{Vom Muskel-Skelett-Apparat inspirierte Roboter} % 6.52, 6.53
				\todo{Content}
			% end
		% end
	% end

	\chapter{Sensoren} % 6.54, 8.1
		\todo{Content}

		\section{Interne Sensoren} % 6.54, 6.55, 6.56
			\todo{Content}

			\subsection{Positionssensoren} % 6.56
				\todo{Content}

				\subsubsection{Potentiometer} % 6.57
					\todo{Content}
				% end

				\subsubsection{Optische Codierer} % 6.58
					\todo{Content}

					\paragraph{Inkrementell} % 6.59, 6.60
						\todo{Content}
					% end

					\paragraph{Absolut} % 6.61, 6.62
						\todo{Content}
					% end
				% end

				\subsubsection{Resolver} % 6.63
					\todo{Content}
				% end
			% end

			\subsection{Geschwindigkeitssensoren} % 6.64
				\todo{Content}
			% end

			\subsection{Beschleunigungssensoren} % 6.65
				\todo{Content}

				\subsubsection{Silizium-Beschleunigungssensor} % 6.66
					\todo{Content}
				% end
			% end

			\subsection{Inertial Navigation System (INS)} % 6.67, 6.71
				\todo{Content}

				\subsubsection{Gyroskop} % 6.68, 6.69, 6.70
					\todo{Content}
				% end

				\subsubsection{Mikromechanische Gyroskope} % 6.72
					\todo{Content}
				% end

				\subsubsection{Drift- und Fehlerkompension} % 6.73
					\todo{Content}
				% end
			% end

			\subsection{Kraft-Momenten-Sensor} % 6.74, 6.75, 6.76, 6.77, 6.78
				\todo{Content}
			% end
		% end

		\section{Externe und intelligente Sensoren} % 8.1, 8.2, 8.3, 8.4, 8.5
			\todo{Content}

			\subsection{Abstandssensoren} % 8.6, 8.7
				\todo{Content}

				\subsubsection{Akustische Abstandssensoren} % 8.8, 8.17
					\todo{Content}

					\paragraph{Ultraschall} % 8.9, 8.10, 8.11
						\todo{Content}
					% end

					\paragraph{Messsituationen} % 8.12, 8.13
						\todo{Content}
					% end

					\paragraph{Abhilfe der Schwierigkeiten} % 8.14
						\todo{Content}
					% end

					\paragraph{Natur} % 8.15, 8.16
						\todo{Content}
					% end
				% end

				\subsubsection{Optische Abstandssensoren} % 8.18, 8.19
					\todo{Content}

					\paragraph{Laufzeitermittlung} % 8.20, 8.21
						\todo{Content}
					% end

					\paragraph{2D-Laserscanner} % 8.22, 8.23, 8.24, 8.25
						\todo{Content}
					% end
				% end
			% end

			\subsection{Visuelle Sensoren} % 8.26, 8.27, 8.28, 8.29, 8.30, 8.31
				\todo{Content}

				\subsubsection{Bilderzeugung} % 8.32, 8.33, 8.34, 8.35
					\todo{Content}
				% end

				\subsubsection{Grundlagen} % 8.38, 8.39, 8.40, 8.41, 8.42, 8.43
					\todo{Content}
				% end

				\subsubsection{Bildvorverarbeitung} % 8.44, 8.45, 8.46
					\todo{Content}

					\paragraph{Nachbarschaftsbildverarbeitung} % 8.47, 8.48, 8.49, 8.50
						\todo{Content}
					% end
				% end

				\subsubsection{Bildverarbeitung} % 8.51, 8.52, 8.53
					\todo{Content}

					\paragraph{Kantendetektion} % 8.54, 8.55, 8.56, 8.57
						\todo{Content}

						\subparagraph{Prewitt-Operator} % 8.58, 8.59, 8.60, 8.61
							\todo{Content}
						% end

						\subparagraph{Zusammenfassung} % 8.62, 8.63
							\todo{Content}
						% end
					% end

					\paragraph{Konturerkennung} % 8.64
						\todo{Content}

						\subparagraph{Lokale Analyse} % 8.65, 8.66
							\todo{Content}
						% end

						\subparagraph{Globale Analyse} % 8.67, 8.68, 8.69, 8.70
							\todo{Content}
						% end
					% end

					\paragraph{Segmentierung} % 8.71, 8.72
						\todo{Content}

						\subparagraph{Schwellwertabfrage} % 8.73, 8.74
							\todo{Content}
						% end

						\subparagraph{Bereichswachstum} % 8.75
							\todo{Content}
						% end
					% end
				% end

				\subsubsection{Merkmalsextraktion} % 8.76, 8.77, 8.78, 8.79
					\todo{Content}
				% end

				\subsubsection{Objektklassifikation} % 8.80, 8.81, 8.82, 8.83, 8.84, 8.85, 8.86, 8.87, 8.88, 8.89, 8.90, 8.91, 8.92
					\todo{Content}
				% end
			% end

			\subsection{3D-Sensoren und Perzeption} % 8.93, 8.94, 8.95, 8.96
				\todo{Content}

				\subsubsection{Bildbasierte 3D-Sensoren} % 8.97, 8.98
					\todo{Content}

					\paragraph{Stereo Vision} % 8.99, 8.100, 8.101, 8.102, 8.103, 8.104, 8.105, 8.106
						\todo{Content}
					% end

					\paragraph{Structured Light Kamera} % 8.107, 8.108, 8.109, 8.110, 8.111, 8.112, 8.113, 8.114
						\todo{Content}
					% end

					\paragraph{Stereo Vision + Structured Light} % 8.115, 8.116, 8.117
						\todo{Content}
					% end

					\paragraph{Time-of-Flight (ToF) Kamera} % 8.118, 8.119, 8.120, 8.121, 8.122, 8.123, 8.124, 8.125
						\todo{Content}
					% end
				% end

				\subsubsection{Laserbasierte 3D-Sensoren} % 8.126, 8.127, 8.128, 8.129
					\todo{Content}
				% end

				\subsubsection{Datenstrukturen und -Repräsentation} % 8.130, 8.131
					\todo{Content}

					\paragraph{kd-Baum} % 8.132, 8.133
						\todo{Content}
					% end

					\paragraph{Octree} % 8.134, 8.135, 8.136
						\todo{Content}
					% end

					\paragraph{Truncated Signed Distance Fields (TSDF)} % 8.137, 8.138
						\todo{Content}
					% end
				% end

				\subsubsection{Point-Cloud Processing} % 8.139, 8.140, 8.141, 8.142, 8.143, 8.144, 8.145, 8.146, 8.147
					\todo{Content}
				% end

				\subsubsection{Point-Cloud Registration} % 8.148, 8.149, 8.150, 8.151, 8.152, 8.153, 8.154
					\todo{Content}
				% end
			% end
		% end
	% end

	\chapter{Regelung} % 6.1, S.158, S.159, S.160
		\todo{Content}

		\section{Lineare Regelung} % 6.1, 6.2, 6.3, 6.4, 6.5, 6.79, 6.85, 6.86, 6.87, S.160
			\todo{Content}

				\subsubsection{Begriffe} % 6.80, 6.81, 6.82, 6.83, 6.84, 6.88, 6.89, 6.90
					\todo{Content}
				% end
			% end

			\subsection{Lineare Systemdynamik und Feder-Masse-System} % 6.91, 6.92, S.161
				\todo{Content}

				\subsubsection{Untersuchung des Bewegungsverhaltens} % 6.93, S.161, S.162, S.163
					\todo{Content}

					\paragraph{Erwartetes Bewegungsverhalten} % 6.94, 6.95, 6.96, 6.97, 6.98
						\todo{Content}
					% end

					\paragraph{Gewünschtes Bewegungsverhalten} % 6.99
						\todo{Content}
					% end
				% end
			% end

			\subsection{PD-Regelung linearer Systeme 2. Ordnung} % 6.100, 6.101, 6.102, S.163, S.164
				\todo{Content}

				\subsubsection{Feder-Masse-System} % 6.103, 6.104, 6.105, 6.106
					\todo{Content}
				% end
			% end

			\subsection{Partinionierung des Regelgesetzes durch Feedback-Linearisierung} % 6.107, 6.108, S.165, S.166
				\todo{Content}

			\subsection{Sollwerttrajektorien-Folgeregelung} % 6.109, 6.110, S.166, S.167
				\todo{Content}
			% end

			\subsection{PID-Regelung linearer Systeme} % 6.111, 6.112, 6.115, S.167, S.168
				\todo{Content}

				\subsubsection{Windup-Effekt} % 6.113, 6.114
					\todo{Content}
				% end
			% end

			\subsection{Kaskadenregelung} % 6.116, 6.117, 6.118, 6.119, 6.120
				\todo{Content}

				\subsubsection{PID-Bahnregelung eines Servomotors} % 6.121, 6.122
					\todo{Content}
				% end

				\subsubsection{Laufbewegungen} % 6.123, 6.124, 6.125, 6.126, 6.127, 6.128, 6.129, 6.130
					\todo{Content}
				% end
			% end

			\subsection{Stabilität als Sprungantwortverhalten und PID-Parameter} % 6.131, 6.132, 6.133
				\todo{Content}

				\subsubsection{Drehmotor} % 6.134, 6.135, 6.136, 6.137, 6.138, 6.139, 6.140, 6.141
					\todo{Content}
				% end

				\subsubsection{Einstellung der PID-Parameter} % 6.142, 6.143
					\todo{Content}
				% end
			% end

			\subsection{Digitale Implementierung eines PID-Reglers} % 6.144, 6.145
				\todo{Content}
			% end
		% end

		\section{Nichtlineare Regelung} % 6.147, 6.148
			\todo{Content}

			\subsection{Systemlinearisierung} % 6.149
				\todo{Content}
			% end

			\subsection{Modellbasierte Manipulatorregelung} % 6.150, 6.151, 6.152
				\todo{Content}
			% end

			\subsection{Adaptive Manipulatorregelung} % 6.153
				\todo{Content}
			% end

			\subsection{Bahnreglung in Weltkoordinaten} % 6.154
				\todo{Content}
			% end
		% end

		\section{Kraft-/Momenten-Regelung} % 6.155, 6.156, 6.157, 6.158, 6.159, 6.160, 6.161, 6.162
			\todo{Content}
		% end

		\section{Bahn-/Kraft-Regelung} % 6.163
			\todo{Content}

			\subsection{Hybride Regelung} % 6.163, 6.164, 6.165
				\todo{Content}
			% end

			\subsection{Parallele Regelung} % 6.166, 6.167
				\todo{Content}
			% end
		% end

		\section{Nachgiebigkeitsregelung (Compliant Control)} % 6.168, 6.169, 6.170, 6.171
			\todo{Content}

			\subsection{Verallgemeinerte Betrachtung nach Hogan} % 6.172, 6.173
				\todo{Content}
			% end

			\subsection{Impedanzregelung} % 6.174, 6.175
				\todo{Content}

				\subsubsection{Greifen} % 6.179, 6.180, 6.181, 6.182, 6.183, 6.184
					\todo{Content}
				% end
			% end

			\subsection{Admittanzregelung} % 6.185, 6.186, 6.187, 6.188, 6.191, 6.192
				\todo{Content}

				\subsubsection{Implementierung} % 6.189, 6.190
					\todo{Content}
				% end
			% end

			\subsection{Aktiv-passive Konzepte für Impedanz/Admittanz} % 6.193, 6.194
				\todo{Content}
			% end
		% end

		\section{Bildgeführte Regelung} % 6.195, 6.196, 6.197, 
			\todo{Content}

			\subsection{Positionsbasiert} % 6.198
				\todo{Content}
			% end

			\subsection{Bildbasiert} % 6.199, 6.200
				\todo{Content}
			% end
		% end

		\section{Multimodale Regelung physikalischer Interaktionen} % 6.202, 6.203, 6.204, 6.205, 6.206
			\todo{Content}
		% end

		\section{Regelung und Steuerung bei Mensch und Tier} % 6.207, 6.208, 6.228
			\todo{Content}

			\subsection{Propriozeption} % 6.209
				\todo{Content}
			% end

			\subsection{Sensoren} % 6.210
				\todo{Content}
			% end

			\subsection{Zentrales Nervensystem} % 6.211, 6.212, 6.213, 6.214, 6.215, 6.216, 6.217
				\todo{Content}
			% end

			\subsection{Neurale Integration} % 6.218, 6.219
				\todo{Content}
			% end

			\subsection{Informationsverarbeitung} % 6.220
				\todo{Content}

				\subsubsection{Reflexe} % 6.221, 6.222, 6.223
					\todo{Content}
				% end
			% end

			\subsection{Sonstiges} % 6.224, 6.225, 6.226, 6.227
				\todo{Content}
			% end
		% end

		\section{Elementare Roboterbewegungen} % 6.229
			\todo{Content}

			\subsection{Elementare Bewegungsarten für Industrieroboter} % 6.230, 6.231
				\todo{Content}

				\subsubsection{Lineare Interpolation in Gelenkkoordinaten} % 6.232, 6.233, 6.234
					\todo{Content}
				% end

				\subsubsection{Lineare Interpolation in Weltkoordinaten} % 6.235
					\todo{Content}
				% end

				\subsubsection{Kreisbogen-Interpolation} % 6.236
					\todo{Content}
				% end

				\subsubsection{Spline-Interpolation} % 6.236
					\todo{Content}
				% end
			% end

			\subsection{Schwierigkeiten bei kartesischer Bahnvorgabe} % 6.237, 6.238
				\todo{Content}
			% end

			\subsection{Programmierung einer Bahn als Folge elementarer Bewegungen} % 6.239, 6.240, 6.241, 6.242, 6.243
				\todo{Content}
			% end

			\subsection{Elementare Bewegungsarten für fahrende Roboter} % 6.244, 6.245, 6.246
				\todo{Content}
			% end

			\subsection{Elementare Bewegungsarten für laufende Roboter} % 6.247, 6.248, 6.249, 6.250, 6.251, 6.252, 6.253, 6.254, 6.255, 6.256, 6.257, 6.258
				\todo{Content}
			% end
		% end
	% end

	\chapter{Bahnplanung} % 7.1, 7.2, 7.3, 7.4, 7.5, 7.6, 7.7, 7.8
		\todo{Content}

		\section{Bahnplanungsarten} % 7.9, 7.10
			\todo{Content}
		% end

		\section{Topologische Wegeplanung} % 7.11, 7.12, 7.13, 7.14
			\todo{Content}
		% end

		\section{Konfigurationsraum} % 7.15, 7.16, 7.17, 7.18, 7.19, 7.20, 7.21, 7.22, 7.23, 7.24
			\todo{Content}
		% end

		\section{Geometrische Bahnplanung} % 7.25, 7.26, 7.27
			\todo{Content}

			\subsection{Metrische Darstellung} % 7.28, 7.29
				\todo{Content}
			% end

			\subsection{Roadmap-Verfahren} % 7.30, 7.31
				\todo{Content}

				\subsubsection{Sichtbarkeitsgraph} % 7.32, 7.33
					\todo{Content}
				% end

				\subsubsection{Tangentengraph} % 7.34, 7.35
					\todo{Content}
				% end

				\subsubsection{Voronoi-Diagramme} % 7.36, 7.37, 7.38
					\todo{Content}
				% end
			% end

			\subsection{Exakte Zellzerlegung} % 7.39, 7.40, 7.41, 7.42, 7.43, 7.44, 7.45
				\todo{Content}
			% end

			\subsection{Approximative Zellzerlegung} % 7.46, 7.47, 7.48, 7.49
				\todo{Content}
			% end

			\subsection{Potentialfeld-Methoden} % 7.50, 7.51, 7.52, 7.53, 7.54, 7.55, 7.56
				\todo{Content}
			% end

			\subsection{Komplexität} % 7.57, 7.58, 7.59
				\todo{Content}
			% end

			\subsection{Stichprobenverfahren} % 7.61, 7.62, 7.63, 7.64, 7.65
				\todo{Content}
			% end

			\subsection{Rapidly Exploring Random Trees (RRTs)} % 7.66
				\todo{Content}
			% end

			\subsection{Beispiele} % 7.67, 7.68, 7.69
				\todo{Content}

				\subsubsection{MINERVA} % 7.70
					\todo{Content}

					\paragraph{Umweltmodell: Belegungskarte} % 7.71, 7.72, 7.73, 7.74
						\todo{Content}
					% end

					\paragraph{"Küstennahe" Bahnplanung} % 7.75
						\todo{Content}
					% end

					\paragraph{Kollisionsvermeidung} % 7.76, 7.77
						\todo{Content}
					% end
				% end
			% end
		% end

		\section{Kinematische und dynamische Trajektorienplanung} % 7.78, 7.79, 7.80, 7.81, 7.82
			\todo{Content}

			\subsection{Allgemeine Formulierung} % 7.83, 7.84, 7.85
				\todo{Content}
			% end

			\subsection{Beispiel: Industrieroboter} % 7.86, 7.87, 7.88
				\todo{Content}
			% end
		% end
	% end

	\chapter{Navigation mobiler Roboter} % 9.1, 9.2, 9.3, 9.4
		\todo{Content}

		\section{Lokalisierung und Positionierung} % 9.5
			\todo{Content}

			\subsection{Relative Positionsbestimmung} % 9.6
				\todo{Content}
			% end

			\subsection{Absolute Positionsbestimmung} % 9.7, 9.8, 9.9
				\todo{Content}

				\subsubsection{Nichtlineare Ausgleichsrechnung} % 9.10
					\todo{Content}
				% end

				\subsubsection{Erkennung künstlicher Landmarken} % 9.11, 9.12, 9.13, 9.14
					\todo{Content}
				% end

			\subsection{Stochastische Positionsbestimmung} % 9.15, 9.15, 9.16
				\todo{Content}
			% end
		% end

		\section{Selbstlokalisierung und Navigation} % 9.17, 9.18, 9.19, 9.20
			\todo{Content}

			\subsection{Metrische Beschreibung des Aufenthaltsortes} % 9.21, 9.22
				\todo{Content}
			% end

			\subsection{Topologische Beschreibung des Aufenthaltsortes} % 9.23
				\todo{Content}
			% end

			\subsection{Lokalisierung mit einer Hypothese} % 9.24, 9.25
				\todo{Content}

				\subsubsection{Koppelnavigation} % 9.26
					\todo{Content}
				% end

				\subsubsection{(Erweitertes) Kalman-Filter} % 9.27, 9.28, 9.29, 9.30, 9.33, 9.37
					\todo{Content}

					\paragraph{Time-Update} % 9.31, 9.32
						\todo{Content}
					% end

					\paragraph{Measurement-Update} % 9.34, 9.35, 9.36
						\todo{Content}
					% end
				% end
			% end

			\subsection{Lokalisierung mit mehreren Hypothesen} % 9.41
				\todo{Content}

				\subsubsection{Verwendung mehrerer Kalman-Filter} % 9.42, 9.43, 9.44, 9.45, 9.46, 9.47, 9.48, 9.49
					\todo{Content}
				% end

				\subsubsection{Diskretisierte Wahrscheinlichkeitsverteilung} % 9.50, 9.51, 9.52, 9.53, 9.54, 9.55, 9.56, 9.57, 9.58, 9.59, 9.60
					\todo{Content}
				% end

				\subsubsection{Monte-Carlo Lokalisierung} % 9.61, 9.62, 9.63, 9.64, 9.65, 9.66, 9.67, 9.68, 9.69, 9.70, 9.71, 9.72, 9.73
					\todo{Content}
				% end
			% end

			\subsection{Simultaneous Localization and Mapping (SLAM)} % 9.74, 9.75, 9.76, 9.77, 9.78, 9.79
				\todo{Content}

				\subsubsection{EKF SLAM} % 9.80, 9.81, 9.82, 9.83, 9.84
					\todo{Content}
				% end

				\subsubsection{FastSLAM} % 9.85, 9.86, 9.87, 9.88
					\todo{Content}
				% end

				\subsubsection{Graph-basiertes SLAM} % 9.89
					\todo{Content}
				% end

				\subsubsection{Limitierungen} % 9.90, 9.91
					\todo{Content}
				% end

				\subsubsection{Visual SLAM} % 9.92, 9.93, 9.94, 9.95
					\todo{Content}
				% end

				\subsubsection{Team HECTOR} % 9.96, 9.97, 9.98, 9.99, 9.100, 9.101, 9.102, 9.103
					\todo{Content}
				% end

				\subsubsection{Google Cartographer} % 9.104, 9.105, 9.106
					\todo{Content}
				% end

				\subsubsection{3D SLAM} % 9.107, 9.108, 9.109, 9.110, 9.111, 9.112, 9.113
					\todo{Content}
				% end
			% end
		% end
	% end

	\chapter{Middleware und Simulation} % 10.1, 10.2, 10.3, 10.4, 10.5, 10.6
		\todo{Content}

		\section{Szenarien, Eigenschaften und Herausforderungen} % 10.7, 10.8, 10.9, 10.10, 10.11
			\todo{Content}
		% end

		\section{Middleware} % 10.12, 10.13, 10.14, 10.15
			\todo{Content}

			\subsection{Nachrichtenbasierte Kommunikation} % 10.16, 10.17
				\todo{Content}
			% end

			\subsection{Laufzeit-Effizienz} % 10.18, 10.19
				\todo{Content}
			% end
		% end

		\section{Sicherstellung von Korrektheit und Zuverlässigkeit} % 10.20, 10.21, 10.22, 10.25, 10.26
			\todo{Content}

			\subsection{Simulation} % 10.27
				\todo{Content}
			% end

			\subsection{Automatisierte Testabläufe} % 10.30, 10.31, 10.32
				\todo{Content}
			% end

			\subsection{Monitoring} % 10.33
				\todo{Content}
			% end

			\subsection{Visuelles Debuggen} % 10.34, 10.35
				\todo{Content}
			% end

			\subsection{Offline Analyse} % 10.36, 10.37
				\todo{Content}
			% end

			\subsection{Kooperierendes Verhalten} % 10.38, 10.39, 10.40
				\todo{Content}
			% end
		% end
	% end

	\chapter{Steuerung autonomer Roboter} % 11.1, 11.2
		\todo{Content}

		\section{Steuerungsarchitekturen} % 11.3
			\todo{Content}

			\subsection{Hierarchisches Steuerungsparadigma} % 11.4, 11.5, 11.6
				\todo{Content}
			% end

			\subsection{Reaktives Steuerungsparadigma} % 11.7, 11.8
				\todo{Content}
			% end

			\subsection{Hybrid deliberativ-reaktives Steuerungsparadigma} % 11.9, 11.10, 11.11
				\todo{Content}
			% end

			\subsection{Beispiel: Subsumption Architecture} % 11.12, 11.13, 11.14, 11.18
				\todo{Content}

				\subsubsection{Schicht 0} % 11.15
					\todo{Content}
				% end

				\subsubsection{Schicht 1} % 11.16
					\todo{Content}
				% end

				\subsubsection{Schicht 2} % 11.17
					\todo{Content}
				% end
			% end
		% end

		\section{Programmierung von Verhalten} % 11.19, 11.20, 11.21, 11.22, 11.23
			\todo{Content}

			\subsection{XABSL} % 11.24, 11.25, 11.26, 11.27, 11.28, 11.29, 11.30
				\todo{Content}
			% end

			\subsection{FlexBE} % 11.31, 11.32
				\todo{Content}

				\subsubsection{Autonomy Level} % 11.33
					\todo{Content}
				% end

				\subsubsection{Beispiel: TurtleBot3} % N/A
					\todo{Content}

					\paragraph{Move Arm} % 11.34, 11.35, 11.36, 11.37, 11.38, 11.39, 11.40, 11.41, 11.42, 11.43, 11.44, 11.45
						\todo{Content}
					% end

					\paragraph{Move Base} % 11.46, 11.47, 11.48
						\todo{Content}
					% end
				% end
			% end
		% end
	% end

	\chapter{Verantwortung der Informatik und der Ingenieurwissenschaften} % 12.1
		\todo{Content}

		\section{Zivilklausel der TU Darmstadt} % 12.2
			\todo{Content}
		% end

		\section{Beispiele} % N/A
			\todo{Content}

			\subsection{Kugelschreiber} % 12.3
				\todo{Content}
			% end

			\subsection{DARPA Robotics Challenge} % 12.4, 12.5, 12.6, 12.7, 12.8
				\todo{Content}
			% end

			\subsection{Roboter und Arbeitsplätze} % 12.9, 12.10, 12.11, 12.12, 12.13
				\todo{Content}
			% end
		% end

		\section{Sonstiges} % 12.14, 12.15, 12.16, 12.17
			\todo{Content}
		% end
	% end





	\appendix

	\chapter{Quaternionen} % S.176
		\todo{Content}

		\section{Einleitung} % S.176
			\todo{Content}
		% end

		\section{Rechenregeln} % S.176
			\todo{Content}
		% end

		\section{Umrechnung: Quaternionen \(\leftrightarrow\) Rotationsmatrizen} % S.177
			\todo{Content}
		% end

		\section{Verkettung von Drehungen} % S.177
			\todo{Content}
		% end

		\section{Repräsentation der Koordinaten eines Punktes bei Rotation} % S.177, S.178
			\todo{Content}
		% end

		\section{Vergleich mit anderen Darstellungsarten} % S.179
			\todo{Content}
		% end
	% end

	\chapter{Zusammenhang zwischen Rotationsmatrix, Drehvektor und Drehwinkel} % S.179
		\todo{Content}

		\section{Drehvektor und Drehwinkel \(\to\) Rotationsmatrix} % S.179
			\todo{Content}
		% end

		\section{Rotationsmatrix \(\to\) Drehvektor und Drehwinkel} % S.179, S.180, S.181
			\todo{Content}
		% end
	% end

	\chapter{Notationen} % S.183, S.184
		\todo{Content}
	% end
\end{document}
