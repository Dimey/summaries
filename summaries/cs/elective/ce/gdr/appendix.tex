\chapter{Quaternionen} % S.176
	\todo{Quaternionen}

	\section{Einleitung} % S.176
		\todo{Quaternionen}
	% end

	\section{Rechenregeln} % S.176
		\todo{Quaternionen}
	% end

	\section{Umrechnung: Quaternionen zu Rotationsmatrizen und zurück} % S.177
		\todo{Quaternionen}
	% end

	\section{Verkettung von Drehungen} % S.177
		\todo{Quaternionen}
	% end

	\section{Repräsentation der Koordinaten eines Punktes bei Rotation} % S.177, S.178
		\todo{Quaternionen}
	% end

	\section{Vergleich mit anderen Darstellungsarten} % S.179
		\todo{Quaternionen}
	% end
% end

\chapter{Zusammenhang zwischen Rotationsmatrix, Drehvektor und Drehwinkel} % S.179
	\todo{Rotationsmatrix, Drehvektor und Drehwinkel}

	\section{Umwandlung von Drehvektor und Drehwinkel zu Rotationsmatrix} % S.179
		\todo{Rotationsmatrix, Drehvektor und Drehwinkel}
	% end

	\section{Umwandlung von Rotationsmatrix zu Drehvektor und Drehwinkel} % S.179, S.180, S.181
		\todo{Rotationsmatrix, Drehvektor und Drehwinkel}
	% end
% end

\chapter{Notationen}
	Typische Symbole und ihre Bedeutung:
	\begin{itemize}
		\item Kleine und kursive Buchstaben bezeichnen skalare Größen, \zB \( t, x, f, g, h \).
		\item Große, fettgedruckte und kursive Buchstaben bezeichnen Matrizen, \zB \( \mat{A}, \mat{B}, \mat{R}, \mat{T} \).
		\item Fettgedruckte Buchstaben bezeichnen Vektoren, \zB \( \vec{x}, \vec{q}, \vec{F} \).
	\end{itemize}
	Tabelle~\ref{tab:notation} zeigt typische Buchstaben- und Symbolzuordnungen und Notationen.

	\begin{table}
		\centering
		\begin{tabular}{rl}
			\multicolumn{2}{l}{\textbf{Allgemein}}                                                                       \\
			                         \(f\), \(g\), \(h\) & (skalare) Funktionsnamen                                      \\
			                                       \(t\) & Zeit                                                          \\
			                            \(t_a\), \(t_s\) & Startzeitpunkt                                                \\
			                            \(t_e\), \(t_f\) & Endzeitpunkt                                                  \\
			                         \(x\), \(y\), \(z\) & Koordinatenachsen                                             \\
			                         \(i\), \(j\), \(k\) & Laufindex                                                     \\
			                                       \(l\) & Länge                                                         \\
			                                       \(m\) & Masse                                                         \\
			                                    \(\rho\) & Dichte                                                        \\
			                                       \(g\) & Gravitationskonstante                                         \\
			                                  \(\delta\) & kleiner Abstand                                               \\
			     \( \dv{f}{x} = f'(x) \) & (totale) Ableitung nach der unabhängigen Variablen \(x\)      \\
			\( \dv{f}{t} = \dot{x}(t) \) & (totale) Ableitung nach der unabhängigen Variablen Zeit \(t\) \\&\\
		\multicolumn{2}{l}{\textbf{Räumliche Darstellungen und Transformationen}} \\
			                \(\vec{p}\) & Koordinatenvektor eines Punktes \\
			                 \(\theta\) & Winkel                          \\
			\((\alpha, \beta, \gamma)\) & Winkeltripel                    \\
			\((\psi, \theta, \varphi)\) & Winkeltripel                    \\
			                \(\mat{E}\) & Einheitsmatrix                  \\
			                \(\mat{R}\) & Rotationsmatrix                 \\
			                \(\vec{r}\) & Translationsvektor              \\
			                \(\mat{T}\) & homogene Transformationsmatrix \\&\\
		\multicolumn{2}{l}{\textbf{Manipulatorkinematik}} \\
			\(S\) & Koordinatensystem \\
			\(n\) & Anzahl der Freiheitsgrade \\
			\( \vec{q} = \begin{bmatrix} q_1 & \cdots & q_n \end{bmatrix}^T \) & Zustandsvektor der Gelenkvariablen \\&\\
		\multicolumn{2}{l}{\textbf{Geschwindigkeit, Jacobi-Matrix und statische Kräfte}} \\
			\(\vec{\omega}\) & Winkelgeschwindigkeitsvektor    \\
			     \(\vec{v}\) & linearer Geschwindigkeitsvektor \\
			     \(\mat{J}\) & Jacobi-Matrix \\&\\
		\multicolumn{2}{l}{\textbf{Manipulatordynamik}} \\
			             \(\mat{I}\) & Trägheitstensor                                      \\
			            \(n\), \(N\) & (skalares) Drehmoment                                \\
			            \(f\), \(F\) & (skalare) Kraft                                      \\
			\(\tau\), \(\vec{\tau}\) & (skalare oder vektorielle) Kraft oder Drehmoment     \\
			             \(\mat{M}\) & Massenmatrix                                         \\
			             \(\vec{C}\) & Vektorfunktion der Zentrifugal- oder Coriolisanteile \\
			             \(\vec{G}\) & Vektorfunktion der Gravitationsanteile               \\
			                   \(K\) & kinetische Energie                                   \\
			                   \(P\) & potentielle Energie
		\end{tabular}
		\caption{Notationen}
		\label{tab:notation}
	\end{table}
% end
